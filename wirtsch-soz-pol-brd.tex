\section{Wirtschafts- und Sozialpolitik der BRD 1950\,--1979}

Tabelle \ref{tab:entw-brd} auf Seite \pageref{tab:entw-brd} bietet
einen Überblick über die ökonomische Entwicklung und sozialpolitische
Maßnahmen in der BRD.

\mar{Kohle und Stahl weniger oder mehr beim Strukturwandel in der
Tabelle?}

\begin{table}
\caption{Entwicklung der Bundesrepublik von 1950 bis 1979}
\label{tab:entw-brd}

\footnotesize
\newlength{\szheight}
\settoheight{\szheight}{ß}
\newlength{\wirtschw}
\settowidth{\wirtschw}{\emph{Wirtschaftswunder}}
\begin{tabularx}{\textwidth}{p{\wirtschw}XX}
\toprule
& 
ökonomische Entwicklung
&
sozialpolitische Entwicklung \\
\midrule

1950er Jahre \newline
\emph{Wirtschaftswunder}\index{Wirtschaftswunder}
&
\vspace{-\szheight}
\begin{tablist}
\item außergewöhnliche Zuwachsrate bei Produktion und Produktivität
\item Einführung der sozialen Marktwirtschaft durch \Nam{Erhard,
Ludwig}{Ludwig Erhard}
\item Integration vvon Flüchtlingen und Vertrreibenen
\item nahezu Vollbeschäftigung -- Einsatz von Gastarbeitern
\index{Gastarbeiter}
\end{tablist}
&
\vspace{-3.38ex}
\begin{chronik}
\item[1949] \ges{Tarifvertragsgesetz}
\item[1950] \ges{Bundesversorgungsgesetz für Kriegsopfer und
Hinterbliebene}
\item[1952] \ges{Betriebsverfassungsgesetz}
\item[1957] \ges{Kindergeldgesetz} zum Ausgleich finanzieller
Belastungen von Familien
\item[1957] Rentenreform und Alterssicherung für Landwirte 
\end{chronik}
\\

1960\,--1973 \\ Normalisierung &
\vspace{-5.85ex}
\begin{tablist}
\item Verlangsamung des Wirtschaftswachstums
\item kräftige Nachfrage, hohe Produktionszuwächse, Export,
Lohnexpansion, geringe Inflation, niedrige Arbeitslosigkeit bei
schrumpfendem Arbeitskräfteangebot 
\end{tablist}
&
\vspace{-7.65ex}
\begin{chronik}
\item[1961] \ges{Bundessozialhilfegesetz}
\item[1963] \ges{Un"-fall"-ver"-si"-che"-rungs"-neu"-re"-ge"-lungs"-ge"-se"-tz}
\item[1968] \ges{Finanzänderungsgesetz}
\item[1969] \ges{Arbeitsförderungsgesetz}
\end{chronik}
\\
 
1970er Jahre \newline Krisenjahre &
\vspace{-\szheight}
\begin{tablist}
\item wirtschaftssteuernde Maßnahmen des Staates
\item drastische Ölpreiserhöhung der Ölförderungsstaaten 1973
$\Rightarrow$ Energiekrise
\item geringes bis negatives Wirtschaftswachstum
\item Stagnation der Produktion, Strukturwandel (Kohle, Stahl)
\item Inflation, Anstieg der Arbeitslosigkeit ab 1975
\item strukturelles Defizit der öffentlichen Haushalte trotz hohem
Konsumniveau (Massenwohlstand $\blacktriangleright\blacktriangleleft$
öffentliche Verschuldung)
\end{tablist}
&
\vspace{-\szheight}
\begin{tablist}
\item soziale Einschnitte und Kürzungen 
\item dennoch Familien- und Frauenpolitik
\item Rentenreform 1972
\item Kindergeld für alle Kinder 1975
\end{tablist}
\\
\bottomrule
\end{tabularx} 
\end{table}

%%%%%%%%%%%%%%%%%%%%%%%%%%%%%%%%%%%%%%%%%%%%%%%%%%%%%%%%%%%%%%%%%%%%%%

\subsection*{Ursachen für den wirtschaftlichen Aufstieg
der BRD in den Fünfzigern}

\begin{itemize}
\item nahezu vollständig erhaltene Produktionskapazität nach dem
Zweiten Weltkrieg
\item Neuinvestitionen durch \ges{Marshall-Plan}
\item Erhöhung des Konsumangebotes reizt die Produktion an\mar{Hä?}
\item großes Angebot an gut ausgebildeten Arbeitskräften
\item großer Güterbedarf nach dem Zweiten Weltkrieg
\item konsequente Umsetzung der \Beg{Soziale Marktwirtschaft}{Sozialen
Marktwirtschaft}
\item ab 1958 Integration der westdeutschen Märkte\mar{In die
europäische Wirtschaftsordnung, oder was?}
\end{itemize}


\subsection*[Wirtschafts- und Europapolitik]{Wirtschafts- und
Europapolitik\mycite[345\,f., 362\,f.]{WeltgeschNeuz}}
\mar{Rapacki}

\begin{chronik}
\item[9. Mai 1950] \ges{Schuman-Plan}\Nam{Schuman, Robert}{}: Vorschlag für einen gemeinsamen deutsch-fran"-zö"-si"-sche 
Kohle- und Stahlproduktion

\item[18. April 1951] Gründung der \Ins{EGKS, Europäische Gemeinschaft
für Kohle und Stahl}{Europäischen Gemeinschaft für Kohle und Stahl}
(auch \ins{Montanunion} -- zwischen Belgien, Frankreich, der
Bundesrepublik Deutschland, Italien, Luxemburg und den Niederlanden)
und Eingliederung derer in den Welthandel

\item[25. März 1957] Unterzeichnung der \Ges{Römische
Verträge}{Römischen Verträge} durch die
Mitgliedsstaaten der EGKS
\begin{itemize}
\item Gründung der \Ins{EWG, Europäische
Wirtschaftsgemeinschaft}{Europäischen Wirtschaftsgemeinschaft} --
Herstellung eines einheitlichen Wirtschaftsraums mit freiem
Warenverkehr binnen zwölf Jahren
\item Gründung der \Ins{EURATOM, Europäische
Atomgemeinschaft}{Europäischen Atomgemeinschaft} -- Bündelung der
Kräfte zur Forschung und friedlichen Nutzung von Kernenergie
\end{itemize}
\end{chronik}


\subsection*{Bewertung anhand des modernen Demokratiebegriffs}

Die Wirtschafts- und Sozialpolitik der BRD in den Jahren 1950 bis 1979
genügt dem modernen Demokratiebegriff: Trotz Krisen blieben die
Ausgaben für soziale Maßnahmen konstant hoch wie es das Grundgesetz
fordert. Gleichzeitig wahrt das System der Sozialen Marktwirtschaft die
wirtschaftliche Freiheit (Vertrags-, Konsumfreiheit, freie
Preisgestaltung, Privateigentum) und Antimonopolpolitik sorgt für
fairen Wettbewerb.

\endinput
