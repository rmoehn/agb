\section{Anspruch und Wirklichkeit einer parlamentarischen Demokratie}

\begin{aufgabe}
Überprüfen Sie, iweiweit die Mittel der parlamentarischen Demokratie
gegen die \Ins{RAF, Rote Armee Fraktion}{RAF} im Sinner der Verfassung
und der Demokratie waren!
\end{aufgabe}

Wie schon im Abschnitt über die RAF (siehe \ref{sec:raf}) besprochen,
waren die ergriffenen Maßnahmen teilweise undemokratisch. Auch die
\ins{Spiegel}-Affäre (siehe \ref{sec:spiegaff}) war es. Aber dies
sind einzelne Vorkommnisse und die Verantwortlichen waren vorher
demokratisch gewählt worden.

In der parlamentarischen Demokratie der BRD lagen also Anspruch und
Wirklichkeit eng beeinander. -- Wenn man von wenigen Ereignissen
absieht, ist das System als demokratisch zu bezeichnen.
