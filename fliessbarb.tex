\section{Fließbandarbeit}
\label{ssc:fliessbarb}
\index{Fließbandarbeit}

Die Arbeitswelt in der ersten Hälfte des 20. Jahrhunderts war geprägt
vom \beg{Taylorismus}, der in Kombination mit dem Fließbandbetrieb die
Massenproduktion ermöglichte.

Dieses neue Prinzip der Trennung von geistiger und körperlicher Arbeit
half bei der Lösung sozialer Probleme und versprach \jar{Wohlstand für
alle}. Der anfängliche Enthusiasmus wich aber bald der
Unzufriedenheit: Durch die Monotonie der Arbeit stellten sich
gesundheitliche Probleme ein. Außerdem konnten sich die Arbeiter nun
kaum noch mit ihrem Betrieb und den Erzeugnissen identifizieren, weil
sie das Endprodukt kaum noch zu Gesicht bekamen.

Die Qualität litt und die Motivation fehlte. Es kam zu Konflikten mit
den Arbeitgebern; wo es möglich war wanderten die Arbeitnehmer in den
Dienstleistungssektor ab.

\endinput
