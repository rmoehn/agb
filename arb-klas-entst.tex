\section{Die Entstehung der Arbeiterschaft als Klasse}
\label{sec:arb-klas-entst}
\index{Arbeiterklasse}

\mar{irgendwie mehr Geographie -- Wo bleiben die Arbeiter?} Ausgehend
von der \dat{Bevölkerungsexplosion ab circa 1800}, deren Ursachen die
Aufhebung der ländlichen Eheverbote, medizinische und
landwirtschaftliche Fortschritte waren und die sich zu einem sich
selbst erhaltenden Prozeß entwickelte, führte die Industrielle
Revolution zu einem \emph{sozialen Wandel}. Dieser schlug sich in der
Herausbildung von Abwanderungs- (vor allem ländliche Regionen wie
Ostpreußen) und Zuwanderungsgebieten (Ballungszentren wie das
Ruhrgebiet) und in der Entstehung von Großstädten und Städten
allgemein nieder.

Dieser Prozeß der \emph{Urbanisierung} ging einher mit der
Herausbildung der Arbeiterklasse, die den Hauptanteil an der
Binnenmigration hatte.

\endinput
