\section[Gründung der BRD]{Die Gründung der BRD\footnote{Es
handelt sich hier nur um einen groben Überblick. \label{ftn:grob-ueb}}
\mycite[333, 334]{WeltgeschNeuz}}
\label{sec:gruend-brd}
\index{BRD!Gründung}

\begin{chronik}
\item[Juli 1948]
Überreichung der \ges{Frankfurter Dokumente} an die
Ministerpräsidenten der Länder: Angebot zur Errichtung eines
westdeutschen Bundesstaates, Grundsätze für dessen Verfassung

\item[August 1948]
Erarbeitung eines Verfassungsentwurfes durch eine
Sachverständigenkommission als Beratungsgrundlage für den
\textsc{\Ins{Parlamentarischer Rat}{Parlamentarischen Rat}}

\item[8.\,5.\,1949]
Verabschiedung des Grundgesetzes durch den parlamentarischen Rat

\item[12.\,--\,23.\,5.\,1949]
Zustimmung der Westalliierten und der Bundesländer\footnote{Bayern
ratifizierte das Grundgesetz nicht, erkannte aber dessen
Rechtsverbindlichkeit an.} zum Grundgesetz 

\item[23.\,5.\,1949]
Inkrafttreten des \Ges{Grundgesetz für die Bundesrepublik
Deutschland}{Grundgesetzes für die Bundesrepublik Deutschland}

\item[14.\,8.\,1949]
freie, geheime, gleiche, direkte und allgemeine Wahl des ersten
\Ins{Deutscher Bundestag}{Deutschen Bundestages} durch die
westdeutsche Bevölkerung
\end{chronik}
