\section{Außenpolitik}

Liest man \Nam{Hitler, Adolf}{Hitler}s Regierungserklärung vor dem
Reichstag vom 17. Mai 1933\mycite{HitlRegErklFrie} und seine Rede vor
der Reichswehrführung in Berlin vom 3. Februar des selben
Jahres\mycite{HitlRedRwfueh}, wird der zweischneidige Charakter seiner
Außenpolitik deutlich: Während er vor dem Reichstag von einer Heilung
der Wunden des Versailler Vertrages innerhalb der Grenzen der Verträge
und durch friedliche Auseinandersetzung mit \enquote{den Nationen}
spricht, fordert er vor der Reichswehr einen \enquote{Kampf gegen
Versailles} und \enquote{Sorge für die Bundesgenossen} unter
Verwendung kriegerischer Mittel.

Man erkennt also eine Verschleierungstaktik, die die eigentliche
Kriegsvorbereitung durch den scheinbaren Weg über Verträge verdecken
sollte. Zu den konkreten Maßnahmen siehe Tabelle
\ref{tab:massn-hitl-ausspol}. Für Karikaturen zu dem Thema siehe den
Hefter.

\begin{table}
\caption{Außenpolitische Maßnahmen Hitlers}
\label{tab:massn-hitl-ausspol}

\begin{tabularx}{\textwidth}{cXX}
\toprule
Jahr    & verschleiernd     & kriegsvorbereitend \\ 
\midrule
1933 & Konkordat mit dem Papst\index{Reichskonkordat} & \\
     & \multicolumn{2}{c}{Austritt aus dem
Völkerbund}\index{Völkerbund!Austritt des Deutschen Reiches}   \\

1934 & \ges{Deutsch-Polnischerr -Nichtangriffspakt} &  \\

1935 & Rückgliederung des Saarlands
(Volksabstimmung)\index{Saarland!Rückgliederung} &  \\
     & \ges{Flottenabkommen} mit Großbritannien &  \\

1936 & \Ins{Olympische Spiele!Deutschland}{Olympische Spiele} in
Deutschland & Rheinlandbesetzung \index{Rheinland!Besetzung} \\

1937 &  & Beitritt Italiens zum \ges{Antikominternpakt} -- \ins{Achse
Berlin\,--\,Rom\,--\,Tokio}\\

1938 &  & Anschluss Österreichs\index{Österreich!Anschluss an das
Deutsche Reich} \\

     &  & Sudetenkrise\index{Sudetenkrise} -- \ges{Münchener Abkommen} \\

1939 &  & \ges{Deutsch-Sowjetischer Nichtangriffspakt} mit geheimem
Zusatzprotokoll \index{Zusatzprotokoll} \\
\bottomrule
\end{tabularx}
\end{table}

\endinput

Hohohohohoho.
