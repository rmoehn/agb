\section[Bau der Berliner Mauer]{Bau der Berliner
Mauer\mycite{Dtl50erJ}}
\index{Mauerbau}
\index{Berliner Mauer}
\label{sec:mauerbau}

\subsection*{Ausgangssituation}

\dat{Seit 1945 strömten DDR-Bürger} in die BRD. Sie sahen darin einen
Ausweg aus den widrigen Verhältnissen des sozialistischen Staates:
Unzureichender Versorgung, fehlendem Fortschritt und
Zwangskollektivierung sowie totalitaristischem Anspruch der SED bei
äußerst geringem politischem Mitbestimmungsrecht in der DDR standen
wirtschaftlicher Aufschwung und ein hohes Maß an politischer Freiheit
in der BRD gegenüber.

So kam es, dass bis zum folgenden Mauerbau 2,7 Millionen Menschen,
darunter zahlreiche junge und hochqualifizierte Facharbeiter Richtung
Westen zogen. Diese fehlten der DDR beim Aufbau einer leistungsfähigen
Wirtschaft, weshalb \dat{schon 1946 die Grenzen} durch Zäune und
Alarmvorrichtungen \dat{gesichert} wurden. Die Grenze in Berlin war
jedoch weiterhin offen, sodass \dat{1959/60} der DDR der
wirtschaftliche Zusammbruch drohte.\mar{Chruschtschows
Berlin-Ultimatum auch?}

\subsection*{Verfaul}

Am \dat{12. August 1961} unterschreibt \Nam{Ulbricht, Walter}{Walter
Ulbricht} auf geheimen Beschluss und nach Erlaubnis der UdSSR den
Befehl, auch die berliner Grenze zu schließen. Daraufhin rückten in
der Nacht Abteilungen der \Ins{NVA, Nationale Volksarmee}{NVA}, der
\ins{Volkspolizei} und Kampfgruppen aus, riegelten die Sektorengrenzen
ab und sperrten die Verkehrswege. Innerhalb eines Jahres wurde dann
die eigentliche die Westsektoren umschließende Grenzbefestigungsanlage
bestehend aus Mauer, Zäunen, Wachtürmen, Hundelaufstreifen,
Selbstschussanlagen und anderen Sicherungseinrichtungen errichtet.

\subsection*{Folgen}

\begin{itemize}
\item ca. 899 Todesopfer bei Fluchtversuchen aus der DDR
\item Ausreise in den Westen nur noch bei Genehmigung
möglich\mar{Nicht vorher schon?}
\item Ende der Massenflucht
\item weitere Verschlechterung des Verhältnisses zwischen Bevölkerung
und Staat
\item Auseinanderreißung von Familien
\item Verlust von ca. 60\,000 Arbeitsplätzen (Pendler)
\item Verschlechterung der Infrastruktur durch zerschnittenes
Verkehrsnetz
\end{itemize}

\subsection*{Bewertung anhand des modernen Demokratiebegriffs}

In Bezug auf die Verletzung der Grund- und Menschenrechte (siehe
Grundgesetz) stehen sich hier verschiedene Meinungen gegenüber:
Einerseits bedeutete der Mauerbau weitreichende Einschränkung der
Freizügigkeit und persönlichen Freiheit. Andererseits sah
\Nam{Kennedy, John Fitzgerald}{Kennedy} darin die bessere Alternative
gegenüber Krieg\mycite{WikMauer} und der britische Premierminister
\Nam{Macmillan, Maurice Harold}{Harold Macmillan} meinte, dass der
Stopp des Auswanderungsstroms
\textquote[\mycite{MacmillanMauer}]{nothing illegal} sei.

Ungeachtet dessen umging die DDR-Regierung mit dem Beschluss zum
Mauerbau wieder einmal Gewaltenteilung und Volkssouveränität, sodass
jener im krassen Gegensatz zum modernen Demokratiebegriff steht.

\endinput
