\chapter{Vorwort}
\label{chp:Vorwort}

\section*{Typographische Konventionen}

\begin{perldesc}
\item[\emph{Kursivschrift}]
wird verwendet, wenn Begriffe neu eingeführt werden, für geographische
Objekte und zur Hervorhebung.
\item[\textsf{Serifenlose Schrift}]
wird für Begriffe verwendet, die im Glossar erklärt werden.
\item[\textbf{Fette Schrift}]
wird für Datumsangaben im Text verwendet und für Überschriften um
Absatzbeginne natürlich ebenso.
\item[\textsc{Kapitälchen}]
werden für Personennamen verwendet.
\item[\enquote*{Text in einfachen Anführungszeichen}]
hat zitatähnliche Bedeutung, ist jedoch nicht exakt zitiert
beziehungsweise bezeichnet einzelne entnommene Wörter.
\item[Fußnoten]
enthalten interessante Informationen, die für das Verständnis des
Fließtexts nicht unbedingt notwendig sind.
\item[\textbf{\Large{a}}
in der Marginalspalte] kennzeichnet Aufgaben, die im Unterricht oder
in Leistungskontrollen zu lösen waren.
\item[Andere Informationen in der Marginalspalte]
sind Markierungen für mich, beipielsweise um auf Unklarheiten
hinzuweisen.
\end{perldesc}

Da ich Abkürzungen nur sehr sparsam verwende, sind diese im Glossar
mit aufgeführt.
