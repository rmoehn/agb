\section{Charakteristika des Ersten Weltkriegs}
\index{Totaler Krieg}

Dieser Abschnitt stellt die Frage, ob der Erste Weltkrieg ein
\emph{totaler Krieg} war. Dieser Begriff wurde erstmals von
\Nam{Ludendorff, Erich}{Erich Ludendorff} geprägt und wird überall
unterschiedlich definiert. Die folgenden Unterabschnitte sind die zwei
Charakteristika, die hier zur Abgrenzung des Begriffs herangezogen
werden.

Aus dem Folgenden kann man schließen, dass der Erste Weltkrieg zwar
die Züge eines totale Krieges trägt, jedoch nicht in die Dimensionen
des Zweiten Weltkriegs vordringt.


\subsection*{Einsatz aller zur Verfügung stehenden Ressourcen}

\subsubsection{Planung als Krieg zur präventiven Selbstverteidigung
(strategische Ressource)}

\begin{itemize}
\item \ges{Schlieffenplan}
\item Blitzkriegstrategie -- Krieg zu einem Zeitpunkt, da die Rüstung
des Feindes noch nicht abgeschlossen ist.
\end{itemize}


\subsubsection{Finanzierung über Kriegsanleihen}

Neun Kriegsanleihen im Gesamtvolumen von 97 Milliarden Reichsmark. Da
der Staat den Schuldendienst (Zins und Tilgung) nicht erwirtschaften
konnte, war das gleichsam Diebstahl.


\subsubsection{Industrialisierung des Krieges (Ausrichtung der
Industrie auf den Krieg)}

\begin{itemize}
\item Transportwesen: Nachschub, Truppenbewegung
\item Einsatz von Giftgas aus der chemischen Industrie
\item Tanks -- Fertigung von Panzern 
\end{itemize}


\subsubsection{Kriegsökonomie}

\begin{itemize}
\item Umstellung auf Kriegswirtschaft
\item stetiger Ausbau der Rüstungproduktion
\item Einsatz von Frauen und Kindern in der Rüstungsproduktion
\item Hilfsdienstgesetz von 1916: Verpflichtung aller nicht an der
Waffe dienenden Männer zwischen 17 und 65 Jahren zur Arbeit in einem
der Versorgung oder dem Krieg dienenden Betrieb 
\end{itemize}


\subsubsection{Einsatz neuer Waffentechnik}

\begin{itemize}
\item Maschinengewehre
\item U-Boote
\item Flugzeuge
\item Tanks
\end{itemize}


\subsubsection{Unerhörte Ausmaße}

\begin{itemize}
\item Großschlachten, beispielswiese bei \ort{Verdun} mit circa 600\,000
Toten
\item Anzahl der Soldaten -- circa 74 Millionen Kriegsteilnehmer
\item Anzahl der Verletzten -- circa 20 Millionen
\item Anzahl der Gefallenen -- circa 10 Millionen
\end{itemize}

%%%%%%%%%%%%%%%%%%%%%%%%%%%%%%%%%%%%%%%%%%%%%%%%%%%%%%%%%%%%%%%%%%%%%%

\subsection*{Verwischung der Grenze zwischen Front und Heimatfront}

\subsubsection{Erfassung der gesamten Bevölkerung durch
Kriegspropaganda}

\begin{itemize}
\item Propagandarede \Nam{Wilhelm \Rm{2}}{Wilhelms \Rm{2}} \ges{An das
deutsche Volk} vom August 1914
\item Gedichte
\item Lieder
\item Film
\item Postkarten
\end{itemize}

$\Longrightarrow$ Krieg als gesamtgesellschaftliche Aufgabe, als
Aufgabe jedes Bürgers


\subsubsection{Einbeziehung der gesamten Bevölkerung}

Heimatfront -- Einbeziehung der Zivilbevölkerung in die
Kriegshandlungen, bespiesweise durch

\begin{itemize}
\item Arbeit in der Rüstungsindustrie
\item Rationierung von Lebensmitteln ab 1915
\item Bombardement von Großstädten wie London
\item Genozid an den Armeniern (christliche Minderheit im damaligen
Osmanischen Reich)
\item Feldpost zur moralischen Stärkung
\end{itemize}
