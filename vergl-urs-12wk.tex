\section{Vergleich der Ursachen des Ersten und Zweiten Weltkriegs}
\mar{wesentlicher Unterschied: unbedingter Kriegswille Hitlers}
Siehe dazu Tabelle \ref{tab:vergl-urs-12wk}.

\begin{table}
\caption{Vergleich der Ursachen des Ersten und Zweiten Weltkriegs}
\label{tab:vergl-urs-12wk}
\footnotesize

\settowidth{\frstcolwdth}{internationale Mäch-}

\begin{tabularx}{\textwidth}{p{\frstcolwdth}XX}
\toprule
Ursachenfeld & Erster Weltkrieg & Zweiter Weltkrieg \\
\midrule

internationale Mächtekonstellation &
\vspace{-0.74em}
\begin{tablist}
\item Imperialismus bzw. Kolonialismus
\item Bündnissysteme
\end{tablist}
&
\vspace{-0.74em}
\begin{tablist}
\item Achsenmächte
\item Scheinpakte
\item beinahe anarchische internationale Politik
\end{tablist}
\\

innenpolitische Lage Deutschlands &
\vspace{-0.74em}
\begin{tablist}
\item Monarchie
\item Nationalismus
\item Kriegsdrängen der Eliten, Kriegseuphorie der Bevölkerung 
\end{tablist}
&
\vspace{-0.74em}
\begin{tablist}
\item Diktatur
\item Täuschung der Bevölkerung durch Propaganda
\item Ausschaltung von Kritikern und Gegnern 
\end{tablist}
\\

außenpolitische Lage Deutschlands &
außenpolitische Isolation bis auf das geheime Bündnis mit Österreich &
Verbindungen mit Österreich, Italien, Japan und ferner der Sowjetunion
\\

wirtschaftliche Lage Deutschlands &
\vspace{-0.74em}
\begin{tablist}
\item Aufrüstungspolitik
\item Umstellung auf Kriegswirtschaft erst während des Krieges 
\end{tablist} 
&
\vspace{-0.74em}
\begin{tablist}
\item Schuldenpolitik -- dem wirtschaflichen Kollaps nahe
\item Umstellung auf Kriegswirtschaft bereits erfolgt 
\end{tablist}
\\

Rolle der Eliten &
\vspace{-0.74em}
\begin{tablist}
\item Militär und Kaiser, weniger Wirtschaft
\item größtenteils euphorisch -- Drängen zum Krieg
\item einige kritische Stimmen
\end{tablist}
&
\vspace{-0.74em}
\begin{tablist}
\item Unternehmer, Parteispitze, Führer
\item Krieg als Ziel 
\end{tablist}
\\

Ideologie &
\vspace{-0.73em}
\begin{tablist}
\item Imperialismus: Nationalismus, Weltmachtstreben, teilweise
Rassismus, Sendungsbewusstsein
\item Krieg als politisches Mittel
\end{tablist}
&
Sozialdarwinismus, Rassenideologie als Grundlage für rassenpolitischen
Krieg, Völkermord, Lebensraumerweiterung 
\\

Rolle der internationalen Diplomatie &
Bündnissysteme als Ursache für die gewaltige Ausmaße des Kriegs &
\vspace{-0.74em}
\begin{tablist}
\item um sich selbst besorgte Staaten -- Völkerbund handlungsunfähig 
\item Unmöglichkeit der Umstimmung des ideologisch bedingten
Kriegswillens durch wirtschaftliche Zugeständnisse
\end{tablist}
\\

\bottomrule
\end{tabularx} 
\end{table}
