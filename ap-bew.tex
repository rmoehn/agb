\section[Außerparlamentarische Bewegungen]{Außerparlamentarische
Bewegungen\mycite{WikStud}
\mycite[522\,f.]{gelbesGeschichts}
\mycite{WeltgeschNeuz}
\mycite{SpiegChronik}
\mycite{GeschPolGes}
\mycite[14\,--\,20]{IzpBZeitWand}
}
\label{sec:ap-bew}

\begin{aufgabe}
Analysieren Sie außerparlamentarische Bewegungen am Beispiel der 68er
in der BRD und bewerten Sie sie hinsichtlich des modernen
Demokratiebegriffes! 
\end{aufgabe}

\subsection*{Kritikpunkte}

\begin{itemize}
\item Vietnamkrieg
\item frühere NSDAP-Mitgliedschaft Bundeskanzler Kiesingers
\item fehlende Aufarbeitung der NS-Vergangenheit in der neuen Wohlstandsgesellschaft
\item Notstandsgesetzgebung (Bedingung für vollständige Souveränität der BRD)
\item mangelhafte Studienbedingungen
\item Schwäche der innerparlamentarischen Opposition
\item überkommene Autoritäten
\item Konsum- und Wohlstandsdenken
\end{itemize}

\subsection*{Entwicklung}

\begin{chronik}
\item[1.\,12.\,1966] Bildung der Großen Koalition (CDU/CSU, SPD),
Schwäche der FDP
$\Rightarrow$ Aufruf Rudi Dutschkes (Studentenführer) zur Bildung einer
Außerparlamentarischen Opposition
\item[2.\,6.\,1967] Benno Ohnesorg bei Studentendemonstration
erschossen
\item[Nov. 1967] Proteste gegen veraltete Hochschulstrukturen
\item[4.\,4.\,1968] Martin Luther King ermordet
\item[11.\,4.\,1968] Attentat auf Rudi Dutschke $\Rightarrow$
Osterunruhen
\item[Mai 1968] Proteste gegen Notstandsgesetzgebung
\item[März 1969] Bundesregierung erachtet den Verband Deutscher
Studentenschaften als \enquote{revolutionären Kampfverband},
Einstellung der Zuschüsse
\end{chronik}

Die Bewegung verlief sich Anfang der Siebziger allmählich im Sande, da
man das Ziel, die Notverordnungsgesetze zu verhindern, nicht erreicht
beziehungsweise verfehlt hatte und außerdem nach Gewaltfreiheit
strebte.

\subsection*{Folgen}

\begin{itemize}
\item keine direkte Änderung des pol. Systems
\item politische Bewusstseinsbildung und Emanzipation
\item Steigerung der Partizipationsmöglichkeiten für Bürger (\jar{in
den Köpfen})
\item Veränderung des Lebensgefühls (Antiautorität (Zurückdrängung
autoritärer Denkmuster), sexuelle
Liberalisierung (\ins{Antibabypille}, Anfänge der Frauenbewegung) etc.)
\item Fortsetzung in RAF-Terror
\item Zuwendung der Mehrheit zur SPD -- sozialliberale Koalition ab
\dat{1969}
\end{itemize}

\noindent $\Longrightarrow$ politische und gesellschaftliche
Modernisierung

\subsection*{Bewertung}

\begin{itemize}
\item politische Beteiligung Jüngerer
\item Demonstration als demokratisches Grundrecht
\item Kritik an mangelhafter Umsetzung der demokratischen Grundrechte
-- Erfolg: Änderung der Notstandsgesetzesentwürfe unter Einführung
eines Widerstandsrechts für die Bürger und Verbot der Anwendung beim
Arbeitskampf\mar{?}
\item Eskalation führt zu Menschenrechtsverletzungen, also
undemokratisch -- Beteiligung beider Seiten
\item Kritik an der Großen Koalition: Angst der Bevölkerung vor
diktatorischem Missbrauch
\end{itemize}

\noindent $\Longrightarrow$ Förderung beziehungsweise Stärkung der
Demokratie



\endinput
