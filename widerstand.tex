\section[Widerstand]{Widerstand\mycite[116-125, 230-245]{GeschDrReich}}
\label{sec:widerstand}
\index{Widerstand gegen den Nationalsozialismus}

\begin{aufgabe}
Begründen Sie, warum trotz der Radikalität und Unmenschlichkeit des
Regimes der innerdeutsche Widerstand realtiv begrenzt blieb! 
\end{aufgabe}

\subsection[Definition]{Definition\mycite{WidDefschwieBeg}}

Es gibt immer wieder Debatten, wie der Begriff des Widerstands im
Nationalsozialismus zu definieren sei.\footnote{Selbst das hier
verwendete \cite{WidDefschwieBeg} weist hier Fehler in der logischen
Durchführung auf.} So schlug \Nam{Broszat, Martin}{Martin Broszat} in
den frühen Achtzigerjahren vor: \enquote{Wirksame Abwehr, Begrenzung,
Eindämmung der NS-Herrschaft oder ihres Anspruchs, gleichgültig von
welchen Motiven, Gründen und Kräften her.}

Es gab jedoch zahlreiche Kritik, die entgegesetzte, dass jene
Definition den Begriff zu weit fasse und schon bei \emph{Haltungen}
beginne, während man erst bei \emph{Handlungen} einsetzen
dürfe.\footnote{Ich bin anderer Ansicht. \enquote{Wirksame Abwehr,
Begrenzung, Eindämmung} sind für mich keine bloßen Haltungen.}
Weiterhin greift man die fehlende Berücksichtigung der Motive an.

Die Tendenz geht als hin zum Widerstand als
\textquote[\mycite{WidDefschwieBeg}]{Handeln, das auf grundsätzlicher
Ablehnung des Nationalsozialismus beruhte, das aus ethischen,
politischen, religiösen, soialen oder individuellen Motiven darauf
abzielte, zum Ende des Regimes beizutragen}.

Tabelle \ref{tab:widerst-def} gibt einen Überblick über verschiedene
Möglichkeiten der Definition.

% verfluchter Blocksatz
\begin{table}
\caption{Definitionen für \emph{Widerstand}\mycite{WidDefschwieBeg}}
\label{tab:widerst-def}

% Berechnung der Spaltenbreiten
\newlength{\colwidth}
\setlength{\colwidth}{0.33\textwidth}
\addtolength{\colwidth}{-2\tabcolsep}

\renewcommand*{\arraystretch}{1.5}

\begin{tabular*}{\textwidth}{*{3}{p{\colwidth}}}
\toprule

\multirow{2}{\colwidth}{\centering Widerstand im weitesten Sinn} &
\multicolumn{2}{c}{Widerstand im engeren Sinn} \\

\cmidrule{2-3}

&
kritische bis abweisende Haltung der Verweigerung und Selbstbehauptung
&
bewusste Anstrengung zur Änderung der Verhältnisse \\

\midrule

zusammenfassender Oberbegriff für gegen den Nationalsozialismus als
Ideologie und praktizierte Herrschaft gerichtete verschiedenartige
Einstellungen, Haltungen und Handlungen &
\emph{Verweigerung} als persönliche Abwehr von Herrschaftsanspruch und
Selbstbehauptung von Gruppen &
bewusster persönlicher Einsatz, verbunden mit einhergehenden
Gefährdungen \\

Umfasst also auch beispielsweise ins Exil geflohene, die kaum
Möglichkeit zur Einwirkung auf das nationalsozialistische Regime
hatten und die die sich weder durch Lockungen noch durch Zwang vom
Nationalsozialismus vereinnahmen ließen. &
\emph{Opposition} als Haltung grundsätzlicher Gegnerschaft gegen das
Unrechtsregime -- passiver Widerstand &
beispielsweise Planung des Sturzes der Diktatur (Attentate)
einhergehend mit der Errichtung einer neuen Gesellschaftsordnung --
aktiver Widerstand \\

\bottomrule 
\end{tabular*}
\end{table}

%%%%%%%%%%%%%%%%%%%%%%%%%%%%%%%%%%%%%%%%%%%%%%%%%%%%%%%%%%%%%%%%%%%%%%

\subsection{Gruppen}

Die deutschen Widerstandsgruppen kamen aus verschiedenen
gesellschaftlichen Schichten. Dabei ist der Mittelstand kaum
ernennenswert; dessen größte Teile waren dem \index{Hitler-Mythos}
Hitler-Mythos erlegen.

Etwas mehr, aber dennoch nur wenig, tat sich bei den Vertretern der
Industrie: Hier waren es hauptsächlich \Nam{Bosch, Robert}{Bosch} und
\Nam{Krupp von Bohlen und Halbach, Gustav}{Krupp}, die \Nam{Goerdeler,
Carl Friedrich}{Goerdeler} unterstützten und andere, die Kontakte zum
amerikanischen Geheimdienst unterhielten. Inwiefern man dabei jeweils
von \emph{Widerstand} sprechen kann, ist natürlich fragliche.

Schließlich formierten sich im politischen Lager zahlreiche locker
zusammengeschlossene Gruppen, die aber meist nur das gemeinsame Ziel
der Beseitigung des Regimes verband.

Alle waren allerdings der Meinung, dass die Befreiung vom
Nationalsozialismus von Deutschland selbst aus erfolgen müsse, da nur
so eine \emph{bedingungslose Kapitulation}, wie sie die Alliierten
forderten abgewendet und Souveränität und Verhandlungsfähigkeit
Deutschlands erhalten werden könnten.\mycite[191-193]{DeuzwDemuDikt}

Dabei waren die Widerstandsbewegungen im Ausland, beispielsweise die
\ins{R\'e{}sistance} in Frankreich und \index{Partisanen}
\ins{Partisanengruppen} in Polen, der Sowjetunion, Jugoslawien,
Griechenland und Italien, zum Teil viel reger und erfolgreicher als im
Reich selbst. Das lag daran, dass sich die politischen und geistigen
Strömung unter dem Eindruck eines gemeinsamen Besatzers zu einem
nationalen Befreiungskampf einten, während die Widerstandsgruppen in
Deutschland nur wenig zusammenarbeiteten, während die
Widerstandsgruppen in Deutschland nur wenig zusammenarbeiteten.

Anfangserfolge im Krieg hatten den \index{Hitler-Mythos} Hitler-Mythos
befeuert, Beamte und andere hatten einen Eid auf \Nam{Hitler,
Adolf}{Hitler} geschworen. Der Terror durch die \ins{Gestapo} und
scharfe Gesetze, die Kritik am Regime und Kontakte zum Ausland und zu
Ausländern im Inland verboten, taten ihr Übriges, um eine
Ablehnungshaltung der Bevölkerung gegenüber Widerständlern und
gewaltsamer Beseitigung der Obrigkeit zu schaffen.

Allerdings kehrte sich diese Haltung mit fortschreitendem
Kriegsverlauf immer weiter ins Gegenteil. Die Verschlechterung des
Zustands, insbesondere der Versorgungslage, ließ den Wunsch nach
baldigem Kriegsende immer größer werden; die unerwartete Kriegslage
verunsicherte das Regime, die Herrschaftsstruktur wurde löchrige,
sodass die Bevölkerung allmählich das Vertrauen verlor.

Im Folgenden sollen beispielhaft einige Gruppen mit ihren Motiven,
Zielen und Handlungen aufgeführt werden, die sich auch schon vor den
meisten Anderen gegen die nationalsozialistische Herrschaft gerichtet
hatte. Dabei werden sämtliche Grade von Kritik bis zum aktiven
Widerstand berücksichtigt.


\subsubsection[Kommunisten und Sozialdemokraten]{Kommunisten und
Sozialdemokraten\footnote{Die beiden Gruppen wirkten keineswegs von
Anfang an gemeinsam im Widerstand und auch später gab es teilweise
große Differenzen. So kämpften die Kommunisten zunächst ebenso gegen
die Sozialdemokraten und auch die Wahl der Mittel fiel unterschiedlich
aus: Während die Ersteren einigermaßen radikal und ohne Rücksicht auf
Verluste vorgingen, während die Anderen mehr Vorsicht walten ließen.}}
\index{Kommunismus}
\index{Sozialdemokratie}

Für deren Aktivitäten lagen natürlich ideologische Gründe vor.

\noindent Mittel:

\begin{itemize}
\item Flugblätter und Broschüren in großer Anzahl
\item Wandparolen
\item Demonstrationen
\item Fahnenhissen
\item Sprechchöre
\item Überzeugungsarbeit \enquote{von Mann zu Mann}
\item stille Verweigerung und
\textquote[{\mycite[120]{GeschDrReich}}]{öffentliche[s] Beharren auf
demokratischen und rechtsstaatlichen Idealen}
\item Milieubildung mit Nachbarschaftshilfe, Austausch von Meinungen,
Nachrichten und Spende von Trost
\item Fluchthilfe
\item Abhören ausländischer Sender
\item Zeitschriften und Publikationen, auch für das Ausland
\end{itemize}


\subsubsection{Kirchen}
\index{Widerstand!Kirchen}

Auch die Kirchen waren untereinander in ihrer Haltung gegenüber dem
Nationalsozialismus gespalten. Während die Katholiken sich zunächst
von \Nam{Hitler, Adolf}{Hitler} betören und täuschen ließen und später
eher seichte Mahnungen an das Regime richteten, wurde das anfängliche
Verlangen der Protestanten nach einem \nat{starken Mann} an der Spitze
des Staates befriedigt.

Allerdings vollzog sich mit der Zeit eine zunehmende Spaltung der
evangelischen Kirche in \Ins{Deutsche Christen}{Deutschen Christen}
als Anhänger des Nationalsozialismus und die \ins{Bekennende Kirche},
die ein entschlosseneres Unternehmen als die Katholiken an den Tag
legte.

Trotzdem muss man erkennen, dass weniger die Kirchen als Institutionen
selbst Widerstand leisteten, sondern eher christliche Einzelpersonen
-- besonders prominent sicherlich \Nam{Niemöller, Martin}{Martin
Niemöller} und \Nam{Bonhoeffer, Dietrich}{Dietrich Bonhoeffer} --
offen gegen den Nationalsozialismus vorgingen.\\

\noindent Kritikpunkte:

\begin{sloppypar}
\begin{itemize}
\item Kampf gegen Ordensgemeinschaften
(\nat{Klostersturm}\index{Klostersturm})
\item \textquote[{\mycite[121]{GeschDrReich}}]{\enquote{Pfaffenprozesse}
\index{Pfaffenprozesse} gegen Ordensgeistliche wegen angeblicher
Devisenschiebereien und Sittlichkeitsvergehen}
\item Eingriff des Staats in Kirchenangelegenheiten
\item Missachtung des \Ges{Reichskonkordat}{Reichskonkordats}
\item Rassenpolitik, Antisemitismus
\item Euthanasie
\item \nat{rassisch-völkische Weltanschauung}
\item Konzentrationslager
\item Willkür der \ins{Gestapo}
\item Missachtung christlicher Grundsätze
\end{itemize}
\end{sloppypar}

\noindent Mittel:

\begin{itemize}
\item Enzyklika \index{Enzyklika} \ges{Mit brennender Sorge}, geheim
vervielfältigt und verteilt
\item Eingaben\footnote{Diese waren eher zurückhaltender Art und
natürlich auch kaum erfolgreich.}
\item Reden von der Kanzel aus -- Predigten
\item \ins{Pfarrernotbund}
\item Untergrundarbeit
\end{itemize}


\subsubsection{Kreis um \Nam{Goerdeler, Carl Friedrich}{Carl Goerdeler}}

Hierzu gehörten hochrangige Militärs (beispielsweise \Nam{Beck,
Ludwig}{Generaloberst Ludwig Beck}), Wirtschaftsverantwortliche,
Industrielle, Gewerkschafter und andere vornehmlich konservative,
christliche und nationalliberale Bürger und Politiker. \\

\noindent Kritikpunkte:

\begin{itemize}
\item waghalsige Kredit-, Finanz- und Wirtschaftspolitik
\item Antisemitismus (negative Wirkung auf das Ansehen Deutschlands im
Ausland)
\item Entfernung des Denkmals für \Nam{Mendelssohn-Bartholdy,
Felix}{Felix Mendelssohn-Bartholdy} in Leipzig
\item Unterschätzung des Auslands
\item Kriegsbestrebungen
\end{itemize}

\noindent Ziele:

\begin{itemize}
\item Staatsstreich
\item Staats- und Gesellschaftsordnung auf Grundlage von
Rechtsstaatlichkeit, Moral, bürgerlichem Anstand und christlicher
Weltanschauung -- stark restaurativer Charakter
\item Kriegsende
\end{itemize}

\noindent Mittel:

\begin{itemize}
\item Kritik in Wirtschaftsgutachten für die Regierung
\item Werbung für Opposition gegen die Nationalsozialisten im Ausland
\item Knüpfung von Beziehungen zu Personen in verschiedensten
gesellschaftlichen Positionen
\item Einsatz des Militärs -- Verhandlungen mit hochrangigen
Offizieren
\end{itemize}


\subsubsection{Kreisauer Kreis\index{Kreisauer Kreis}}

Hier handelte es ich um eine Gruppe von Regimegegner
unterschiedlichster Profession, die sich auf dem Gut des \Nam{Moltke,
Helmuth James Graf von}{Helmuth James Graf von Moltke} in
\ort{Kreisau}, \ort{Niederschlesien}, zu Gesprächen traf. Führender
Kopf war neben dem Gutsbesitzer \Nam{Wartenburg, Peter Graf Yorck
von}{Peter Graf Yorck von Wartenburg}.

Bei dieser Gruppe muss man außerdem hervorheben, dass sie
hauptsächlich von den Verbrechen der Nationalsozialisten gegen Juden,
Kriegsgefangene und Bewohner der besetzten Gebiete zum Widerstand
angeregt wurden. Das war bei den bisher genannten Personen und
Körperschaften nicht primär der Fall.\\

\noindent Ziele\footnote{In diesem Abschnitt werden zahlreiche
Wortkombinationen verwendet, wie sie genau so in
\cite[236-238]{GeschDrReich} zu finden sind. Ich verzichte auf
ständige Anführungszeichen und Partikulärnachweise.}:

\begin{itemize}
\item Überwindung des Nationalsozialismus, des Machtstaats und des
Rassendenkens

\item Neuordnung und Neuorientierung von Staat und Gesellschaft
\begin{itemize}
\item Wiederherstellung eines humanen Rechtsstaats
\item Garantie von Glaubens- und Gewissensfreiheit
\item Recht auf Arbeit und Eigentum
\item Selbstbestimmung und Verantwortlichkeit statt Befehl und
Gehorsam
\item politische Verantwortung und Mitwirkung jedes Einzelnen statt
Diktatur und Unterwerfung
\item Gründung einer Völkergemeinschaft im Geiste internationaler
Toleranz
\end{itemize}

\item Bestrafung der nationalsozialistischen Verbrecher
\end{itemize}

\noindent Prinzipien:

\begin{itemize}
\item Humanität
\item christliche Ethik
\item Gerechtigkeit
\item Ablehnung von Gewalt\footnote{Hiermit ist hauptsächlich die
Ablehung eines gewaltsamen Sturzes des Regimes und einer Ermordung
\Nam{Hitler, Adolf}{Hitlers} gemeint. \Nam{Moltke, Helmuth James Graf
von}{Moltke} befürchtete nämlich, dass ein solches Vorgehen nach dem
Umsturz ähnlich der \emph{Dolchstoßlegende}\index{Dolchstoßlegende}
propagandistisch ausgeschlachtet würde.}
\end{itemize}


\subsubsection{Weitere}

Mit den genannten Gruppen ist ein großer Teil des Spektrums der Motive
und Ziele von Widerstandsgruppen abgedeckt. Für weitere Informationen,
insbesondere über die Attentate auf \Nam{Hitler, Adolf}{Hitler} und
über studentischen Widerstand wie den der \Ins{Weiße Rose}{Weißen
Rose} siehe auch \cite[239-245]{GeschDrReich}.

\endinput
