\documentclass[%
    10pt,
    headinclude=false,
    footinclude=false,
    draft,
]{scrartcl}

\usepackage[%
    paperwidth=8cm,
    paperheight=115mm,
    margin=3mm,
]{geometry}


\usepackage{pdfopts}

\pagestyle{empty}

\begin{document}
\tableofcontents

\section{Freude}

\Nam{}{Montesquieu} geht in seinem Werk \emph{Vom Geist der
Gesetze}\mycite{VomGeistderGes} unter anderem auf die Frage ein, wie
ein Staat gestaltet sein muss, damit die politische Freiheit der
Bürger gewährleistet ist. Dabei definiert er \jar{politische Freiheit}
als \textquote{jene geistige Beruhigung, die aus Überzeugung
hervorgeht, die jedermann von seiner Sicherheit hat}.

Er stellt dazu fest, dass es in jedem Staat drei Gewalten gibt, die
wir heute \emph{Legislative} (Erlässen von Gesetzen), \emph{Exekutive}
(Umsetzung öffentlicher Beschlüsse) und \emph{Judikative}
(Verhandlung von Streitfällen und Verbrechen). Um also die politische
Freiheit des Bürgers zu sichern, ist es essentiell, dass jene Gewalten
personell strikt voneinander getrennt werden:

\subsection{schöner}

\SetBlockThreshold{1}
\blockquote[{\mycite[216\,f.]{VomGeistderGes}\,\footnote{Der
Quellenverweis gilt auch für den obigen Abschnitt.} }]{Alles wäre
verloren, wenn ein und derselbe Mann beziehungsweise die gleiche
Körperschaft entweder der Mächtigsten oder der Adligen oder des Volkes
folgende drei Machtvollkommenheiten ausübte: Gesetze erlassen,
öffentliche Beschlüsse in die Tat umsetzen, Verbrechen und private
Streitfälle aburteilen.}

\section{Götterfunken}
\end{document}
