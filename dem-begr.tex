\section{Der Demokratiebegriff}
\label{sec:dem-begr}
\index{Demokratie}

\begin{aufgabe}
Untersuchen Sie, inwieweit die Verfassungen der neugegründeten Staaten
dem modernen Demokratiebegriff genügen!
\end{aufgabe}

\subsection*{Einteilung}

\begin{multicols}{2}
\noindent \emph{direkte Demokratie}:
direkte Entscheidungsfindung durch das Volk, beispielsweise in
Volksabstimmungen

\columnbreak

\noindent \emph{repräsentative (indirekte) Demokratie}:
Bürger wählen Repräsentanten, Sachentscheidungen durch Volksvertreter
\end{multicols}

\subsection*{Merkmale}

\begin{itemize}
\item Volkssouveränität (direkt, indirekt oder Mischform),
beispielsweise durch Wahlen oder Bürgerentscheide
\item Gewaltenteilung
\item Kontrolle der Exekutive und Legislative durch neutrale und
politisch unabhängige Judikative
\item Garantie der Menschen- und Bürgerrechte sowie der individuellen
Grundrechte (Schutz von Leben, Freiheit von Eigentum, Recht auf freie
Meinungsäußerung, Versammlungsfreiheit, Pressefreiheit)
\item Meinungspluralismus -- Parteienvielfalt
\item Verfassung
\end{itemize}

\endinput
