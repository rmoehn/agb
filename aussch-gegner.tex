\section{Ausschaltung realer und vermeintlicher Gegner des NS-Regimes}
\label{sec:aussch-gegner}

\begin{aufgabe}
Erläutern Sie ideologische Grundlagen der Ausschaltung vermeintlicher
Gegner!

Beschreiben Sie die Umsetzung der Ideologie an den beiden
vermeintlichen Gegnergruppen! 

Bewerten sie die Maßnahmen auf der Basis demokratischer Politik- und
Moralvorstellungen!
\end{aufgabe}

\begin{aufgabe}
Arbeiten Sie aus dem Text die elemente des historischen Gesamtrahmen,
in dem sicher nationalsozialistische Massenmord entwickelten, heraus!

Der Autor behauptet, dass die Rolle Hitlers entscheindend für die
Judenverfolgung/-vernichtung war. Erarbeiten Sie die dazu gehörende
Argumentation anhand des Textes und die zu Grunde liegenden
ideologischen Ansätze!

\Nam{Friedländer, Saul}{Friedländer} nennt die ideologische
Radikalisierung des ausgehenden 19. Jahrhunderts als einen Faktor, der
in Kombination mit weiteren den Holocaust \enquote{vorbereitete}.
Überprüfen Sie diese Theorie der Radikalisierung innerhalb der
Maßnahmen des Regimes von 1933 bis 1945!
\end{aufgabe}


\begin{description}
\item[Grundlage:] Rassenideologie und Sozialdarwinismus
\index{Rassenideologie} \index{Sozialdarwinismus}
\item[Ziel:] Verhindern der Beeinträchtigung des \nat{deutschen
Volkskörpers} und Schaffung der \nat{reinen Rasse}
\end{description}

Die Höherentwicklung des deutschen Volkes (\nat{Herrenmensch}) wurde
laut Ideologie bedroht durch\footnote{Beide Bereiche greifen
ineinander. Die Trennung dient nur der Veranschaulichung.}\mar{Das ist
ein Fall von Scheinheiligkeit.}:

\begin{multicols}{2}
Unerwünschtes (\nat{lebensunwertes}) Leben:
\nat{schlechte Erbanlagen} bedrohen die Höherentwicklung des Volkes
sowie der Menschheit.

$\rightarrow$ \nat{Rassenhygiene}\\

\emph{Euthanasie}

\columnbreak

Juden und \nat{Art-}/\nat{Rassenfremde}: \nat{rassisch bedingte 
Verderbtheit der Juden als minderwertige Rasse}

$\rightarrow$ Rassenantisemitismus\\

\emph{\beg{Holocaust}}
\end{multicols}

%%%%%%%%%%%%%%%%%%%%%%%%%%%%%%%%%%%%%%%%%%%%%%%%%%%%%%%%%%%%%%%%%%%%%%

\subsection{Euthanasie}
\label{ssc:Euthanasie}
\index{Euthanasie}

\subsubsection{Gesetzliche Grundlagen}

\begin{chronik}
\item[14.\,7.\,1933] \ges{Gesetz zur Verhütung erbkranken Nachwuchses}:
Zwangssterilisierung von vermeintlich Erbranken -- circa 6\,000 Tote,
am meisten Frauen

\item[26.\,6.\,1935]
Ergänzung: Legalisierung des Schwangerschaftsabbruchs bei
diagnostizierter Erbkrankheit

\item[15.\,9.\,1935]
\ges{Gesetz zum Schutz des deutschen Blutes und der deutschen Ehre}:
siehe \ref{sss:nürnberger-ges}

\item[18.\,10.\,1935]
\ges{Ehegesundheitsgesetz}: Verbot von Eheschließungen mit geistig
Behinderten und Erbkranken

\item[1.\,9.\,1939]
\emph{Geheimer Führererlaß}: Ermächtigung zur Durchführung der
Euthanasie
\end{chronik}

\subsubsection{Aktion T4}
\index{Aktion T4}

\setlength{\parskip}{0mm}

Diese nach dem Krieg nach der Adresse -- Tiergartenstraße 4 -- der
Euthanasiezentrale in Berlin benannte Aktion beinhaltete die Umsetzung
des Euthanasiebefehls. Der daran beteiligte Personenkreis teilte
sich in vier Aufgabenbereiche:

Die \ins{Reichsarbeitsgemeinschaft Heil- und Pflegeanstalten} (Leiter:
\Nam{Heyde, Werner}{Werner Heyde} (medizinisch), \Nam{Bohne,
Gerhard}{Gerhard Bohne} (Verwaltung)) war für die Erfassung der Opfer
zuständig und verschickte zu diesem Zweck ab Ende 1939 Meldebögen, die
Auskunft über Art und Dauer der Krankheit und die Arbeitsfähigkeit
verlangten.

Die \ins{Gemeinnützige Stiftung für Anstaltspflege} (Leiter:
\Nam{Schneider, Willy}{Willy Schneider}) diente als offizieller
Arbeitgeber der etwa 400 T4-Mitarbeiter.

Die \ins{Zentralverrechnungsstelle Heil- und Pflegeanstalten} (Leiter:
\Nam{Kaufmann, Gustav a.}{Gustav A. Kaufmann}) wickelte die Kosten ab.

Die \ins{Gemeinnützige Krankentransport GmbH} (Leiter: \Nam{Vorberg,
Reinhold}{Reinhold Vorberg}) fuhr mit meist grauen Bussen die Opfer in
die Zwischen- beziehungsweise Tötungsanstalten.\\

\noindent Die Aktion T4 begann 1939 mit der Kindereuthanasie, dem
an mindestens 5\,000 erbkranken oder behinderte Kindern und
Säuglingen.  Wenig später wurden die Tötungen auf Erwachsene
ausgeweitet. Etwa 70\,000 geistig Behinderte, psychisch Kranke, mehr
als fünf Monate stationär Behandelte, Straftäter und
\nat{Fremdrassige} wurden dabei vergast.

Aufgrund von Protesten der Kirche, der Angehörigen und in der
Bevölkerung stellte man die Vergasungsaktionen 1941 offiziell ein. Da
man aber wegen des Luftkriegs gegen Deutschland immer mehr
Krankenhausplätze benötigte, begann nun die sogenannte \emph{Wilde
Euthanasie}\index{Wilde Euthanasie}: Etwa 30\,000 Menschen wurden
heimlich vergiftet oder zu Tode gehungert.

%%%%%%%%%%%%%%%%%%%%%%%%%%%%%%%%%%%%%%%%%%%%%%%%%%%%%%%%%%%%%%%%%%%%%%

\subsection{Die Verfolgung der Juden}
\label{ssc:juden-verf}
\index{Holocaust}
\index{Judenverfolgung}

% Breite der ersten Spalte
\newlength{\frstcoljud}
\settowidth{\frstcoljud}{Phase}
% Breite der übrigen Spalten
\newlength{\othcol}
\newlength{\othcolsum} % der ganze Rest
\setlength{\othcolsum}{\textwidth}
\addtolength{\othcolsum}{-\frstcoljud}
\setlength{\othcol}{0.25\othcolsum}
\addtolength{\othcol}{-2\tabcolsep}

\begin{tabular*}{\textwidth}{p{\frstcoljud}*{4}{p{\othcol}}}
Phase & \Rm{1} & \Rm{2} & \Rm{3} & \Rm{4} \\
      & 1933-35 & 1935-38 & 1938-41 & 1941-45 \\
      & Boykottaktio"-nen & Ausgrenzung durch Gesetz und Terror
      & Deportation & Endlösung
\end{tabular*}

\subsubsection{Boykottaktionen\mycite[135, 136]{GeschDrReich}}
\label{sss:boykott}
\index{Judenboykott}

\begin{chronik}
\item[1.\,--\,3.\,4.\,1933]
befristeter\footnote{Laut Frau Seipold schlug die Aktion fehl, da die
Bevölkerung aus Gewohnheit weiter bei Juden einkaufte.
\cite[135]{GeschDrReich} stellt das etwas anders beziehungsweise
differenzierter dar.} Boykott jüdischer Geschäfte\footnote{Diese
Aktion bildete das Ende der spontanen Gewalt gegen die jüdische
Bevölkerung. Was folgte, waren organisierte Verbrechen.}

\item[7.\,4.\,1933]
\ges{Gesetz zur Wiederherstellung des Berufsbeamtentums} (ein Art
staatlicher \emph{Arierparagraph}): Entlassung von
\nat{\beg{Nicht-Ariern}} aus dem öffentlichen Dienst

\item[7.\,4.\,1933]
Möglichkeit des Entzugs der Zulassung jüdischer Anwälte

\item[April 1933]
\ges{Gesetz gegen die Überfüllung der deutschen Schulen und
Hochschulen}: Verdrängung der Juden aus den
Bildungsanstalten\footnote{Der vollständige Ausschluß erfolgte 1938.}

\item[5.\,10.\,1933]
\ges{Schriftleitergesetz}: Verdrängung jüdischer Journalisten aus den
Redaktionen

\item[1934]
Pause von antijüdischen Gesetzen
\end{chronik}


\subsubsection[\dat{Nürnberger Gesetze 15. September 1935}]
{\dat{Nürnberger Gesetze 15. September 1935}
\mycite[417-419]{gelbesGeschichts}}
\label{sss:nürnberger-ges}
\index{Nürnberger Gesetze}

\paragraph{\ges{Reichsbürgergesetz}} 
Juden verloren alle politischen Bürgerrechte -- sie wurden statt
als \nat{Reichsbürger} nur noch als \nat{Staatsangehörige}
gesehen und somit aus der \emph{Volksgemeinschaft} ausgeschlossen.

\paragraph{\ges{Gesetz zum Schutz des deutschen Blutes und der
deutschen Ehre}}
Diese Gesetz verbot Juden

\begin{itemize}
\item Ehen und außereheliche Beziehungen mit \nat{Ariern}
\item die Beschäftigung \nat{arischer} Dienstmädchen unter 45
Jahren
\item das Hissen der \nat{Reichs- und Nationalflagge} und das
Zeigen der \nat{Reichsfarben}
\end{itemize}

In den Folgejahren wurden die Rechte der Juden durch Sondergesetze und
Verordnungen immer weiter eingeschränkt,
beispielsweise\footnote{In \cite[135-137]{GeschDrReich} ist dies
differenziert dargestellt.}:

\begin{itemize}
\item Entfernung aus Beamtenpositionen und Wegfall der Pensionen
\item Boykottierung jüdischer Geschäftsleute und Industrieller
\item Entzug jeglichen Rechtsschutzes
\item Ungültigmachung mit Juden geschlossener Verträge
\item Zutrittsverbot zu Hotels, Pensionen, kulturellen Einrichtungen
und Parks
\end{itemize}

\noindent $\Longrightarrow$ völlige Entrechtung und gesellschaftliche
Isolierung der Juden \\

Im Hinblick auf die \dat{Olympischen Spiele} in Deutschland reduzierte
man die Repressalien \dat{1936} wieder, nur um sie im folgenden Jahr
erneut zu forcieren.


\subsubsection{\dat{Reichspogromnacht 9./10. November 1938}}
\label{sss:pogromnacht}
\index{Reichspogromnacht}
\index{Reichskristallnacht}

Am \dat{7. November 1938} verübte \Nam{Grynspan, Herschel}{Herschel
Grynszpan} in Paris ein Attentat auf den dortigen Botschaftssekretär
\Nam{Rath, Ernst vom}{Ernst vom Rath}.  Dieses nahmen die
Nationalsozialisten in Deutschland zum Anlaß und Vorwand für
gewaltsame Aussshreitungen gegen Deutsche jüdischen Glaubens zwei Tage
später.

Im Zuge derer kam es zu Morden und Plünderungen, Synagogen, jüdische
Gebetshäuser, Friedhöfe, Geschäfte und Wohnungen wurden in Brand
gesteckt beziehungsweise zerstört. Der Name \emph{Reichskristallnacht}
rührt von den zahlreichen Fensterscheiben her, die in jener Nacht
zerschlagen wurden.

Die Folgen der \emph{Novemberpogrome} waren mindestens 1\,300 Tote,
mehr als 1\,400 zerstörte Synagogen, Verhaftungen und Deportationen
und weitere Verschlechterung der Lebensbedingungen für Juden. Für die
entstandenen Schäden von ungefähr 1,12 Milliarden Reichsmark mußten
die Opfer selbst aufkommen.

Mit dem endgültigen Verlust finanzieller Mittel wurde den Juden eine
eventuelle Ausreise aus noch weiter erschwert.


\subsubsection[Radikalisierung der Judenpolitik \dat{September
1939\,--\,1941}]
{Radikalisierung der Judenpolitik \dat{September
1939\,--\,1941}\mycite[209, 216]{GeschDrReich}\mycite[433,
434]{gelbesGeschichts}}
\label{sss:rad-jud-pol}

\paragraph{Deportationsmaßnahmen/-pläne} \index{Deportation}
Durch ihre Kriegserfolge angespornt, stellten die Nationalsozialisten
verschiedene Pläne, die Juden aus Europa zu verbannen, auf. Diese
wurde jedoch aufgrund der schieren Anzahl zu Deportierender nicht
umgesetzt: 

\begin{chronik}
\item[1939\,--\,1941]
Zwangsumsiedlung nach Polen, Zusammenfassung und Isolierung in
\beg{Ghettos} \index{Ghetto}

\item[1940]
(Sieg über Frankreich): \beg{Madagaskarplan} \index{Madagaskarplan}

\item[1941]
(Überfall auf die UdSSR): Umsiedlung nach Sibirien
\end{chronik}


\paragraph{Vorgehen der SS-Einsatztrup"-pen beim Überfall auf Polen}
während des Überfalls auf Polen führten die SS-Einsatztrup"-pen im
Schatten der Wehrmacht \nat{politische Säuberungen} durch: Mit dem
iel der Ausrottung der jüdischen Bevölkerung kam es zu
assenerschießungen und Massakern. In den ersten Kriegswochen starben
abei circa 5\,000 Menschen.

ußerdem wurde die Kennzeichnungspflicht für Juden eingeführt und
man pferchte sie in Ghettos zusammen, wo sie Zwangsarbeit für die
Rüstungsindustrie zu leisten hatten.

\paragraph{Vertreibung der Juden nach dem Polenfeldzug}
\index{Deportation}
Mit der Begründung, Wohnraum werde \nat{aus kriegswirtschaftlichen
Gründen dringend benötigt}, deportierte man die Juden in Lager
in der Gegend von Lublin und im \ort{Generalgouvernement}. Das geschah
zunächst Bewohnern des \ort{Gaus Warteland}, ein halbes Jahr später
denen aus \ort{Pommern}. Im \dat{Februar 1940} vertrieb man die
Juden aus der Gegend von \ort{Stettin} und im \dat{März 1940} aus
\ort{Schneidemühl} in Preußen.

\paragraph{Vorgehen gegen die jüdische Bevölkerung während des
Krieges gegen die Sowjetunion}
Mit dem Krieg gegen die UdSSR wurden die Tötungsaktionen des Überfalls
auf Polen fortgesetzt. Neben der Wehrmacht, die dabei in die
Machenschaften der SS integriert wurde, rekrutierte man nun auch Teile
der Bevölkerung als \nat{Reserve-Polizeibataillone} sowie verbündete
Truppen, vor allem aus Weißrußland und Rumänien.

Von den 4,7 Millionen sowjetischen Juden kamen so bis Ende 1942 2,2
Millionen zu Tode.


\subsubsection{Konzentrations- und Vernichtungslager -- Die
\nat{Endlösung der Judenfrage}}
\label{sss:konz-vern-endl}
\index{Konzentrationslager}
\index{Vernichtungslager}
\index{Endlösung}

\begin{chronik}
\item[Juni 1941]
Vergasungsanlagen für \ort{Auschwitz}, Einsatz ab Herbst als
Vernichtungslager\footnote{Nach \ort{Chełmno}, wo noch die sogenannten
\ort{Gaswagen} verwendet wurden\cite{WikChelmno}, war es damit das
zweite Vernichtungslager und sollte das größte derer werden.}

\item[31.\,7.\,1941] Anweisung an \Nam{Heydrich, Erwin}{Reinhard
Heydrich}, eine \nat{Endlösung der Judenfrage} zu finden

\item[20.\,1.\,1942]
\emph{Wannseekonferenz}:
\begin{itemize}
\item Organisation und Koordination der Deportation der europäischen
Juden
\item Beschluß, Juden in ganz Europa als Arbeistkräfte auszubeuten und
zu ermorden\footnote{Laut \cite{WikWannsee} war dieser Beschluß
faktisch schon gefaßt. Ich füge ihn hier ein, da sich kein anderer
Platz ergeben hat.}
\end{itemize}

\item[1942]
Errichtung weiterer Vernichtungslager in \ort{Bełżec}, \ort{Sobibór},
\ort{Treblinka} und \ort{Majdanek}
\end{chronik}

\endinput
