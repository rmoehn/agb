\section[Die Diskussion über die Kriegsschuldfrage]{Die Diskussion
über die Kriegsschuldfrage\mycite{19Jh1}}
\index{Kriegsschuldfrage}

Seit im \ges{Vertrag von Versailles} festgelegt wurde, dass
Deutschland die alleinige Schuld am Ersten Weltkrieg trage, treibt die
Diskussion über die Kriegsschuldfrage die Mühlen der Propagandisten
und der Historiker an. Im Folgenden sollen bespielhaft die Positionen
der Geschichtsforscher \Nam{Fischer, Fritz}{Fritz
Fischer}\mycite{KriegsschFF} (Fischer-Kontroverse) und \Nam{Freund,
Michael}{Michael Freund}\mycite{KriegsschMF} nach dieser
Aufgabenstellung gegenübergestellt werden:\\
\mar{Wie sollen wir das richtig beurteilen?}

\begin{aufgabe}
Erörtern Sie die kontroversen Ansichten der beiden Historiker zur
Kriegsschuldfrage und formulieren Sie ein eigenes Urteil!\\
\end{aufgabe}

\noindent Fritz Fischer behauptet, dass \textquote{Deutschland den
österreichisch-serbischen Krieg gewollt, gewünscht und gedeckt} habe.
-- Dass es ihn \enquote{gedeckt} hat, ist klar und Folge des Drei-
beziehungsweise Zweibundes. Ob Deutschland den Krieg \enquote{gewollt}
und \enquote{gewünscht} hat, ist die Frage. Wenn es den
österreichisch-serbischen Krieg gewünscht hat, dann nur, um ihn dort
zu belassen. Eine Ausweitung war keineswegs geplant, zumindest, wenn
man \cite{19Jh1} Glauben schenkt. Ob Deutschland es \textquote{bewusst
auf einen Konflikt mit Russland und Frankreich ankommen
ließ}\footnote{Vermutlich ist schon der Text, der mir vorlag falsch
zitiert.}, ist fraglich. 

Michael Freund bezieht sich auf das Buch von Fritz Fischer. Er
urteilt, dass es \textquote{im Jahre 1919 stecken geblieben sei} und
es darin \textquote{Zu der bloßen Umstülpung dieser Unschuldslüge,
nämlich zu einer aufgewärmten Kriegsschuldlüge der Alliierten in ihrer
krassesten Form} komme. Das lässt sich am vorliegenden Textauszug
Fischers nicht festmachen, der nur schreibt, dass \textquote{die
deutsche Reichsführung einen erheblichen Teil der historischen
Verantwortung für den Ausbruch des allgemeinen Krieges} trage.
Insgesamt stellt Michael Freund Fischers Ansichten dar, als würden sie
genau der Behauptung der Alliierten entsprechen, wobei er vermutlich
das ganze Buch gelesen hat und ich nur den kleinen Ausschnitt.

Michael Freund bringt dabei seine Befürchtung zum Ausdruck, dass auch
die neue Bundesrepublik an der neuerlichen Ausschlachtung der
Kriegsschuldfrage zugrunde gehen könnte. Ich bin zwar der Meinung,
dass das etwas übertrieben ist, halte aber ebenso die Ansicht Fischers
für kurzsichtig: Sicher trug die \enquote{deutsche Reichsführung}
\enquote{erheblichen Teil der historischen Verantwortung für den
Ausbruch des allgemeinen Krieges}. Doch nicht nur die: Auch die
deutsche Wirtschaft, das deutsche Militär und die deutsche
Bevölkerung. Auch Österreich. Auch Frankreich, Russland und
Großbritannien. Um es Zusammenzufassen: Nationalismus, Militarismus
und Imperialismus hatten überall in Europa zum Schluss von
Kriegsbündnissen und zur Aufrüstung geführt.

Somit kann man die Kriegsschuld auf keinen Fall an einem Einzelnen
festmachen.
\mar{Letztendlich scheint Fischers Position größtenteils zu stimmen.}



\endinput
