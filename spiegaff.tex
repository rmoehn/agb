\section[\ins{Spiegel}-Affäre]{\ins{Spiegel}-Affäre\mycite[4\,f.]{IzpBZeitWand}}
\label{sec:spiegaff}

Am \dat{10. Oktober 1962} veröffentlichte die Wochenzeitschrift
\ins{DER SPIEGEL} den Artikel \ges{Bedingt
abwehrbereit}\mycite{BedEinsber}. Der Verfasser \Nam{Ahlers,
Conrad}{Conrad Ahlers} analysierte darin ein zuvor stattgefundenes
NATO-Manöver und \textquote{kam zu dem Schluß, daß die Verteidigung
der Bundesrepublik im Falle eines Angriffs des \ges{Warschauer
Pakt}{Warschauer Pakts} keinesweigs gesichert sie und daß das [von
Bundesverteidigungsminister \Nam{Strauß, Franz-Josef}{Franz-Josef
Strauß} verfolgte] Konzept des vorbeugenden Schlages den Frieden eher
gefährdete als sicherte}.

Auf der Grundlage eines Gutachtens des
Bundesverteidigungsministeriums, welches aussagte, dass mit jenem
Artikel \textquote{geheimzuhaltende Tatsachen} veröffentlicht worden
waren, ließ die Bundesanwaltschaft ab dem \dat{26. Oktober 1962} auf
Verdacht des Landesverrats die \ins{Spiegel}-Redaktion besetzen und
einige leitende Mitarbeiter verhaften. Bundesjusitzminister
\Nam{Stammberger, Wolfgang}{Wolfgang Stammberger} und Hamburger
Innensenator \Nam{Schmidt, Helmut}{Helmut Schmidt} wurden vorher nicht
informiert.  Außerdem ließ Strauß indem er das Auswärtige Amt umging
Ahlers im Urlaub in Spanien über den Militärattaché der deutschen
Botschaft verhaften.

\subsection*{Reaktionen}

\begin{itemize}
\item Unterstützung des \ins{Spiegel}s bei der Erstellung des
nächstens Hefts durch die Redaktionen anderer Zeitungen
\item Proteste von Intellektuellen und Gewerkschaften gegen den
angeblichen Angriff auf Presse- und Meinungsfreiheit
\item Regierungskrise: Rücktrittsforderungen der FDP an Strauß und
Abzug der FDP-Minister aus der Bundesregierung
\end{itemize}


\subsection*{Folgen}

\begin{itemize}
\item Ende der Besetzung der \ins{Spiegel}-Redaktion am \dat{26.
November}
\item Entlassung der \ins{Spiegel}-Mitarbeiter aus der
Untersuchungshaft
\item Amtsverzichts Strauß' am \dat{30. November}
\item Rücktrittsankündigung \Nam{Adenauer, Konrad}{Konrad Adenauer}s
für Herbst 1963
\end{itemize}


\subsection*{Bewertung anhand des modernen Demokratiebegriffs}

\begin{itemize}
\item Umgehung der Gewaltenteilung
\item Angriff auf Presse- und Meinungsfreiheit
\item Behauptung der Pressefreiheit
\item erstmalige öffentliche politische Stellungnahme der Bevölkerung
nach dem Krieg -- Sieg der Öffentlichkeit
\end{itemize}

$\Longrightarrow$ grenzwertig in der Auslösung, demokratisch in der
Lösung
