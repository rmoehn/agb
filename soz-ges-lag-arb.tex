\section{Die soziale und gesellschaftliche Lage der Arbeiterschaft
während der Industrialisierung}
\label{sec:soz-ges-lag-arb}
\index{Arbeiterklasse!soz. und ges. Lage}

\subsection{Wohn- und Lebensverhältnisse}

\begin{itemize}
\item Wohnungsknappheit durch rasantes Bevölkerungswachstum
$\Rightarrow$ Bau von \beg{Mietskasernen}:

\begin{itemize}
\item kleinste dunkle feuchte (meist Einraum-) Wohnungen $\Rightarrow$
keine Privatsphäre
\item miserable sanitäre Verhältnisse (eine Toilette für 50 Personen)
\item kaum Heizung
\item Schlafgängertum
\end{itemize}

\item mangelnde Hygiene, Krankheiten, Seuchen
\item Familienväter oft betrunken:

\begin{itemize}
\item Lösung familiärer Konflikte oft mit Gewalt
\item keine Erziehungsinstanz
\item keine Vorbilder für die Heranwachsenden
\end{itemize}

\item schlechte Erziehung und fehlende Schulbildung $\Rightarrow$ kaum
Chancen auf sozialen Aufstieg, Kriminalität
\end{itemize}

\subsection{Arbeitsverhältnisse}

\begin{itemize}
\item Hohe Spezialisierung der Arbeit führte zu Produktivitäts- und
Qualitätssteigerung, aber auch zu Sinnentleerung, Stupidität und
Monotonie.
\item härteste Arbeitsbedingungen
\item hohes Unfallrisiko
\item keine Versicherung
\item kein Kündigungsschutz
\item überlange Arbeitszeiten (bis zu 16 Stunden an sechs Tagen in der
Woche)
\item Hungerlöhne
\item Arbeiterheer führte zu Lohndumping. -- Selbst Kinder mußten
arbeiten.
\end{itemize}

Es entstand also eine neue Zweiklassengesellschaft mit der besitzenden
Klasse, deren Attribute Reichtum, politische und wirtschaftliche Macht
waren, auf der einen Seite und der arbeitenden Klasse, die durch
Massenverelendung und -armut gekennzeichnet war, auf der anderen.

Allerdings muss man dazusagen, dass es auch in der Arbeiterklasse
Standesunterschiede gab. So standen die Facharbeiter, deren äußeres
Kennzeichen der Hut war über den älteren (30\,--\,40 Jahre) und den
jüngeren einfachen Arbeitern. Diese hatten eine Mütze auf dem Kopf.
Auf der untersten Stufe standen die Tagelöhner, die keinerlei feste
Anstellung hatten. Die Entlohnung und die Qualität der
Lebensverhältnisse lief proportional zu dieser Rangordnung. Die Folge
waren starke Spannungen und Uneinigkeit innerhalb dieser
Klasse\footnote{Den Facharbeiter ging es beispielsweise recht gut. Da
sie außerdem Macht über die anderen Arbeiter besaßen, waren sie
verhaßt.} Das behinderte die Arbeiter bei ihrem späteren Kampf für
bessere Lebensverhältnisse und verzögerte die Lösung der sozialen
Frage.

\endinput
