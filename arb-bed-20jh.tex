\section{Arbeitsbedingungen in der ersten Hälfte des 20. Jahrhunderts}
\label{sec:arb-bed-20jh}

Die Arbeitswelt in der ersten Hälfte des 20. Jahrhunderts war auch
geprägt von miserablen Arbeitsbedingungen: \emph{Niedrige
Mindestanforderugen}, \emph{fehlende Aufstiegschancen}, 
\emph{hohe Fluktuation}, \emph{niedrige Löhne} und \emph{geringe
Arbeitsplatzsicherheit} machten den Arbeitern das Leben schwer.

Doch man tat auch einiges, um jene Bedingungen zu verbessern.  Gab es
um \dat{1900} wenigstens einige \dat{Herbergen für Wanderarbeiter},
brachte das \dat{\emph{\Nam{Stinnes, Hugo}{Stinnes}-\Nam{Legien,
Carl}{Legien}-Abkommen} vom 15.\,11.\,1918} große Fortschritte. Der
aus Furcht vor einer Vergesellschaftung der deutschen Industrie im Zuge
der Novemberrevolution zwischen Gewerkschafts- \Nam{}{Legien} und
Industrievertretern \Nam{}{Stinnes} geschlossene Vertrag legte die
Zusammenarbeit von Arbeitnehmern und -gebern fest. Die Arbeitgeber
verpflichteten sich so, die Rolle der Gewerkschaften als Vertreter der
Arbeiterinteressen anzuerkennen und sie als gleichberechtigte
Tarifpartner zu betrachten. Außerdem wurde der \emph{Achtstundentag}
bei vollem Lohnausgleich eingeführt.
\index{Arbeitnehmer}
\index{Arbeitgeber}
\index{Gewerkschaft}
\index{Tarifverhandlung}
\index{Achtstundentag}

Damit wurden ab 1918/1919 in allen Branchen \emph{Tarifverträge},
Regelung der Arbeitsbedingungen durch \emph{Kollektivvereinbarungen},
Anerkennung der \emph{Koalitionsfreiheit} durch die Arbeitgeber und
\emph{Arbeiterausschüsse} in Betrieben mit mindestens 50 Beschäftigten
möglich.
\index{Tarifvertrag}
\index{Kollektivvereinbarung}
\index{Koalitionsfreiheit}
\index{Arbeiterausschuss}

\endinput
