\section{Die Aufhebung der Rückständigkeit Deutschlands}
\label{sec:aufh-rueck-d}

\begin{figure}
\centering
\begin{sideways}
\input{vergl-d-e.eepic}
\end{sideways}
\caption{Die deutschen Verhältnisse}
\label{pic:dt-verh}
\index{Deutschland!Rückständigkeit}
\end{figure}

\paragraph{Wie Abbildung \ref{pic:dt-verh} zeigt,} waren um 1800
dringend Maßnahmen nötig, um ebenfalls in die industrielle Revolution
einsteigen zu können. Dieser Abschnitt soll zeigen, wie diese
beschaffen waren und was sie bewirkten.

\subsection{Die Preußischen Reformen}
\index{Preußische Reformen}

\begin{aufgabe}
Untersuchen Sie die Preußischen Reformen auf ihre Veränderungen
für das Wirtschaftsgefüge hin!

Fassen Sie sämtliche Informationen zusammen,
die zum wirtschaftlichen Aufschwung Deutschlands als Voraussetzung
anzusehen sind und stellen Sie deren Wirkung dar!
\end{aufgabe}

\subsubsection{Situation:}

\begin{itemize}
\item Einschränkung der Mobilität der Landbevölkerung durch auf
persönlicher Abhängigkeit beruhende Gutswirtschaft
\end{itemize}

%%%%% INT
%% fehlendes Material
%%%%% 20.12.2009


\subsection{\dat{Die Fr"uhindustrialisierung 1770-1850}}
\label{ssc:frueh-ind}
\index{Fr"uhindustrialisierung}

\begin{aufgabe}
Stellen Sie Ansätze wirtschaftlichen Aufschwunges in Deutschland
anhand der Frühindustrialisierung dar!

Bewerten Sie die Wirkungsweise dieser auf den
Industrialisierungsprozeß insgesamt!
\end{aufgabe}

\paragraph{Die Verarbeitung von Agrarprodukten} wie sie in
Zuckerfabriken, Branntweinbrennereien, Brauereien, Ölmühlen und
Tabakfabriken betrieben wurde, bildete die Wurzeln des
Unternehmertums. So wurde beispielsweise der Zuckerrübenbauer zum
Zuckerfabrikanten und der Textilverleger zum Textilfabrikanten.

\paragraph{Die Textilproduktion beruhte auf dem Verlagssystem}
(beispielsweise heimgewerbliche Leinenherstellung), das in dieser
Phase der Industrialisierung zur Blüte kam
\cite[207]{gelbesGeschichts}.  Mit der \dat{Einführung mechanischer
Webstühle 1830} wurden dann die Voraussetzungen für die
Textilindustrie geschaffen.

\paragraph{Ferner erschloss man neue Industriezweige,} wie den
Kohlebergbau und die Erzgewinnung.

Diese Ansätze wirtschaftlichen Aufschwungs schufen einen Bedarf nach
Maschinen. -- Kleine Reparaturwerkstätten entwickelten sich zu
Maschinenfabriken. -- Man benötigte Metall. -- Um die isolierten
Produktionsinseln zu verbinden, musste man die Infrastruktur aufbauen.
-- Man baute 1835 die erste Eisenbahnstrecke. -- Wieder brauchte man
Metall.

Hier zeigen sich die Grundlagen der \beg{Interdependenzen} -- starker
Rückkopplungseffekte, die die folgende rasante Entwicklung
Deutschlands vom Agrar- zum Industriestaat bedingten.

Die Wirtschaft selber entwickelte sich in der Zeit der
Frühindustrialisierung allerdings nur langsam.


\subsection{Der Zollverein}
\index{Zollverein}

\begin{aufgabe}
Stellen Sie den Entstehungsprozess des Zollvereins dar!

Bewerten Sie seine Bedeutung für den Wirtschaftsaufschwung/die
Industrialisierung in Deutschland!
\end{aufgabe}

\endinput
