\section{Inflation}
\label{sec:soz-veraend}
\index{Inflation!Weimarer Republik}
\index{Hyperinflation}

Während des Krieges hatte die Inflationsrate im Deutschen Reich zu
steigen begonnen. Diese Tendenz kuliminierte \dat{1923}, als die
schleichende Inflation zur galoppierenden und schließlich zur
\dat{Hyperinflation} wurde. Nun erst wurde das Problem mit der
\dat{Währungsreform vom 15. November 1923}, die die \ins{Rentenmark}
einführte, wobei eine Rentenmark den Wert von 1\,000\,000\,000\,000 (1
Bio.) Papiermark und 4,2 Dollar den Wert einer Rentenmark
hatte.\mycite[28]{IzpBWeimRep}

\subsection*{Folgen}

\subsubsection{Wirtschaft}

\begin{itemize}
\item materielle Verluste
\item Vermögensumschichtung
\item Löhne auf Vorkriegsnieveau
\item Vernichtung von Geldvermögen
\item Konzentrationsprozesse -- Konternbildung 
\end{itemize}


\subsubsection{Bevölkerung}

\begin{itemize}
\item Abhängigkeit von der Armenfürsorge
\item Zerfall des klassischen Bildungsbürgertums als Träger der
Gesellschaft durch finanzielle Entmachtung
\item Sachwertbesitzer werden zu Geldwertbesitzern
\item materielle und damit seelische Not
\item soziale Schichtung (Klassengesellschaft im Übergang)\mar{?}
\item Neuerungen im sozialen Bereich (bspw. Sozialverischerung) 
\item hohe Arbeitslosenrate
\end{itemize}


\subsubsection{Politik}

Die Inflation wirkte sich auf die Politik insofern aus, als die
Radikalisierung, das heißt die Entfernung von den Parteien der Mitte,
vorangetrieben wurde. Dies zeigte sich in den Wahlen von 1924.


\subsection*{Bilanz}

\subsubsection{Gewinner}

\begin{itemize}
\item Staat -- Lösung des Schuldenproblems
\item Sachwertbesitzer
\item Unternehmer -- Kredite aus getätigten Modernisierungs- oder
Erweiterungsinvestitionen konnten mühelos getilgt werden. 
\end{itemize}

\subsubsection{Verlierer}

\begin{itemize}
\item Staat -- Die Menschen schoben die Schuld am Elend den
staatstragenen Parteien zu und wendeten sich an die Radikalen.
\item Mittelschicht, Stadtbewohner -- Verlust von Erspartem
(Alterssicherung), großen Einkommensverluste vor der Reform
\item Ausland -- Gesamtverluste auf deutschen Bankguthaben und
Wertpapierbeständen von circa 15 Milliarden Goldmark
\end{itemize}

\endinput
