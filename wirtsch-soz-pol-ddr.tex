\section{Wirtschafts- und Sozialpolitik der DDR 1970\,--1980}

Am \dat{3. Mai 1971} wurde \Nam{Ulbricht, Walter}{Walter Ulbricht}
durch \Nam{Honecker, Erich}{Erich Honecker} als Erster Sekretär des
Zentralkommitees der SED abgelöst. Damit wurden jegliche Bestrebungen
nach einer Wiedervereinigung Deutschlands beendet, die verbliebenen
privaten Betriebe verstaatlicht und eine neue politische Zielsetzung
eingeführt: War Ulbrichts Devise noch \enquote{Wie wir heute arbeiten,
werden wir morgen leben.}, orientierte Honecker nun auf eine
\jar{Einheit von Wirtschafts- und Sozialpolitik}.

Das bedeutete Erhöhung des Lebensstandards der Bevölkerung und
Ausweitung von Sozialleistungen zur Bewältigung gesellschaftlicher
Probleme (Versorgungsengpässe, Geburtenrückgang) bei gleichzeitigem
hohem Wirtschaftswachstum. Die Ausgaben sollten durch Kredite
finanziert werden, die sich mit der Zeit zu einem immer größeren
Schuldenproblem entwickeln sollten.\footnote{Der Leiter der
\Ins{Staatliche Planungskommission}{Staatlichen Planungskommission}
\Nam{Schürer, Gerhard Paul}{Gerhard Schürer} sah diese Entwicklung
voraus und soll Honeckers Maßnahmen mit einer Umdrehung der Devise
Ulbrichts kommentiert haben: \enquote{Wie wir heute leben, werden wir
morgen arbeiten.}}

\subsection*{Maßnahmen}

\begin{itemize}
\item Wohnungsbauprogramm -- Neubaugebiete mit relativ komfortablen
Wohnungen in den Städten
\item Kauf westlicher Produktionsanlagen
\item umfangreiche Subventionen von Lebensmitteln, Mieten,
öffentlichen Verkehrsmitteln und anderem
\item Familienförderung durch Geburtenprämien, zinslose Kredite bei
Familiengründungen und anderes
\end{itemize}

\subsection*{Ergebnisse}

\begin{itemize}
\item Verbesserung des Lebensstandards -- höchster in den RGW-Staaten
\item Einkommensanstieg, \jar{zweite Lohntüte}, Verbesserung der
Versorgung mit Konsumgütern
\item Anstieg der Geburtenrate
\item starker Anstieg der Staatsverschuldung\mar{Was soll der übrige
zusammengeklaubte Quatsch auf dem Arbeitsblatt, bei dem zudem die
Bewertung am moderenen Demokratiebegriff fehlt?}
\end{itemize}

\endinput
