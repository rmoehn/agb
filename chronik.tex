\chapter{Chronologie}\label{cap:Chronik}

\begin{chronik}
\item[1640\,--\,1688]
Große Revolution in England

\item[ab ca. 1800]
Bevölkerungsexplosion

\item[1800\,--\,1835]
Frühindustrialisierung

\item[1803]
\beg{Reichsdeputationshauptschluss}

\item[1806]
Gründung des \Beg{}{Rheinbund}es

\item[1807]
Oktoberedikt

\item[1810]
allgemeine Gewerbesteuer

\item[1811]
Regulierungsedikt

\item[1814/15]
Wiener Kongress

\item[Oktober 1817]
Wartburgfest

\item[1.\,1.\,1819]
Einführung einheitlicher Zolltarife in Preußen

\item[August 1819]
Karlsbader Beschlüsse

\item[Juli 1830]
Julirevolution in Frankreich

\item[27.\,--30.\,5.\,1832]
Hambacher Fest

\item[Juli 1832]
Zehn Artikel Metternichs

\item[1.\,1.\,1834]
Zollvereinigungsvertrag

\item[1835\,--\,1873]
Take-off-Phase der Industrialisierung

\item[1844]
Aufstand der schlesischen Weber

\item[Februar 1848]
Februarrevolution in Frankreich

\ \\

\item[1848/49]
Märzrevolution in Deutschland

\item[13.\,--\,15.\,3.\,1848]
Ausbruch der Revolution in Wien

\item[18.\,--\,19.\,3.\,1848]
Barrikadenkämpfe in Berlin

\item[30.\,3.\,1848]
Frankfurter Vorparlament

\item[April 1848]
Wahlen zur Nationalversammlung

\item[Mai 1848]
Eröffnung der Nationalversammlung

\item[Mai\,--\,Okt. 1848]
Arbeit an den Grundrechten

\item[Oktober 1848]
Kaisertruppen gewinnen Wien zurück

\item[10.\,1848\,--\,3.\,1849]
Arbeit an der Verfassung

\item[November 1848]
Preußische Truppen besetzen Berlin

\item[28.\,3.\,1949]
Annahme der Verfassung und Wahl des Kaisers

\item[28.\,4.\,1949]
Ablehnung der Kaiserkrone durch den preußischen König

\item[Juli 1849]
Abreise zahlreicher Abgeordneter, \emph{Rumpfparlament}

\item[18.\,6.\,1849]
Auflösung des Rumpfparlaments durch das Militär

\ \\

\item[1858]
Beginn der \Beg{}{Neuen Ära}

\item[1861]
Gründung der \Ins{}{Deutschen Fortschrittspartei}

\item[1861]
\Nam{}{Wilhelm \Rm{1}} besteigt den preußischen Thron

\item[1864]
Deutsch-Dänischer Krieg

\item[Juni/Juli 1866]
Preußisch-Österreichischer Krieg

\item[1869]
Koalitionsrecht für Preußen

\item[1870/71]
Deutsch-Französischer Krieg

\item[1870\,--\,1873]
\beg{Gründerjahre}

\item[18.\,1.\,1871]
Proklamation des Deutschen Reichs

\item[1871]
Koalitionsrecht für das Deutsche Reich

\item[1872]
Kanzelparagraph

\item[1873\,--\,1900]
Zweite Industrialisierung

\item[1873]
Maigesetze

\item[1874]
Klostergesetz

\item[1874/75]
Einführung der Zivilehe

\item[1878]
\Ges{}{Gesetz gegen die gemeingefährlichen Bestrebungen der
Sozialdemokratie}

\item[1878]
\emph{Weihnachtsbrief} \Nam{}{Bismarck}s

\item[1879]
Einführung von Schutzzöllen im Deutschen Reich

\item[1879] \ges{Zweibund}

\item[1883]
Einführung der Krankenversicherung in Preußen

\item[1884]
Einführung der Unfallversicherung in Preußen

\item[1889]
Einführung der Alters- und Invaliditätsversicherung in Preußen

\item[1890]
Aufhebung des Sozialistengesetzes, Vereinsgesetz für Preußen

\item[ab 1890]
stärkere Konzentration auf Frauen-, Kinder- und Jugendschutz

\item[1892] Militärkonvention zwischen Russland und Frankreich

\item[ab 1898] Flottenpolitik: Ausbau der deutschen Flotte

\item[1902] geheimes italienisch-französisches Neutralitätsabkommen

\item[1904] \ges{Entente Cordiale}

\item[1904/05] russisch-japanischer Krieg

\item[1905/1906] \ins{Erste Marokkokrise}

\item[1907] \ges{Triple Entente}

\item[1908] \ins{Balkankrise}

\item[1910]
Erfindung des \emph{Haber-Bosch-Verfahrens}

\item[1911] \ins{Zweite Marokkokrise}

\item[1911]
Zusammenfassung der Sozialgesetze Bismarcks zur
\Ges{}{Reichsversicherungsordnung} (RVO)

\item[28. Juni 1914] Attentat in Sarajewo

\item[28. Juli 1914] Kriegserklärung Österreich-Ungarns an Serbien

\item[1. August 1914] Kriegserklärung des Deutschen Reichs an Russland

\item[3. August 1914] Kriegserklärung des Deutschen Reichs an Frankreich

\item[1914]
beginnende Inflation in Deutschland

\item[1916]
\Ges{}{Hilfsdientsgesetz}

\ \\

\item[3.\,10.\,1918]
Ernennung \Nam{}{Max von Baden}s zum Reichskanzler

\item[18.\,10.\,1918]
Deutschland wird parlamentarische Monarchie

\item[29.\,10.\,1918]
Auslaufbefehl an die Hochseeflotte in Richtung Themsemündung, Meuterei

\item[3.\,11.\,1918]
Ausdehnung des Aufstands nach Kiel

\item[ab 4.\,11.\,1918]
Bildung von Arbeiter- und Soldatenräten

\item[6.\,11.\,1918]
Ausdehnung des Aufstands auf alle großen Nordseehäfen

\item[7.\,11.\,1918]
Sturz des ersten Throns in Münchens, Ausrufung der Republik

\item[9.\,11.\,1918]
Demonstrationen der kriegsmüden Bevölkerung in Berlin für Frieden und
Abdankung des Kaisers

\item[9.\,11.\,1918]
eigenmächtige Erklärung der Abdankung des Kaisers durch \Nam{}{Max von
Baden}

\item[9.\,11.\,1918]
Republiksausrufungen durch SPD und Spartakus

\item[11.\,11.\,1918]
Unterzeichnung des Waffenstillstandsabkommens

\item[15.\,11.\,1918]
\Nam{}{Stinnes}-\Nam{}{Legien}-Abkommen

\item[1.\,1.\,1919]
Gründung der KPD

\item[Januar 1919]
Beginn der Friedensverhandlungen in \Ort{}{Versailles}

\item[6.\,2.\,1919]
Zusammentritt des Parlaments in Weimar

\item[11.\,2.\,1919]
Wahl \Nam{}{Friedrich Ebert}s zum
Reichspräsidenten

\item[13.\,2.\,1919]
Wahl \Nam{}{Philipp Scheidemann}s zum
Reichskanzler

\item[23.\,6.\,1919]
Übergabe der Friedensbedingungen an die deutsche Delegation

\item[28.\,6.\,1919]
Unterzeichnung des \Ges{}{Friedensvertrags von Versailles}

\item[31.\,7.\,1919]
Annahme der \Ges{}{Weimarer Verfassung}

\ \\

\longitem[1919\,--\,November 1923]
Separatismus im Westen der Weimarer Republik

\item[1918]
Einführung des \emph{Achtstundentags}

\item[1920]
\Nam{}{Kapp}-\Nam{}{Lüttwitz}-Putsch

\item[1922]
Vertrag von \Ort{}{Rapallo}

\item[11.\,1.\,--\,26.\,9.\,1923]
Ruhrkampf

\item[8./9.\,11.\,1923]
\Nam{}{Hitler}-Putsch

\item[15.\,11.\,1923]
Einführung der \Ins{}{Rentenmark}

\item[1924\,--\,1928]
\emph{Goldene Zwanziger}

\item[1924]
\Nam{}{Dawes}-Plan

\item[1925]
\Ort{}{Locarno}-Pakt

\item[1925]
Gründung der \textsf{\Ins{}{I.\,G. Farbenindustrie AG}}

\item[1926]
Aufnahme des Deutschen Reichs in den Völkerbund

\item[1927]
Einführung der Arbeitslosenversicherung

\item[1929]
\Nam{}{Young}-Plan

\item[25.\,10.\,1929]
\emph{Schwarzer Freitag}

\longitem[29.\,3.\,1930\,--\,30.\,5.\,1932]
Kabinett \Nam{}{Brüning}

\longitem[1.\,6.\,--\,17.\,11.\,1932]
Kabinett \Nam{}{von Papen}

\longitem[3.\,12.\,1932\,--\,29.\,1.\,1933]
Kabinett \Nam{}{von Schleicher}

\ \\

\item[30.\,1.\,1933]
Machtübernahme durch \Nam{}{Hitler} und die NSDAP

\item[23.\,3.\,1933]
\Ges{}{Gesetz zur Behebung der Not von Volk und Reich}

\item[31.\,3.\,1933]
Entmachtung der Länderparlamente

\item[1.\,--\,3.\,4.\,1933]
befristeter Boykott jüdischer Geschäfte

\item[7.\,4.\,1933]
\Ges{}{Gesetz zur Wiederherstellung des Berufsbeamtentums}

\item[7.\,4.\,1933]
Möglichkeit des Entzugs der Zulassung jüdischer Anwälte

\item[April 1933]
\Ges{}{Gesetz gegen die Überfüllung der deutschen Schulen und
Hochschulen}

\item[2.\,5.\,1933]
Zerschlagung der Gewerkschaften

\item[2.\,5.\,1933]
Zerschlagung der Gewerkschaften

\item[6.\,5.\,1933]
Gründung der \Ins{}{Deutschen Arbeitsfront}

\item[Mai 1933]
Bücherverbrennung

\item[30.\,6.\,1933]
\Nam{}{Röhm}-Revolte

\item[14.\,7.\,1933]
\Ges{}{Gesetz zur Verhütung erbkranken Nachwuchses}

\item[Juli 1933]
\Ges{}{Reichskonkordat} mit dem Papst

\item[September 1933]
Gründung des Winterhilfswerks

\item[29.\,9.\,1933]
\Ges{}{Reichserbhofgesetz}

\item[5.\,10.\,1933]
\Ges{}{Schriftleitergesetz}

\item[Oktober 1933]
Austritt aus dem Völkerbund

\item[November 1933]
Gründung der Organisation \Ins{}{Kraft durch Freude}

\item[26.\,6.\,1935]
Legalisierung des Schwangerschaftsabbruchs bei diagnostizierter
Erbkrankheit

\item[15.\,9.\,1935]
\Ges{}{Gesetz zum Schutz des deutschen Blutes und der deutschen Ehre}

\item[18.\,10.\,1935]
\Ges{}{Ehegesundheitsgesetz}

\item[1934\,--\,1937]
\Beg{}{MEFO}-Wechselsystem

\item[1934]
Pause von antijüdischen Gesetzen

\item[Januar 1934]
deutsch-polnischer Nichtangriffspakt

\item[1.\,8.\,1934]
Herstellung des Führerstaates

\item[Februar 1935]
Rückgliederung des Saarlands

\item[Juni 1935]
deutsch-britisches Flottenabkommen

\item[Juni 1935]
Einführung von Reichsarbeits- und Wehrdienst

\item[15.\,9.\,1935]
Nürnberger Gesetze

\item[März 1936]
Besetzung des \Ort{}{Rheinlands} durch deutsche Truppen

\item[1936]
Olympische Spiele in Deutschland

\item[November 1936]
Antikominternpakt

\item[März 1938]
Anschluss Österreichs

\item[September 1938]
Münchener Abkommen

\item[9./10.\,11.\,1938]
Reichspogromnacht

\item[August 1939]
\Nam{}{Hitler}-\Nam{}{Stalin}-Pakt

\item[1.\,9.\,1939]
\emph{Geheimer Führererlass}: Ermächtigung zur Durchführung der
Euthanasie

\item[1.\,9.\,1939]
deutscher Überfall auf Polen -- Beginn des \emph{Zweiten Weltkriegs}

\item[1.\,9.\,1941]
Tragen des sogenannten \emph{Judensterns} Pflicht

\item[20.\,4.\,1942]
Wannseekonferenz

\ \\

\item[8.\,5.\,1945]
Offizielles Ende des \emph{Zweiten Weltkriegs}

\item[5.\,6.\,1945]
Berliner Erklärung

\item[Juli 1945]
Zulassung von Parteien in der \textsf{\Ort{}{SBZ}}

\item[Juli 1945]
Gründung des \Ins{Block antifaschistisch-demokratischer
Parteien}{Blocks antifaschistisch-demokratischer Parteien}

\item[17.\,7.\,--\,2.\,8.\,1945]
\Ort{}{Postdamer} Konferenz

\item[Februar 1946]
Gründung des \Ins{}{FDGB}

\item[22.\,4.\,1946]
Vereinigung von SPD und KPD zur SED

\item[März 1948]
Zweiter Deutscher Volkskongress

\item[Juli 1948]
Überreichung der \Ges{}{Frankfurter Dokumente}

\item[August 1948]
Erarbeitung des Verfassungsentwurfs für die BRD

\item[19.\,3.\,1949]
Verabschiedung des Verfassungsentwurfes durch den \Ins{}{Deuschen
Volksrat}

\item[8.\,5.\,1949]
Verabschiedung des Grundgesetzes durch den \Ins{}{Parlamentarischen Rat}

\item[15./16.\,5.\,1949]
Wahl des Dritten Deutschen Volkskongresses

\item[23.\,5.\,1949]
Inkrafttreten des \Ges{}{Grundgesetzes für die Bundesrepublik
Deutschland}

\item[14.\,8.\,1949]
Wahl des ersten \Ins{}{Deutschen Bundestages}

\item[7.\,10.\,1949]
Inkraftsetzung der \Ges{}{Verfassung der Deutschen Demokratischen
Republik}

\item[15.\,10.\,1950]
erste Volkskammerwahlen in der DDR
\end{chronik}

\endinput
