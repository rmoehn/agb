\section{Außenpolitik Wilhelms II.}
\mar{Wilhelm: Kriegsverträge}

Nach der Entlassung \Nam{Bismarck, Otto von}{Bismarck}s \dat{1890} kam
es zu einer Neugestaltung der Außenpolitik. Der geheime
Rückversicherungsvertrag mit Russland wurde nicht verlängert.
\nam{Wilhelm \Rm{2}}, stark dem Militarismus verhaftet, sah das
Deutsche Reich im Aufbruch und erstrebte einen Aufstieg zur Weltmacht.
Diese kolonialistische beziehungsweise imperialistische Ausrichtung
fand breite Zustimmung in der deutschen Wirtschaft und beim deutschen
Militär. Bei den etablierten Großmächten Europas stieß der Kaiser mit
seinen Zielen allerdings auf Widerstand.

\begin{chronik}
\item[1890] keine Verlängerung des geheimen
\ges{Rückversicherungsvertrag}{Rückversicherungsvertrags} zwischen dem
Deutschen Reich und Russland

\item[1892] Militärkonvention zwischen Russland und Frankreich

\item[ab 1898] Flottenpolitik: Ausbau der deutschen Flotte --
Konfliktpotential mit Großbritannien

\item[1902] geheimes italienisch-französisches Neutralitätsabkommen --
faktisches Ende des Dreibunds (eigentlich bis 1815)

\item[1904] \ges{Entente Cordiale} zwischen dem Vereinigten Königreich
und Frankreich: Verzicht Frankreichs auf Ägypten, freie Hand für
Frankreich im nicht kolonialisierten Frankreich $\Rightarrow$
Beilegung der britisch-französischen Kolonialkonflikte

\item[1904/05] russisch-japanischer Krieg:  Plan des deutschen
Generalstabs für einen Krieg gegen Frankreich (\ges{Schlieffenplan})
wegen der geschwächten russisch-französischen Allianz -- Kaiser
dagegen

\item[1905/1906] \ins{Erste Marokkokrise}: Das Deutsche Reich engagiert
sich in Marokko gegen Frankreich -- aktiver Eingriff in die
Kolonialpolitik

\item[1907] \ges{Triple Entente}: Erweiterung der Entente Cordiale
durch Russland als Folge der russisch-britischen Annäherung

\item[1908] \ins{Balkankrise}: Russland versuchte von
Österreich-Ungarn die Zustimmung für eine freie Durchfahrt durch die
türkischen Meerengen zu erlangen.  Es bot im Gegenzug an, beim
Gelingen der Durchfahrt Österreich die formelle Anexion Bosniens und
Herzegowinas zu gestatten. Russland erhielt aber die für das Vorhaben
benötigte Zustimmung Englands und Frankreichs nicht. Österreich
annektierte aber schon während der Verhandlungen Bosnien und
Herzegowina und geriet dadurch in Spannungen mit Serbien, welches von
Russland protegiert wurde. Das Deutsche Reich stellte sich hinter
Österreich und fordert, Serbien zurückzuhalten und die Annexion
anzuerkennen. Dem gab Russland nach, da es nach dem verlorenen Krieg
gegen Japan geschwächt war.

Russland strebte nach der Hegemonie in Südosteuropa und betrieb eine
panslawistische Politik, indem es sich als Schutzmacht der slawischen
Völker sah.

\item[1911] \ins{Zweite Marokkokrise} -- England kündigt die
Unterstützung Frankreichs gegen Bedrohungen an.
\end{chronik}
