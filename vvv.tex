\section{Der \ins{Vertrag von Versailles}}
\label{sec:vvv}
\index{Erster Weltkrieg}
\index{Friedensvertrag!Versailles}

Die Verhandlungen über den Friedensvertrag der Alliierten des Ersten
Weltkrieges mit Deutschland \ort{begannen im Januar 1919}
\ort{Versailles}. Die Bedingungen wurden am \dat{23. Juni 1919} an die
deutsche Delegation, die an den vorangehenden Zusammenkünften nicht
teilnehmen durfte, übergeben und am \dat{28. Juni} im Spiegelsaal
\index{Spiegelsaal} von Versailles unterzeichnet.

\subsection*{Gebietsregelungen}

\begin{itemize}
\item \ort{Elsaß-Lothringen} geht an Frankreich.
\item \ort{Westpreußen} geht an Polen.
\item Schaffung eines polnischen Korridors zur Ostsee
\item \ort{Danzig} wird freie Stadt; Polen erhält Sonderrechte im
Hafen von Danzig
\item Das \ort{Memelgebiet} geht an Litauen, später an die
Sowjetunion.
\item \ort{Eupen-Malmedy} geht an Belgien.
\item Das \ort{Saargebiet} wird dem \ins{Völkerbund} unterstellt.
\item Die \ins{Saargruben} gehen zur Ausbeutung 15 Jahre an
Frankreich.
\item Die deutschen Kolonien\index{Kolonien!deutsch} werden dem
Völkerbund unterstellt.
\end{itemize}

\noindent $\Longrightarrow$ Deutschland verliert 10\,\% seiner
Bevölkerung und 13\,\% seiner Fläche


\subsection*{Politische Regelungen}

\begin{itemize}
\item Verbot eines Zusammenschluss des Deutschen Reiches mit
Österreich (eigentlich durch den \ges{Vertrag von Saint
Germain}\Ort{Saint Germain}{}
\item Das Deutsche Reich erhält mit dem §\,231 die alleinige
Kriegsschuld und daraus abgeleitet die Pflicht zur Zahlung aller
Reparationen. \index{Kriegsschuldparagraph}\index{Reparationen!Erster
Weltkrieg} 
\end{itemize}


\subsection*{Militärische Regelungen}

\begin{itemize}
\item Reduzierung der deutschen Armee auf 100\,000 Mann, der Marine
auf 15\,000 Mann.
\item Verbot von Flugzeugen, Panzern, U-Booten und schweren
Kriegsschiffen
\item Auslieferung der Kriegsflotte
\item Verbot eines Generalstabes\index{Generalstab}
\item militärische Besetzung des linken \Ort{Rheinufer}{Rheinufers}
durch Frankreich, einige rechtsrheinische
Brückenköpfe\index{Rheinufer!Brückenköpfe} für 15 Jahre
\item Entmilitarisierung eines rheinischen Streifens
\end{itemize}


\subsection*{Wirtschaftliche Regelungen}

\begin{itemize}
\item durch die Gebietsabtrennungen betroffen: Bergbau,
Stahlindustrie, Hüttenwesen, Landwirtschaft
\item \dat{1921 Fixierung} der Reparationen auf 132 Milliarden
Reichsmark\index{Reparationen!Erster Weltkrieg}
\end{itemize}


\subsection*{Folgen in Deutschland}

\begin{itemize}
\item Austritt der \ins{DDP} aus der \ort{Weimarer Koalition}

\item Rücktritt des Ministerpräsidenten \Nam{Scheidemann,
Philipp!Rücktritt}{Scheidemann} und des Außenministers
\Nam{Brockdorff-Rantzau, Ulrich von!Rücktritt}{von Brockdorff-Rantzau}

\item gleichfalls Rücktritt des Reichswehrministers \Nam{Noske,
Gustav!Rücktritt}{Noske} (\ins{SPD}) und des verfassungstreuen Generals
und Chefs der \ins{Heeresleitung} \Nam{Reinhardt,
Walther!Rücktritt}{Reinhardt} -- Neubesetzung der Stellen durch den
\beg{Vernunftrepublikaner} \Nam{Gessler, Otto}{Gessler} (DDP) und den
Antirepublikaner General \Nam{Seeckt, Hans von}{von Seeckt}

\item Hohe Reparationszahlungen bremsen den Wiederaufbau und führen zu
Unzufriedenheit, Armut und Arbeitslosigkeit. Die übertriebenen
Ansprüche der Alliierten führen schließlich zum
\emph{Ruhrkampf}\index{Ruhrkampf} (siehe \ref{sec:krisj-1923}).
\end{itemize}



\endinput
