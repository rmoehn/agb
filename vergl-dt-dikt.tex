\section{Das Problem der Vergleichbarkeit der beiden Diktaturen in
Deutschland}

\dictum[Margherita von Brentano]{Der
bloße Vergleich des Dritten Reiches mit der DDR ist eine schreckliche
Verharmlosung. Das Dritte Reich hinterließ Berge von Leichen. Die DDR
hinterließ Berge von Karteikarten.}\mar{Das ist der Ausgangspunkt.}


\subsection*{Probleme im Vergleichsansatz}

\begin{itemize}
\item Vergleich beinhaltet Gefahr der Gleichsetzung -- Gefahr der
Verharmlosung des Nationalsozialismus

\item Singularität des Nationalsozialismus

\item Gefahr der Verharmlosung der DDR (menschliche Schicksale hinter
den \enquote{Karteikarten}) 
\end{itemize}


\subsection*{Prämissen}

\begin{enumerate}
\item Vergleich darf nicht zur Gleichsetzung geraten
\item \emph{Versuch} eines Vergleichs anhand geeigneter Kriterien
\item Wertung/Auseinandersetzung der/mit der Vergleichbarkeit der
Diktaturen
\end{enumerate}


\subsection*{Vergleichende Darstellung}

{
% joy with calculation
\newlength{\frstcolwdth}
\settowidth{\frstcolwdth}{Selbstdar-}

\newlength{\dummywdth}
\newlength{\othcolwidth}
\newlength{\fieldwidth}
\setlength{\dummywdth}{\textwidth}
\addtolength{\dummywdth}{-\frstcolwdth}
\addtolength{\dummywdth}{-2\tabcolsep}
\setlength{\othcolwidth}{0.25\dummywdth}
\addtolength{\othcolwidth}{-2\tabcolsep}
\setlength{\fieldwidth}{2\othcolwidth}

\newcolumntype{x}{p{\othcolwidth}}
\newcommand{\dublcol}[1]{\multicolumn{2}{p{\fieldwidth}}{#1}}

\tablefirsthead{%
\toprule
Kriterium & 
\multicolumn{2}{c}{Nationalsozialismus} & \multicolumn{2}{c}{DDR} \\
\midrule
}

\tabletail{\midrule}
\tablehead{\midrule}

\tablelasttail{\bottomrule}

\renewcommand*{\arraystretch}{0.8}

%%%%%%%%%%%%%%%%%%%%%%%%%%%%%%%%%%%%%%%%%%%%%%%%%%%%%%%%%%%%%%%%%%%%%%


\footnotesize
\begin{supertabular*}{\textwidth}{p{\frstcolwdth}xxxx}

\vspace{0em}Herr"-schafts"-struk"-tur &
&
\dublcol{
%%\vspace{-0.74em}
\begin{tablist}
\setlength{\listparindent}{0.2\columnwidth}
\item nur einseitige Partizipationsmöglichkeiten
\item keine Gewaltenteilung
\item Totalitarismus
\item Staatssicherungsorganisationen (\ins{Gestapo}, \Ins{Stasi,
Ministerium für Staatssicherheit}{Stasi})
\item Einheitsorganisationen
\end{tablist}}
&
\\

&
\dublcol{
%\vspace{-0.74em}
\begin{tablist}
\item Gleichschaltung der Länder, Ausschaltung des Parlaments --
Zentralismus
\item alleinige Herrschaft einer Person
\item keine Wahlen seit August 1934
\item nur eine Partei zugelassen
\item keine Verfassung
\end{tablist}
}
&
\dublcol{
%\vspace{-0.74em}
\begin{tablist}
\item weniger ausgeprägter Zentralismus
\item Herrschaft der Partei
\item Wahl nach Einheitsliste
\item unterschiedliche Parteien -- Zusammenfassung im Block 
\end{tablist}
}
\\

%%%%%%%%%%%%%%%%%%%%%%%%%%%%%%%%%%%%%%%%%%%%%%%%%%%%%%%%%%%%%%%%%%%%%%

\vspace{0em}Selbst"-dar"-stel"-lung &
&
\dublcol{
%\vspace{-0.74em}
\begin{tablist}
\item Vereinnahmung der Jugend
\item Vereinnahmung der Kultur 
\item Propaganda
\item Intransparenz
\item Militär
\end{tablist}
}
&
\\

&
\dublcol{
%\vspace{-0.74em}
\begin{tablist}
\item Rechtfertigung der Expansionspolitik, des Judenmords und der
Euthanasie mit der Ideologie
\item Autarkie
\item Führerkult
\item Nationalismus -- Chauvinismus
\end{tablist}
}
&
\dublcol{
%\vspace{-0.74em}
\begin{tablist}
\item Präsentation als \jar{antifaschistischer Friedensstaat} --
Antikapitalismus, Klassenkampf
\item Partei
\end{tablist}
}
\\

%%%%%%%%%%%%%%%%%%%%%%%%%%%%%%%%%%%%%%%%%%%%%%%%%%%%%%%%%%%%%%%%%%%%%%

\vspace{0em}ideologische Grundlagen/In"-sti"-tu"-tio"-nen &
&
\dublcol{
%\vspace{-0.74em}
\begin{tablist}
\item Einheitsorganisationen
\item Militär
\item Staatssicherungorganisationen -- Angst, Terror
\item Gleichschaltung
\end{tablist}
}
&
\\

&
\dublcol{
%\vspace{-0.74em}
\begin{tablist}
\item Machtkampf der Minister und Ministerien
\item \Ins{BDM, Bund Deutscher Mädel}{BDM}, \Ins{HJ, Hitlerjugend}{HJ}
(Militär, Haushalt)
\end{tablist}
}
&
\dublcol{
%\vspace{-0.74em}
\begin{tablist}
\item Antifaschismus 
\item Einbindung in einen Militärblock
\item \ins{Pionierorganisation \enquote{Ernst Thälmann}}, \Ins{FDJ,
Freie Deutsche Jugend}{FDJ} (politische Betätigung)
\end{tablist}
}
\\

%%%%%%%%%%%%%%%%%%%%%%%%%%%%%%%%%%%%%%%%%%%%%%%%%%%%%%%%%%%%%%%%%%%%%%

\vspace{0em}Rolle der Bevölkerung &
&
\dublcol{
%%\vspace{-0.73em}
\begin{tablist}
\item Befürworter und Gegner
\item Regime auf Bevölkerung angewiesen
\item wenig Widerstand
\item erzwungene Haltung
\end{tablist}
}
&
\\

&
\dublcol{
\vspace{0.2em}
Nationalsozialismus aus der Bevölkerung heraus entstanden 
}
&
\dublcol{
\vspace{0.2em}
DDR-System von der UdSSR angelegt und stärker von der Ideologie
durchdrungen.
}
\\

%%%%%%%%%%%%%%%%%%%%%%%%%%%%%%%%%%%%%%%%%%%%%%%%%%%%%%%%%%%%%%%%%%%%%%

\vspace{0.5em}Umgang mit Gegnern &
&
\dublcol{
\vspace{0.50em}
\begin{tablist}
\item Gegner: Andersdenkende (mit einzelnen Dingen nicht
einverstanden), Gegner der Ideologie (mit dem gesamten System nicht
einverstanden)
\item Haft 
\end{tablist}
}
&
\\

&
\dublcol{
%\vspace{-0.44em}
\begin{tablist}
\item Morde, Konzentrationslager
\item SS, Gestapo
\item keine Verhandlungen
\item sehr viele Todesopfer
\item Sippenhaft 
\end{tablist}
}
&
\dublcol{
%\vspace{-0.44em}
\begin{tablist}
\item Mauertote, Tote beim Volksaufstand
\item Ministerium für Staatssicherheit
\item gesellschaftliche Benachteiligung
\end{tablist}
}
\\

\end{supertabular*}
}


\subsection*{Wertung}

\dictum[Alfred Grosser]{Selbst die Feststellung eines völligen
Gegensatzes kann immer nur das Ergebnis eines Vergleichs sein.}

Man kann erkennen, dass zahlreiche Parallelen zwischen den beiden
Diktaturen existieren, die aber doch nur Parallelen sind und keine
Zusammenfallenden. Denn die Ereignisse stehen auf vollkommen
unterschiedlichen Stufen. -- So findet der Völkermord der
Nationalsozialisten seine Entsprechung gerade einmal in den Handlungen
des Ministeriums für Staatssicherheit der DDR und dem Führerprinzip
steht die Parteidiktatur gegenüber. Das nationalsozialistische Regime
ist singulär; Holocaust und Zweiter Weltkrieg finden keinerlei
Entsprechung in der DDR.

Außerdem gäbe es ein Problem, wenn man vorgäbe, keinen Vergleich
vorzunehmen: Man vergliche dennoch, indem man beide Systeme einfach
als Diktaturen einordnete und damit gleichsetzte. Ein Vergleich ist
also notwendig. Gemeinsamkeiten und Unterschiede müssen allerdings
klar und sorgfältig herausgearbeitet werden.
