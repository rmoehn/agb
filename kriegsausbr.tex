\section{Ausbruch des Ersten Weltkriegs}

\subsection*{Anlass}

\begin{chronik}
\item[28. Juni 1914] Ermordung des österreichischen Thronfolgers
\nam{Franz Ferdinand} durch einen serbischen Nationalisten in
\ort{Sarajewo} in Bosnien

\item[28. Juli 1914] Kriegserklärung Österreich-Ungarns an Serbien --
Auslösung des Mechanismus' der Bündnisverpflichtungen (Panslawismus,
Zweibund, Militärkonvention), Mobilmachung Russlands

\item[1. August 1914] Kriegserklärung des Deutschen Reichs an Russland

\item[3. August 1914] Kriegserklärung des Deutschen Reichs an Frankreich
\end{chronik}


\subsection*{Gründe}

\begin{itemize}
\item Bündnisverpflichtungen
\item imperiales Weltmachtstreben, Kolonialismus -- Konflikt der
nach Erhaltung der Ordnung strebenden traditionellen Kolonialmächte
Frankreich und Vereinigtes Königreich mit den aufstrebenden neuen
Kolonialmächte Deutsches Reich, Japan und USA

\item Bereits aufgerüstete Wirtschaft und Militär der Weltmächte
drängen auf Krieg.

\item Flottenwettrüsten $\Rightarrow$ Verschärfung des Konflikts
zwischen dem Deutschen Reich und dem Vereinigte Königreich

\item Panslawismus

\item Erzrivalität zwischen dem Deutschen Reich und Frankreich,
französischer Nationalismus mit Verlangen nach einer Revanche für den
Deutsch-Französischen Krieg
\end{itemize}
