\section[Volksaufstand in der DDR]{Volksaufstand in der
DDR\mycite{Dtl50erJ}}
\index{Voksaufstand}
\index{17. Juni 1953}

\subsection*{Ursachen}

\begin{itemize}
\item Militärblock- und Rüstungspolitik

\item Gründung der \Ins{Stasi, Ministerium für
Staatssicherheit}{Stasi} 1950 

\item wirtschaftliche Probleme:
\begin{itemize}
\item Planwirtschaft führt zu Priorisierung der Schwerindustrie und zu
Mangel in der Lebensmittel- und Konsumgüterproduktion
\item Rohstoffknappheit zwingt zu Importen
\item Reparationen
\item Ablehnung der Verstaatlichungsmaßnahmen (kaum noch
Privatwirtschaft), besonders in der Landwirtschaft führt zur
Auswanderung von Bauern und so zur weiteren
Verschärfung der Probleme.
\item Erhöhung der Arbeitsnorm
\end{itemize}

\item Unklarheit über die zukünftige Politik nach \Nam{Stalin,
Josef}{Stalin}s Tod am 5. Juni 1953

\item Erzwingung eines \jar{Neuen Kurses} durch die Sowjetunion am 9.
Juni:
\begin{itemize}
\item politische Lockerungen
\item Zugeständnisse an die Bauern und den Mittelstand
\item keine Rücknahme der Normerhöhung
\end{itemize}
\end{itemize}

%%%%%%%%%%%%%%%%%%%%%%%%%%%%%%%%%%%%%%%%%%%%%%%%%%%%%%%%%%%%%%%%%%%%%%

\subsection*{Forderungen}

Arbeiter waren die Initiatoren und Hauptträger des Aufstandes.
Deswegen war auch die Forderung nach der Rücknahme der Normerhöhung
tonangebend. Im Verlaufe der Proteste wurden aber auch politische
Probleme angeschnitten: Rücktritt der Regierung, Wiedervereinigung
Deutschlands, freie Wahlen, freie Parteien und Gewerkschaften.

%%%%%%%%%%%%%%%%%%%%%%%%%%%%%%%%%%%%%%%%%%%%%%%%%%%%%%%%%%%%%%%%%%%%%%

\subsection*{Verlauf}

Der Aufstand \dat{begann} bereits am \dat{16. Juni 1953} mit
Arbeitsniederlegungen an zwei berliner Großbaustellen
(\ort{Stalinallee}). Die Beteiligten zogen vor das Politbüro, wo
Industrieminister \Nam{Selbmann, Friedrich}{Selbmann} zu
beschwichtigen versuchte und die Normerhöhung zurücknahm. Dies war
jedoch vergeblich und die Protestierenden riefen über Boten und
Rundfunk auch die übrige Bevölkerung zum Streik auf.

So kam es am \dat{17. Juni} zu \dat{flächendeckendem Aufbegehren} in
cat 560 Städten der DDR -- hauptsächlich Industriestandorte wie
\ort{Leuna}, \ort{Wolfen} und \ort{Jena}. Die Demonstranten befreiten
Häftlinge aus den Gefängnissen, es kam zu Zusammenstößen mit der
Polizei. Darauf verhängte die \Ins{SMAD, Sowjetische
Militäradministration}{SMAD} am Mittag in den großen Städten den
Ausnahmezustand. Mithilfe sowjetischer Panzer wurde der Aufstand
blutig niedergeschlagen.

%%%%%%%%%%%%%%%%%%%%%%%%%%%%%%%%%%%%%%%%%%%%%%%%%%%%%%%%%%%%%%%%%%%%%%

\subsection*{Folgen}

\begin{itemize}
\item ca. 21 Tote, zahlreiche Verletzte 
\item Verfolgung und Inhaftierung der \jar{Rädelsführer} und andere
Beteiligter -- ca. 1400 Verurteilungen zu langjährigen
Zuchthausstrafen
\item \jar{Reinigung} der SED von \jar{feindlichen Elementen}
\item Festigung der Machposition \Nam{Ulbricht, Walter}{Walter
Ulbricht}s
\item anhaltende Furcht der Staatsführung vor einer Wiederholung des
17. Juni
\item Preissenkungen der \Ins{HO, Handelsorganisation}{HO}
\item Massenfluchtbewegung in den Westen $\Rightarrow$ Mauerbau (siehe
\ref{sec:mauerbau})
\item Ende der Reparationsentnahmen aus der laufenden Produktion durch
die UdSSR 1954
\end{itemize}

%%%%%%%%%%%%%%%%%%%%%%%%%%%%%%%%%%%%%%%%%%%%%%%%%%%%%%%%%%%%%%%%%%%%%%

\subsection*{Bewertung anhand des modernen Demokratiebegriffs}

\begin{itemize}
\item weitere Freiheitseinschränkungen
\item Beschluss der Niederschlagung allein durch die Exekutive
\item brutale Unterdrückung der Volkssouveränität
\end{itemize}

$\Longrightarrow$ keine rechtsstaatliche Lösung 

\endinput
