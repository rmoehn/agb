\part{Demokratie und Diktatur -- Anspruch und Wirklichkeit von
Gesellschaftsmodellen in der zweiten Hälfte des 20. Jahrhunderts}
\label{prt:dem-dikt2}
\thispagestyle{empty}

\chapter{Neubeginn im besiegten Land}
\label{sec:neubeginn}

\begin{chronik}
\item[8.\,5.\,1945]
bedingugslose Kapitulation Deutschlands -- offizielles Ende des
\emph{Zweiten Weltkriegs} \index{Zweiter Weltkrieg!Ende}

\item[5.\,6.\,1945]
\ges{Berliner Erklärung} \index{Berliner Erklärung} (USA,
Großbritannien, Frankreich, UdSSR): Übernahme der Regierungsgewalt und
des Oberkommandos der Wehrmacht durch die Alliierten

\item[17.\,7.\,--\,2.\,8.\,1945]
\ort{Postdamer} Konferenz \index{Potsdamer Konferenz}
\begin{itemize}
\item Aufteilung Deutschlands in vier Besatzungszonen
\index{Besatzungszonen}
\item Betrachtung der deutschen Wirtschaft als Einheit
\item Ziele: Dezentralisierung, Demokratisierung, Entmilitarisierung,
Entnazifizierung \index{Demokratisierung} \index{Dezentralisierung}
\index{Entnazifizierung} \index{Entmilitarisierung}
\item beinahes Scheitern der Verhandlungen über den Fragen der
Reparationen und der Ostgrenze
\end{itemize}
\end{chronik}

\section[Gründung der BRD]{Die Gründung der BRD\footnote{Es
handelt sich hier nur um einen groben Überblick. \label{ftn:grob-ueb}}
\mycite[333, 334]{WeltgeschNeuz}}
\label{sec:gruend-brd}
\index{BRD!Gründung}

\begin{chronik}
\item[Juli 1948]
Überreichung der \ges{Frankfurter Dokumente} an die
Ministerpräsidenten der Länder: Angebot zur Errichtung eines
westdeutschen Bundesstaates, Grundsätze für dessen Verfassung

\item[August 1948]
Erarbeitung eines Verfassungsentwurfes durch eine
Sachverständigenkommission als Beratungsgrundlage für den
\textsc{\Ins{Parlamentarischer Rat}{Parlamentarischen Rat}}

\item[8.\,5.\,1949]
Verabschiedung des Grundgesetzes durch den parlamentarischen Rat

\item[12.\,--\,23.\,5.\,1949]
Zustimmung der Westalliierten und der Bundesländer\footnote{Bayern
ratifizierte das Grundgesetz nicht, erkannte aber dessen
Rechtsverbindlichkeit an.} zum Grundgesetz 

\item[23.\,5.\,1949]
Inkrafttreten des \Ges{Grundgesetz für die Bundesrepublik
Deutschland}{Grundgesetzes für die Bundesrepublik Deutschland}

\item[14.\,8.\,1949]
freie, geheime, gleiche, direkte und allgemeine Wahl des ersten
\Ins{Deutscher Bundestag}{Deutschen Bundestages} durch die
westdeutsche Bevölkerung
\end{chronik}

\section[Gründung der DDR]{Die Gründung der DDR
\footref{ftn:grob-ueb}
\mycite[334, 335]{WeltgeschNeuz}}
\label{sec:gruend-ddr}
\index{DDR!Gründung}

\begin{chronik}
\item[10.\,7.\,1945]
zentrale Zulassung von Parteien in der \textsf{\ort{SBZ}} durch die
\beg{SMAD}

\item[14.\,7.\,1945]
Gründung des \Ins{Block antifaschistisch-demokratischer
Parteien}{Blocks antifaschistisch-demokratischer Parteien} --
Zusammenarbeit von KPD, SPD, CDU und LDPD
\begin{itemize}
\item nur einstimmige Beschlussfassung
\item Bindung für alle Parteien an die Beschlüsse
\item Festlegung eines nicht näher definierten gemeinsamen Weges
\end{itemize}

\item[22.\,4.\,1946]
Auf Druck der sowjetischen Gesatzungsmacht vereinigen sich SPD und KPD
zur SED\index{SED!Gründung}, da man befürchtete, dass die KPD in
freien Wahlen nicht genügend Stimmen gewinne.

\item[Feb. 1946]
Gründung des \textsf{\ins{FDGB}}\index{FDGB!Gründung}
(Einheitsorganisation) -- Während zwar
alle Parteien in der Leitung vertreten waren, sicherten sich die
Kommunisten dennoch die Vorherrschaft.

\item[Mitte März 1948]
Zusammentreten des \Ins{Zweiter Deutscher Volkskongresses}{Zweiten
Deutschen Volkskongresses} bestehend aus 2\,000
Delegierten\footnote{Die Stimmenmehrheit lag bei der SED und deren
Verbündeten} der Blockparteien und der Massenorganisationen
(beispielsweise FDJ und FDGB) auf Initiative der SED, Wahl des
\Ins{Deutscher Volksrat}{Deutschen Volksrates} (400 Mitglieder) --
Aufgabe: Ausarbeitung einer Verfassung für eine \enquote{unteilbare
deutsche Republik}

\item[19.\,3.\,1949]
Verabschiedung eines Verfassungsentwurfes durch den Deutschen Volksrat

\item[15./16.\,5.\,1949]
Wahl des \Ins{Dritter Deutscher Volkskongress}{Dritten Deutschen
Volkskongresses} durch die Bevölkerung der \ort{SBZ} nach
Einheitsliste.

Der \Ins{Dritter Deutscher Volkskongress}{Dritte Deutsche
Volkskongress} wählt den \Ins{Zweiter Deutscher Volksrat}{Zweiten
Deutschen Volksrat}, welcher sich zur provisorischen \ins{Volkskammer}
erklärt und die \ges{Verfassung der Deutschen Demokratischen Republik}
nach Bestätigung durch den \emph{Volkskongress} in Kraft setzt. Der
Präsident der DDR wurde \nam{Wilhelm Pieck}.

\item[7.\,10.\,1949]
Inkraftsetzung der \ges{Verfassung der Deutschen Demokratischen
Republik}

\item[15.\,10.\,1950]
Wahlen zur \ins{Volkskammer} über Einheitsliste\footnote{Artikel 51
der \ges{Verfassung der Deutschen Demokratischen Republik} sah
eigentlich eine Verhältniswahl vor.} durch die ostdeutsche
Bevölkerung
\end{chronik}

\endinput



\chapter{Das Demokratieverständnis in beiden deutschen Staaten}
\label{chp:demverst-beid-staat}
\index{Demokratieverständnis}

\section{Der Demokratiebegriff}
\label{sec:dem-begr}
\index{Demokratie}

\begin{aufgabe}
Untersuchen Sie, inwieweit die Verfassungen der neugegründeten Staaten
dem modernen Demokratiebegriff genügen!
\end{aufgabe}

\subsection*{Einteilung}

\begin{multicols}{2}
\noindent \emph{direkte Demokratie}:
direkte Entscheidungsfindung durch das Volk, beispielsweise in
Volksabstimmungen

\columnbreak

\noindent \emph{repräsentative (indirekte) Demokratie}:
Bürger wählen Repräsentanten, Sachentscheidungen durch Volksvertreter
\end{multicols}

\subsection*{Merkmale}

\begin{itemize}
\item Volkssouveränität (direkt, indirekt oder Mischform),
beispielsweise durch Wahlen oder Bürgerentscheide
\item Gewaltenteilung
\item Kontrolle der Exekutive und Legislative durch neutrale und
politisch unabhängige Judikative
\item Garantie der Menschen- und Bürgerrechte sowie der individuellen
Grundrechte (Schutz von Leben, Freiheit von Eigentum, Recht auf freie
Meinungsäußerung, Versammlungsfreiheit, Pressefreiheit)
\item Meinungspluralismus -- Parteienvielfalt
\item Verfassung
\end{itemize}

\endinput

\section[Volksaufstand in der DDR]{Volksaufstand in der
DDR\mycite{Dtl50erJ}}
\index{Voksaufstand}
\index{17. Juni 1953}

\subsection*{Ursachen}

\begin{itemize}
\item Militärblock- und Rüstungspolitik

\item Gründung der \Ins{Stasi, Ministerium für
Staatssicherheit}{Stasi} 1950 

\item wirtschaftliche Probleme:
\begin{itemize}
\item Planwirtschaft führt zu Priorisierung der Schwerindustrie und zu
Mangel in der Lebensmittel- und Konsumgüterproduktion
\item Rohstoffknappheit zwingt zu Importen
\item Reparationen
\item Ablehnung der Verstaatlichungsmaßnahmen (kaum noch
Privatwirtschaft), besonders in der Landwirtschaft führt zur
Auswanderung von Bauern und so zur weiteren
Verschärfung der Probleme.
\item Erhöhung der Arbeitsnorm
\end{itemize}

\item Unklarheit über die zukünftige Politik nach \Nam{Stalin,
Josef}{Stalin}s Tod am 5. Juni 1953

\item Erzwingung eines \jar{Neuen Kurses} durch die Sowjetunion am 9.
Juni:
\begin{itemize}
\item politische Lockerungen
\item Zugeständnisse an die Bauern und den Mittelstand
\item keine Rücknahme der Normerhöhung
\end{itemize}
\end{itemize}

%%%%%%%%%%%%%%%%%%%%%%%%%%%%%%%%%%%%%%%%%%%%%%%%%%%%%%%%%%%%%%%%%%%%%%

\subsection*{Forderungen}

Arbeiter waren die Initiatoren und Hauptträger des Aufstandes.
Deswegen war auch die Forderung nach der Rücknahme der Normerhöhung
tonangebend. Im Verlaufe der Proteste wurden aber auch politische
Probleme angeschnitten: Rücktritt der Regierung, Wiedervereinigung
Deutschlands, freie Wahlen, freie Parteien und Gewerkschaften.

%%%%%%%%%%%%%%%%%%%%%%%%%%%%%%%%%%%%%%%%%%%%%%%%%%%%%%%%%%%%%%%%%%%%%%

\subsection*{Verlauf}

Der Aufstand \dat{begann} bereits am \dat{16. Juni 1953} mit
Arbeitsniederlegungen an zwei berliner Großbaustellen
(\ort{Stalinallee}). Die Beteiligten zogen vor das Politbüro, wo
Industrieminister \Nam{Selbmann, Friedrich}{Selbmann} zu
beschwichtigen versuchte und die Normerhöhung zurücknahm. Dies war
jedoch vergeblich und die Protestierenden riefen über Boten und
Rundfunk auch die übrige Bevölkerung zum Streik auf.

So kam es am \dat{17. Juni} zu \dat{flächendeckendem Aufbegehren} in
cat 560 Städten der DDR -- hauptsächlich Industriestandorte wie
\ort{Leuna}, \ort{Wolfen} und \ort{Jena}. Die Demonstranten befreiten
Häftlinge aus den Gefängnissen, es kam zu Zusammenstößen mit der
Polizei. Darauf verhängte die \Ins{SMAD, Sowjetische
Militäradministration}{SMAD} am Mittag in den großen Städten den
Ausnahmezustand. Mithilfe sowjetischer Panzer wurde der Aufstand
blutig niedergeschlagen.

%%%%%%%%%%%%%%%%%%%%%%%%%%%%%%%%%%%%%%%%%%%%%%%%%%%%%%%%%%%%%%%%%%%%%%

\subsection*{Folgen}

\begin{itemize}
\item ca. 21 Tote, zahlreiche Verletzte 
\item Verfolgung und Inhaftierung der \jar{Rädelsführer} und andere
Beteiligter -- ca. 1400 Verurteilungen zu langjährigen
Zuchthausstrafen
\item \jar{Reinigung} der SED von \jar{feindlichen Elementen}
\item Festigung der Machposition \Nam{Ulbricht, Walter}{Walter
Ulbricht}s
\item anhaltende Furcht der Staatsführung vor einer Wiederholung des
17. Juni
\item Preissenkungen der \Ins{HO, Handelsorganisation}{HO}
\item Massenfluchtbewegung in den Westen $\Rightarrow$ Mauerbau (siehe
\ref{sec:mauerbau})
\item Ende der Reparationsentnahmen aus der laufenden Produktion durch
die UdSSR 1954
\end{itemize}

%%%%%%%%%%%%%%%%%%%%%%%%%%%%%%%%%%%%%%%%%%%%%%%%%%%%%%%%%%%%%%%%%%%%%%

\subsection*{Bewertung anhand des modernen Demokratiebegriffs}

\begin{itemize}
\item weitere Freiheitseinschränkungen
\item Beschluss der Niederschlagung allein durch die Exekutive
\item brutale Unterdrückung der Volkssouveränität
\end{itemize}

$\Longrightarrow$ keine rechtsstaatliche Lösung 

\endinput

\section[Bau der Berliner Mauer]{Bau der Berliner
Mauer\mycite{Dtl50erJ}}
\index{Mauerbau}
\index{Berliner Mauer}
\label{sec:mauerbau}

\subsection*{Ausgangssituation}

\dat{Seit 1945 strömten DDR-Bürger} in die BRD. Sie sahen darin einen
Ausweg aus den widrigen Verhältnissen des sozialistischen Staates:
Unzureichender Versorgung, fehlendem Fortschritt und
Zwangskollektivierung sowie totalitaristischem Anspruch der SED bei
äußerst geringem politischem Mitbestimmungsrecht in der DDR standen
wirtschaftlicher Aufschwung und ein hohes Maß an politischer Freiheit
in der BRD gegenüber.

So kam es, dass bis zum folgenden Mauerbau 2,7 Millionen Menschen,
darunter zahlreiche junge und hochqualifizierte Facharbeiter Richtung
Westen zogen. Diese fehlten der DDR beim Aufbau einer leistungsfähigen
Wirtschaft, weshalb \dat{schon 1946 die Grenzen} durch Zäune und
Alarmvorrichtungen \dat{gesichert} wurden. Die Grenze in Berlin war
jedoch weiterhin offen, sodass \dat{1959/60} der DDR der
wirtschaftliche Zusammbruch drohte.\mar{Chruschtschows
Berlin-Ultimatum auch?}

\subsection*{Verfaul}

Am \dat{12. August 1961} unterschreibt \Nam{Ulbricht, Walter}{Walter
Ulbricht} auf geheimen Beschluss und nach Erlaubnis der UdSSR den
Befehl, auch die berliner Grenze zu schließen. Daraufhin rückten in
der Nacht Abteilungen der \Ins{NVA, Nationale Volksarmee}{NVA}, der
\ins{Volkspolizei} und Kampfgruppen aus, riegelten die Sektorengrenzen
ab und sperrten die Verkehrswege. Innerhalb eines Jahres wurde dann
die eigentliche die Westsektoren umschließende Grenzbefestigungsanlage
bestehend aus Mauer, Zäunen, Wachtürmen, Hundelaufstreifen,
Selbstschussanlagen und anderen Sicherungseinrichtungen errichtet.

\subsection*{Folgen}

\begin{itemize}
\item ca. 899 Todesopfer bei Fluchtversuchen aus der DDR
\item Ausreise in den Westen nur noch bei Genehmigung
möglich\mar{Nicht vorher schon?}
\item Ende der Massenflucht
\item weitere Verschlechterung des Verhältnisses zwischen Bevölkerung
und Staat
\item Auseinanderreißung von Familien
\item Verlust von ca. 60\,000 Arbeitsplätzen (Pendler)
\item Verschlechterung der Infrastruktur durch zerschnittenes
Verkehrsnetz
\end{itemize}

\subsection*{Bewertung anhand des modernen Demokratiebegriffs}

In Bezug auf die Verletzung der Grund- und Menschenrechte (siehe
Grundgesetz) stehen sich hier verschiedene Meinungen gegenüber:
Einerseits bedeutete der Mauerbau weitreichende Einschränkung der
Freizügigkeit und persönlichen Freiheit. Andererseits sah
\Nam{Kennedy, John Fitzgerald}{Kennedy} darin die bessere Alternative
gegenüber Krieg\mycite{WikMauer} und der britische Premierminister
\Nam{Macmillan, Maurice Harold}{Harold Macmillan} meinte, dass der
Stopp des Auswanderungsstroms
\textquote[\mycite{MacmillanMauer}]{nothing illegal} sei.

Ungeachtet dessen umging die DDR-Regierung mit dem Beschluss zum
Mauerbau wieder einmal Gewaltenteilung und Volkssouveränität, sodass
jener im krassen Gegensatz zum modernen Demokratiebegriff steht.

\endinput

\section[\ins{Spiegel}-Affäre]{\ins{Spiegel}-Affäre\mycite[4\,f.]{IzpBZeitWand}}
\label{sec:spiegaff}

Am \dat{10. Oktober 1962} veröffentlichte die Wochenzeitschrift
\ins{DER SPIEGEL} den Artikel \ges{Bedingt
abwehrbereit}\mycite{BedEinsber}. Der Verfasser \Nam{Ahlers,
Conrad}{Conrad Ahlers} analysierte darin ein zuvor stattgefundenes
NATO-Manöver und \textquote{kam zu dem Schluß, daß die Verteidigung
der Bundesrepublik im Falle eines Angriffs des \ges{Warschauer
Pakt}{Warschauer Pakts} keinesweigs gesichert sie und daß das [von
Bundesverteidigungsminister \Nam{Strauß, Franz-Josef}{Franz-Josef
Strauß} verfolgte] Konzept des vorbeugenden Schlages den Frieden eher
gefährdete als sicherte}.

Auf der Grundlage eines Gutachtens des
Bundesverteidigungsministeriums, welches aussagte, dass mit jenem
Artikel \textquote{geheimzuhaltende Tatsachen} veröffentlicht worden
waren, ließ die Bundesanwaltschaft ab dem \dat{26. Oktober 1962} auf
Verdacht des Landesverrats die \ins{Spiegel}-Redaktion besetzen und
einige leitende Mitarbeiter verhaften. Bundesjusitzminister
\Nam{Stammberger, Wolfgang}{Wolfgang Stammberger} und Hamburger
Innensenator \Nam{Schmidt, Helmut}{Helmut Schmidt} wurden vorher nicht
informiert.  Außerdem ließ Strauß indem er das Auswärtige Amt umging
Ahlers im Urlaub in Spanien über den Militärattaché der deutschen
Botschaft verhaften.

\subsection*{Reaktionen}

\begin{itemize}
\item Unterstützung des \ins{Spiegel}s bei der Erstellung des
nächstens Hefts durch die Redaktionen anderer Zeitungen
\item Proteste von Intellektuellen und Gewerkschaften gegen den
angeblichen Angriff auf Presse- und Meinungsfreiheit
\item Regierungskrise: Rücktrittsforderungen der FDP an Strauß und
Abzug der FDP-Minister aus der Bundesregierung
\end{itemize}


\subsection*{Folgen}

\begin{itemize}
\item Ende der Besetzung der \ins{Spiegel}-Redaktion am \dat{26.
November}
\item Entlassung der \ins{Spiegel}-Mitarbeiter aus der
Untersuchungshaft
\item Amtsverzichts Strauß' am \dat{30. November}
\item Rücktrittsankündigung \Nam{Adenauer, Konrad}{Konrad Adenauer}s
für Herbst 1963
\end{itemize}


\subsection*{Bewertung anhand des modernen Demokratiebegriffs}

\begin{itemize}
\item Umgehung der Gewaltenteilung
\item Angriff auf Presse- und Meinungsfreiheit
\item Behauptung der Pressefreiheit
\item erstmalige öffentliche politische Stellungnahme der Bevölkerung
nach dem Krieg -- Sieg der Öffentlichkeit
\end{itemize}

$\Longrightarrow$ grenzwertig in der Auslösung, demokratisch in der
Lösung

\section{Wirtschafts- und Sozialpolitik der BRD 1950\,--1979}

Tabelle \ref{tab:entw-brd} auf Seite \pageref{tab:entw-brd} bietet
einen Überblick über die ökonomische Entwicklung und sozialpolitische
Maßnahmen in der BRD.

\mar{Kohle und Stahl weniger oder mehr beim Strukturwandel in der
Tabelle?}

\begin{table}
\caption{Entwicklung der Bundesrepublik von 1950 bis 1979}
\label{tab:entw-brd}

\footnotesize
\newlength{\szheight}
\settoheight{\szheight}{ß}
\newlength{\wirtschw}
\settowidth{\wirtschw}{\emph{Wirtschaftswunder}}
\begin{tabularx}{\textwidth}{p{\wirtschw}XX}
\toprule
& 
ökonomische Entwicklung
&
sozialpolitische Entwicklung \\
\midrule

1950er Jahre \newline
\emph{Wirtschaftswunder}\index{Wirtschaftswunder}
&
\vspace{-\szheight}
\begin{tablist}
\item außergewöhnliche Zuwachsrate bei Produktion und Produktivität
\item Einführung der sozialen Marktwirtschaft durch \Nam{Erhard,
Ludwig}{Ludwig Erhard}
\item Integration vvon Flüchtlingen und Vertrreibenen
\item nahezu Vollbeschäftigung -- Einsatz von Gastarbeitern
\index{Gastarbeiter}
\end{tablist}
&
\vspace{-3.38ex}
\begin{chronik}
\item[1949] \ges{Tarifvertragsgesetz}
\item[1950] \ges{Bundesversorgungsgesetz für Kriegsopfer und
Hinterbliebene}
\item[1952] \ges{Betriebsverfassungsgesetz}
\item[1957] \ges{Kindergeldgesetz} zum Ausgleich finanzieller
Belastungen von Familien
\item[1957] Rentenreform und Alterssicherung für Landwirte 
\end{chronik}
\\

1960\,--1973 \\ Normalisierung &
\vspace{-5.85ex}
\begin{tablist}
\item Verlangsamung des Wirtschaftswachstums
\item kräftige Nachfrage, hohe Produktionszuwächse, Export,
Lohnexpansion, geringe Inflation, niedrige Arbeitslosigkeit bei
schrumpfendem Arbeitskräfteangebot 
\end{tablist}
&
\vspace{-7.65ex}
\begin{chronik}
\item[1961] \ges{Bundessozialhilfegesetz}
\item[1963] \ges{Un"-fall"-ver"-si"-che"-rungs"-neu"-re"-ge"-lungs"-ge"-se"-tz}
\item[1968] \ges{Finanzänderungsgesetz}
\item[1969] \ges{Arbeitsförderungsgesetz}
\end{chronik}
\\
 
1970er Jahre \newline Krisenjahre &
\vspace{-\szheight}
\begin{tablist}
\item wirtschaftssteuernde Maßnahmen des Staates
\item drastische Ölpreiserhöhung der Ölförderungsstaaten 1973
$\Rightarrow$ Energiekrise
\item geringes bis negatives Wirtschaftswachstum
\item Stagnation der Produktion, Strukturwandel (Kohle, Stahl)
\item Inflation, Anstieg der Arbeitslosigkeit ab 1975
\item strukturelles Defizit der öffentlichen Haushalte trotz hohem
Konsumniveau (Massenwohlstand $\blacktriangleright\blacktriangleleft$
öffentliche Verschuldung)
\end{tablist}
&
\vspace{-\szheight}
\begin{tablist}
\item soziale Einschnitte und Kürzungen 
\item dennoch Familien- und Frauenpolitik
\item Rentenreform 1972
\item Kindergeld für alle Kinder 1975
\end{tablist}
\\
\bottomrule
\end{tabularx} 
\end{table}

%%%%%%%%%%%%%%%%%%%%%%%%%%%%%%%%%%%%%%%%%%%%%%%%%%%%%%%%%%%%%%%%%%%%%%

\subsection*{Ursachen für den wirtschaftlichen Aufstieg
der BRD in den Fünfzigern}

\begin{itemize}
\item nahezu vollständig erhaltene Produktionskapazität nach dem
Zweiten Weltkrieg
\item Neuinvestitionen durch \ges{Marshall-Plan}
\item Erhöhung des Konsumangebotes reizt die Produktion an\mar{Hä?}
\item großes Angebot an gut ausgebildeten Arbeitskräften
\item großer Güterbedarf nach dem Zweiten Weltkrieg
\item konsequente Umsetzung der \Beg{Soziale Marktwirtschaft}{Sozialen
Marktwirtschaft}
\item ab 1958 Integration der westdeutschen Märkte\mar{In die
europäische Wirtschaftsordnung, oder was?}
\end{itemize}


\subsection*[Wirtschafts- und Europapolitik]{Wirtschafts- und
Europapolitik\mycite[345\,f., 362\,f.]{WeltgeschNeuz}}
\mar{Rapacki}

\begin{chronik}
\item[9. Mai 1950] \ges{Schuman-Plan}\Nam{Schuman, Robert}{}: Vorschlag für einen gemeinsamen deutsch-fran"-zö"-si"-sche 
Kohle- und Stahlproduktion

\item[18. April 1951] Gründung der \Ins{EGKS, Europäische Gemeinschaft
für Kohle und Stahl}{Europäischen Gemeinschaft für Kohle und Stahl}
(auch \ins{Montanunion} -- zwischen Belgien, Frankreich, der
Bundesrepublik Deutschland, Italien, Luxemburg und den Niederlanden)
und Eingliederung derer in den Welthandel

\item[25. März 1957] Unterzeichnung der \Ges{Römische
Verträge}{Römischen Verträge} durch die
Mitgliedsstaaten der EGKS
\begin{itemize}
\item Gründung der \Ins{EWG, Europäische
Wirtschaftsgemeinschaft}{Europäischen Wirtschaftsgemeinschaft} --
Herstellung eines einheitlichen Wirtschaftsraums mit freiem
Warenverkehr binnen zwölf Jahren
\item Gründung der \Ins{EURATOM, Europäische
Atomgemeinschaft}{Europäischen Atomgemeinschaft} -- Bündelung der
Kräfte zur Forschung und friedlichen Nutzung von Kernenergie
\end{itemize}
\end{chronik}


\subsection*{Bewertung anhand des modernen Demokratiebegriffs}

Die Wirtschafts- und Sozialpolitik der BRD in den Jahren 1950 bis 1979
genügt dem modernen Demokratiebegriff: Trotz Krisen blieben die
Ausgaben für soziale Maßnahmen konstant hoch wie es das Grundgesetz
fordert. Gleichzeitig wahrt das System der Sozialen Marktwirtschaft die
wirtschaftliche Freiheit (Vertrags-, Konsumfreiheit, freie
Preisgestaltung, Privateigentum) und Antimonopolpolitik sorgt für
fairen Wettbewerb.

\endinput

\section{Wirtschafts- und Sozialpolitik der DDR 1970\,--1980}

Am \dat{3. Mai 1971} wurde \Nam{Ulbricht, Walter}{Walter Ulbricht}
durch \Nam{Honecker, Erich}{Erich Honecker} als Erster Sekretär des
Zentralkommitees der SED abgelöst. Damit wurden jegliche Bestrebungen
nach einer Wiedervereinigung Deutschlands beendet, die verbliebenen
privaten Betriebe verstaatlicht und eine neue politische Zielsetzung
eingeführt: War Ulbrichts Devise noch \enquote{Wie wir heute arbeiten,
werden wir morgen leben.}, orientierte Honecker nun auf eine
\jar{Einheit von Wirtschafts- und Sozialpolitik}.

Das bedeutete Erhöhung des Lebensstandards der Bevölkerung und
Ausweitung von Sozialleistungen zur Bewältigung gesellschaftlicher
Probleme (Versorgungsengpässe, Geburtenrückgang) bei gleichzeitigem
hohem Wirtschaftswachstum. Die Ausgaben sollten durch Kredite
finanziert werden, die sich mit der Zeit zu einem immer größeren
Schuldenproblem entwickeln sollten.\footnote{Der Leiter der
\Ins{Staatliche Planungskommission}{Staatlichen Planungskommission}
\Nam{Schürer, Gerhard Paul}{Gerhard Schürer} sah diese Entwicklung
voraus und soll Honeckers Maßnahmen mit einer Umdrehung der Devise
Ulbrichts kommentiert haben: \enquote{Wie wir heute leben, werden wir
morgen arbeiten.}}

\subsection*{Maßnahmen}

\begin{itemize}
\item Wohnungsbauprogramm -- Neubaugebiete mit relativ komfortablen
Wohnungen in den Städten
\item Kauf westlicher Produktionsanlagen
\item umfangreiche Subventionen von Lebensmitteln, Mieten,
öffentlichen Verkehrsmitteln und anderem
\item Familienförderung durch Geburtenprämien, zinslose Kredite bei
Familiengründungen und anderes
\end{itemize}

\subsection*{Ergebnisse}

\begin{itemize}
\item Verbesserung des Lebensstandards -- höchster in den RGW-Staaten
\item Einkommensanstieg, \jar{zweite Lohntüte}, Verbesserung der
Versorgung mit Konsumgütern
\item Anstieg der Geburtenrate
\item starker Anstieg der Staatsverschuldung\mar{Was soll der übrige
zusammengeklaubte Quatsch auf dem Arbeitsblatt, bei dem zudem die
Bewertung am moderenen Demokratiebegriff fehlt?}
\end{itemize}

\endinput

\section[Außerparlamentarische Bewegungen]{Außerparlamentarische
Bewegungen\mycite{WikStud}
\mycite[522\,f.]{gelbesGeschichts}
\mycite{WeltgeschNeuz}
\mycite{SpiegChronik}
\mycite{GeschPolGes}
\mycite[14\,--\,20]{IzpBZeitWand}
}
\label{sec:ap-bew}

\begin{aufgabe}
Analysieren Sie außerparlamentarische Bewegungen am Beispiel der 68er
in der BRD und bewerten Sie sie hinsichtlich des modernen
Demokratiebegriffes! 
\end{aufgabe}

\subsection*{Kritikpunkte}

\begin{itemize}
\item Vietnamkrieg
\item frühere NSDAP-Mitgliedschaft Bundeskanzler Kiesingers
\item fehlende Aufarbeitung der NS-Vergangenheit in der neuen Wohlstandsgesellschaft
\item Notstandsgesetzgebung (Bedingung für vollständige Souveränität der BRD)
\item mangelhafte Studienbedingungen
\item Schwäche der innerparlamentarischen Opposition
\item überkommene Autoritäten
\item Konsum- und Wohlstandsdenken
\end{itemize}

\subsection*{Entwicklung}

\begin{chronik}
\item[1.\,12.\,1966] Bildung der Großen Koalition (CDU/CSU, SPD),
Schwäche der FDP
$\Rightarrow$ Aufruf Rudi Dutschkes (Studentenführer) zur Bildung einer
Außerparlamentarischen Opposition
\item[2.\,6.\,1967] Benno Ohnesorg bei Studentendemonstration
erschossen
\item[Nov. 1967] Proteste gegen veraltete Hochschulstrukturen
\item[4.\,4.\,1968] Martin Luther King ermordet
\item[11.\,4.\,1968] Attentat auf Rudi Dutschke $\Rightarrow$
Osterunruhen
\item[Mai 1968] Proteste gegen Notstandsgesetzgebung
\item[März 1969] Bundesregierung erachtet den Verband Deutscher
Studentenschaften als \enquote{revolutionären Kampfverband},
Einstellung der Zuschüsse
\end{chronik}

Die Bewegung verlief sich Anfang der Siebziger allmählich im Sande, da
man das Ziel, die Notverordnungsgesetze zu verhindern, nicht erreicht
beziehungsweise verfehlt hatte und außerdem nach Gewaltfreiheit
strebte.

\subsection*{Folgen}

\begin{itemize}
\item keine direkte Änderung des pol. Systems
\item politische Bewusstseinsbildung und Emanzipation
\item Steigerung der Partizipationsmöglichkeiten für Bürger (\jar{in
den Köpfen})
\item Veränderung des Lebensgefühls (Antiautorität (Zurückdrängung
autoritärer Denkmuster), sexuelle
Liberalisierung (\ins{Antibabypille}, Anfänge der Frauenbewegung) etc.)
\item Fortsetzung in RAF-Terror
\item Zuwendung der Mehrheit zur SPD -- sozialliberale Koalition ab
\dat{1969}
\end{itemize}

\noindent $\Longrightarrow$ politische und gesellschaftliche
Modernisierung

\subsection*{Bewertung}

\begin{itemize}
\item politische Beteiligung Jüngerer
\item Demonstration als demokratisches Grundrecht
\item Kritik an mangelhafter Umsetzung der demokratischen Grundrechte
-- Erfolg: Änderung der Notstandsgesetzesentwürfe unter Einführung
eines Widerstandsrechts für die Bürger und Verbot der Anwendung beim
Arbeitskampf\mar{?}
\item Eskalation führt zu Menschenrechtsverletzungen, also
undemokratisch -- Beteiligung beider Seiten
\item Kritik an der Großen Koalition: Angst der Bevölkerung vor
diktatorischem Missbrauch
\end{itemize}

\noindent $\Longrightarrow$ Förderung beziehungsweise Stärkung der
Demokratie



\endinput

\section[Rote Armee Fraktion]{Rote Armee Fraktion\mycite{Dtl70er80er}}
\label{sec:raf}

Die \Ins{RAF, Rote Armee Fraktion}{Rote Armee Fraktion} (RAF) entstand
\dat{1970} aus einigen Aktivisten und Splittern der Studentenbewegung
(siehe \ref{sec:ap-bew}). Sie war eine linksgerichtete
Terrororganisation, die über Verbindungen in der ganzen Welt verfügte
und unter anderem Attentate und Sprengstoffanschläge verübte.

Die ideologischen Grundlagen der Bewegung waren vor allem
kommunistischer Art. Sie sahen Terror und Gewalt als Vorbereitung zur
Revolution, die eine Herrschaft der Arbeiterklasse errichten
sollte.\mar{Belege?}

\subsection*{Kritikpunkte}

\begin{itemize}
\item Wiederaufrüstung
\item Verbot der KPD 1956\mar{Nicht ein bisschen lange her?}
\item Notstandsgesetze
\item Vietnamkrieg
\item Präsenz der USA in der BRD
\item \jar{bundesdeutscher Imperialismus} -- Verhalten der
Oberen\mar{?}
\end{itemize}


\subsection*{Aktionen}

\begin{chronik}
\item[11. Mai 1972] Bombenanschlag auf das Hauptquartier der US-Armee
in Frankfurt
\item[24. April 1975] Stürmung der deutschen Botschaft in Stockholm --
zwei erschossene Diplomaten
\item[7. April] Ermordung des Generalbundesanwalt \Nam{Buback,
Siegfried}{Siegfried Buback}
\item[30. Juli 1977] Ermordung des Vorstandssprechers der
\ins{Dresdner Bank AG} \Nam{Ponto, Jürgen}{Jürgen Ponto}
\item[5. September 1977] Entführung des Präsidenten der
\ins{Bundesvereinigung der Deutschen Arbeitgebervereine} und des
\ins{Bundesverbandes der Deutschen Industrie} \Nam{Schleyer, Hanns
Martin}{Hanns Martin Schleyer} -- später Ermordung
\item[Oktober 1977] Entführung der Lufthansamaschine \Ins{Landshut
(Flugzeug)}{Landshut} durch palästinensische Verbündete
\item[25. Juni 1979] Anschlag auf den NATO-Oberbefehlshaber in Europa
\Nam{Haig, Alexander}{Alexander Haig}
\end{chronik}


\subsection*{Maßnahmen des Staates}

\renewcommand*{\dictumwidth}{0.8\textwidth}
\dictum[\Nam{Schmidt, Helmut}{Helmut Schmidt} am 15. Januar 1979
in einem Interview mit dem \ins{Spiegel}\mycite{BpbAusnZust}]{Ich kann
nur nachträglich den deutschen Juristen danken, daß sie das alles
nicht verfassungsrechtlich untersucht haben.}

\begin{itemize}
\item Erlass von Gesetzen zur Einschränkung von Rechten der
Verteidigung, beispielsweise Möglichkeit der Strafprozessführung in
Abwesenheit des Angeklagten, sowie die Verbrechensverfolgung durch
Staats- und Bundesanwaltschaft erleichterten

\item Kontaktverbot der Gefangenen untereinander und zur Außenwalt
während der Zeit der Entführung Schleyers

\item freiwillige Nachrichtensperre für die Medien
-- Vorwurf der Instrumentalisierung der Medien\mycite{BpbAusnZust}

\item \jar{Lauschangriffe} auf verdächtige Anwälte und Gespräche der
Verteidiger mit den Inhaftierten\mycite{BpbAusnZust}

\item Einführung der \ins{Trennscheibe} zur Trennung von Verteidiger
und Angeklagten nachdem aufgedeckt wurde, dass verschiedene
Gegenstände eingeschmuggelt worden waren

\item teilweise stark fragwürdige Methoden des zuständigen
Justizapparates als Reaktion auf die permanenten Provokationen durch
Angeklagte und Verteidiger\mycite{BpbStammProz}

\item Entlassung von Gefangenen im Austausch gegen Geiseln der RAF
\end{itemize}


\subsection*{Bewertung anhand des modernen Demokratiebegriffs}

Die außergewöhnliche Situation, die die Terroristen der RAF mit
Absicht herbeigeführten hatten, stellte den Rechtsstaat Bundesrepublik
Deutschland auf eine äußerst harte Probe. Vor diesem Hintergrund
müssen die Maßnahmen gesehen werden, die der Staat ergriff und die
die Grenzen eben jenes Rechtsstaates überschritten, die Verfassung
also verletzten (vgl. das obige Zitat Schmidts).

Der Umgang mit dem RAF-Terror und den RAF-Terroristen genügt also
dem modernen Demokratiebegriff nicht, sind jedoch im Kontext des
geringen Handlungsspielraumes zu sehen, der den Verantwortlichen zur
Verfügung stand.

\endinput

- Entlarvung des Rechtsstaats
- Grenzen des Rechtsstaats kaum durchbrochen




\chapter{Beurteilen von historischen Prozessen}
\mar{Prozesse?}

\section{Das Problem der Vergleichbarkeit der beiden Diktaturen in
Deutschland}

\dictum[Margherita von Brentano]{Der
bloße Vergleich des Dritten Reiches mit der DDR ist eine schreckliche
Verharmlosung. Das Dritte Reich hinterließ Berge von Leichen. Die DDR
hinterließ Berge von Karteikarten.}\mar{Das ist der Ausgangspunkt.}


\subsection*{Probleme im Vergleichsansatz}

\begin{itemize}
\item Vergleich beinhaltet Gefahr der Gleichsetzung -- Gefahr der
Verharmlosung des Nationalsozialismus

\item Singularität des Nationalsozialismus

\item Gefahr der Verharmlosung der DDR (menschliche Schicksale hinter
den \enquote{Karteikarten}) 
\end{itemize}


\subsection*{Prämissen}

\begin{enumerate}
\item Vergleich darf nicht zur Gleichsetzung geraten
\item \emph{Versuch} eines Vergleichs anhand geeigneter Kriterien
\item Wertung/Auseinandersetzung der/mit der Vergleichbarkeit der
Diktaturen
\end{enumerate}


\subsection*{Vergleichende Darstellung}

{
% joy with calculation
\newlength{\frstcolwdth}
\settowidth{\frstcolwdth}{Selbstdar-}

\newlength{\dummywdth}
\newlength{\othcolwidth}
\newlength{\fieldwidth}
\setlength{\dummywdth}{\textwidth}
\addtolength{\dummywdth}{-\frstcolwdth}
\addtolength{\dummywdth}{-2\tabcolsep}
\setlength{\othcolwidth}{0.25\dummywdth}
\addtolength{\othcolwidth}{-2\tabcolsep}
\setlength{\fieldwidth}{2\othcolwidth}

\newcolumntype{x}{p{\othcolwidth}}
\newcommand{\dublcol}[1]{\multicolumn{2}{p{\fieldwidth}}{#1}}

\tablefirsthead{%
\toprule
Kriterium & 
\multicolumn{2}{c}{Nationalsozialismus} & \multicolumn{2}{c}{DDR} \\
\midrule
}

\tabletail{\midrule}
\tablehead{\midrule}

\tablelasttail{\bottomrule}

\renewcommand*{\arraystretch}{0.8}

%%%%%%%%%%%%%%%%%%%%%%%%%%%%%%%%%%%%%%%%%%%%%%%%%%%%%%%%%%%%%%%%%%%%%%


\footnotesize
\begin{supertabular*}{\textwidth}{p{\frstcolwdth}xxxx}

\vspace{0em}Herr"-schafts"-struk"-tur &
&
\dublcol{
%%\vspace{-0.74em}
\begin{tablist}
\setlength{\listparindent}{0.2\columnwidth}
\item nur einseitige Partizipationsmöglichkeiten
\item keine Gewaltenteilung
\item Totalitarismus
\item Staatssicherungsorganisationen (\ins{Gestapo}, \Ins{Stasi,
Ministerium für Staatssicherheit}{Stasi})
\item Einheitsorganisationen
\end{tablist}}
&
\\

&
\dublcol{
%\vspace{-0.74em}
\begin{tablist}
\item Gleichschaltung der Länder, Ausschaltung des Parlaments --
Zentralismus
\item alleinige Herrschaft einer Person
\item keine Wahlen seit August 1934
\item nur eine Partei zugelassen
\item keine Verfassung
\end{tablist}
}
&
\dublcol{
%\vspace{-0.74em}
\begin{tablist}
\item weniger ausgeprägter Zentralismus
\item Herrschaft der Partei
\item Wahl nach Einheitsliste
\item unterschiedliche Parteien -- Zusammenfassung im Block 
\end{tablist}
}
\\

%%%%%%%%%%%%%%%%%%%%%%%%%%%%%%%%%%%%%%%%%%%%%%%%%%%%%%%%%%%%%%%%%%%%%%

\vspace{0em}Selbst"-dar"-stel"-lung &
&
\dublcol{
%\vspace{-0.74em}
\begin{tablist}
\item Vereinnahmung der Jugend
\item Vereinnahmung der Kultur 
\item Propaganda
\item Intransparenz
\item Militär
\end{tablist}
}
&
\\

&
\dublcol{
%\vspace{-0.74em}
\begin{tablist}
\item Rechtfertigung der Expansionspolitik, des Judenmords und der
Euthanasie mit der Ideologie
\item Autarkie
\item Führerkult
\item Nationalismus -- Chauvinismus
\end{tablist}
}
&
\dublcol{
%\vspace{-0.74em}
\begin{tablist}
\item Präsentation als \jar{antifaschistischer Friedensstaat} --
Antikapitalismus, Klassenkampf
\item Partei
\end{tablist}
}
\\

%%%%%%%%%%%%%%%%%%%%%%%%%%%%%%%%%%%%%%%%%%%%%%%%%%%%%%%%%%%%%%%%%%%%%%

\vspace{0em}ideologische Grundlagen/In"-sti"-tu"-tio"-nen &
&
\dublcol{
%\vspace{-0.74em}
\begin{tablist}
\item Einheitsorganisationen
\item Militär
\item Staatssicherungorganisationen -- Angst, Terror
\item Gleichschaltung
\end{tablist}
}
&
\\

&
\dublcol{
%\vspace{-0.74em}
\begin{tablist}
\item Machtkampf der Minister und Ministerien
\item \Ins{BDM, Bund Deutscher Mädel}{BDM}, \Ins{HJ, Hitlerjugend}{HJ}
(Militär, Haushalt)
\end{tablist}
}
&
\dublcol{
%\vspace{-0.74em}
\begin{tablist}
\item Antifaschismus 
\item Einbindung in einen Militärblock
\item \ins{Pionierorganisation \enquote{Ernst Thälmann}}, \Ins{FDJ,
Freie Deutsche Jugend}{FDJ} (politische Betätigung)
\end{tablist}
}
\\

%%%%%%%%%%%%%%%%%%%%%%%%%%%%%%%%%%%%%%%%%%%%%%%%%%%%%%%%%%%%%%%%%%%%%%

\vspace{0em}Rolle der Bevölkerung &
&
\dublcol{
%%\vspace{-0.73em}
\begin{tablist}
\item Befürworter und Gegner
\item Regime auf Bevölkerung angewiesen
\item wenig Widerstand
\item erzwungene Haltung
\end{tablist}
}
&
\\

&
\dublcol{
\vspace{0.2em}
Nationalsozialismus aus der Bevölkerung heraus entstanden 
}
&
\dublcol{
\vspace{0.2em}
DDR-System von der UdSSR angelegt und stärker von der Ideologie
durchdrungen.
}
\\

%%%%%%%%%%%%%%%%%%%%%%%%%%%%%%%%%%%%%%%%%%%%%%%%%%%%%%%%%%%%%%%%%%%%%%

\vspace{0.5em}Umgang mit Gegnern &
&
\dublcol{
\vspace{0.50em}
\begin{tablist}
\item Gegner: Andersdenkende (mit einzelnen Dingen nicht
einverstanden), Gegner der Ideologie (mit dem gesamten System nicht
einverstanden)
\item Haft 
\end{tablist}
}
&
\\

&
\dublcol{
%\vspace{-0.44em}
\begin{tablist}
\item Morde, Konzentrationslager
\item SS, Gestapo
\item keine Verhandlungen
\item sehr viele Todesopfer
\item Sippenhaft 
\end{tablist}
}
&
\dublcol{
%\vspace{-0.44em}
\begin{tablist}
\item Mauertote, Tote beim Volksaufstand
\item Ministerium für Staatssicherheit
\item gesellschaftliche Benachteiligung
\end{tablist}
}
\\

\end{supertabular*}
}


\subsection*{Wertung}

\dictum[Alfred Grosser]{Selbst die Feststellung eines völligen
Gegensatzes kann immer nur das Ergebnis eines Vergleichs sein.}

Man kann erkennen, dass zahlreiche Parallelen zwischen den beiden
Diktaturen existieren, die aber doch nur Parallelen sind und keine
Zusammenfallenden. Denn die Ereignisse stehen auf vollkommen
unterschiedlichen Stufen. -- So findet der Völkermord der
Nationalsozialisten seine Entsprechung gerade einmal in den Handlungen
des Ministeriums für Staatssicherheit der DDR und dem Führerprinzip
steht die Parteidiktatur gegenüber. Das nationalsozialistische Regime
ist singulär; Holocaust und Zweiter Weltkrieg finden keinerlei
Entsprechung in der DDR.

Außerdem gäbe es ein Problem, wenn man vorgäbe, keinen Vergleich
vorzunehmen: Man vergliche dennoch, indem man beide Systeme einfach
als Diktaturen einordnete und damit gleichsetzte. Ein Vergleich ist
also notwendig. Gemeinsamkeiten und Unterschiede müssen allerdings
klar und sorgfältig herausgearbeitet werden.

\section{Anspruch und Wirklichkeit einer parlamentarischen Demokratie}

\begin{aufgabe}
Überprüfen Sie, iweiweit die Mittel der parlamentarischen Demokratie
gegen die \Ins{RAF, Rote Armee Fraktion}{RAF} im Sinner der Verfassung
und der Demokratie waren!
\end{aufgabe}

Wie schon im Abschnitt über die RAF (siehe \ref{sec:raf}) besprochen,
waren die ergriffenen Maßnahmen teilweise undemokratisch. Auch die
\ins{Spiegel}-Affäre (siehe \ref{sec:spiegaff}) war es. Aber dies
sind einzelne Vorkommnisse und die Verantwortlichen waren vorher
demokratisch gewählt worden.

In der parlamentarischen Demokratie der BRD lagen also Anspruch und
Wirklichkeit eng beeinander. -- Wenn man von wenigen Ereignissen
absieht, ist das System als demokratisch zu bezeichnen.


\endinput
