\section{Parteien}
\mar{Siehe hierzu auch das Arbeitsblatt aus der Sekundarstufe \Rm{1}}

\subsection*{Republiksgründung}

Tabelle \ref{tab:vergl-part-weimrep} bietet eine Übersicht über die
Parteien, die bei der Gründung der Weimarer Republik bestimmend waren.
Dabei ist noch zu bemerken, dass nur SPD, DDP und Zentrum
staatstragende Parteien waren, alle Parteien eine Revision des
\ges{Vertrag von Versailles}{Vertrages von Versailles} anstrebten und
dass die DNVP den Wiedererwerb von Kolonien und die Erneuerung des
deutschen Kaisertums befürwortete.

%{
%\small
%\renewcommand{\arraystretch}{0.8}
%
%% Wir müssen doch mal wieder berechnen.
%\newlength{\frstcolwdth}
%\settowidth{\frstcolwdth}{\textit{Zentrum}}
%
%\newlength{\dumwdth}
%\setlength{\dumwdth}{\textwidth}
%\addtolength{\dumwdth}{-\frstcolwdth}
%\newlength{\othcolwdth}
%\setlength{\othcolwdth}{0.5\dumwdth}
%\addtolength{\othcolwdth}{-3\tabcolsep}
%
%\tablecaption{bla}
%\tablefirsthead{
%\toprule
%
%Partei &
%Politisches System &
%Wirtschaftliches System \\
%
%\midrule
%}
%
%\begin{supertabular*}{\textwidth}{cp{\othcolwdth}p{\othcolwdth}}
\begin{table}
\caption{Vergleich von Parteien der Weimarer Republik}
\label{tab:vergl-part-weimrep}

\renewcommand{\arraystretch}{0.8}
\footnotesize

\begin{tabularx}{\textwidth}{cXX}
\toprule

Partei &
Politisches System &
Wirtschaftliches System \\

\midrule

\Ins{DNVP, Deutschnationale Volkspartei}{DNVP} &
\vspace{-0.7em}
\begin{tablist}
\item überparteiliche (Hohenzollern-) Monarchie 
\item starke Exekutive
\item Mitwirkung des Parlaments bei der Gesetzgebung
\item Ständevertretung zusätzlich zur Volksvertretung
\end{tablist}
&
\vspace{-0.7em}
\begin{tablist}
\item Eigenwirtschaft und Privateigentum
\item Sozialisierung mit äußerster Vorsicht
\item Förderung eines starken Mittelstandes und der Landwirtschaft 
\end{tablist}
\\

\Ins{DVP, Deutsche Volkspartei}{DVP} &
\vspace{-0.7em}
\begin{tablist}
\item Monarchie
\item Verantwortliche Mitwirkung des Parlaments an der Regierung 
\end{tablist}
&
\vspace{-0.7em}
\begin{tablist}
\item Privatwirtschaft
\item keine Sozialisierung
\item Förderung der Landwirtschaft und des Mittelstandes 
\end{tablist}
\\

\Ins{DDP, Deutsche Demokratische Partei}{DDP} &
\vspace{-0.7em}
\begin{tablist}
\item demokratische Republik auf dem Boden der Weimarer Verfassung
\item liberaler und sozialer Rechtsstaat 
\end{tablist}
&
\vspace{-0.7em}
\begin{tablist}
\item Privatwirtschaft
\item gegen jegliche Vergesellschaftung
\item Schutz von Handwerk und Mittelstand 
\end{tablist}
\\

\Ins{Zentrum, Deutsche Zentrumspartei}{Zentrum} &
\vspace{-0.7em}
\begin{tablist}
\item demokratische Republik 
\item christliche Grundsätze
\item bürgerliche Freiheiten
\item soziale Gerechtigkeit
\end{tablist}
&
\vspace{-0.7em}
\begin{tablist}
\item Recht auf Privateigentum
\item Förderung des Genossenschaftswesens
\item Verstaatlichung nur gegen Entschädigung
\item Schutz des Mittelstands und des Bauernstands
\end{tablist}
\\

\Ins{SPD, Sozialdemokratische Partei Deutschlands!Weimarer
Republik}{SPD} &
\vspace{-0.7em}
\begin{tablist}
\item demokratische Republik
\item Demokratisierung des Staates und der Gesellschaft
\item Erneuerung der Gesellschaft
\item sozialistisches Gemeinwesen
\end{tablist}
&
\vspace{-0.77em}
\begin{tablist}
\item Überwindung des kapitalistischen Systems
\item Förderung des gemeinwirtschaftlichen Gedankens
\item Genossenschaftswesen
\item Überführung der Konzerne in die GemeInschaft
\item Verstaatlichung von Grund und Boden
\end{tablist} 
\\

\Ins{KPD, Kommunistische Partei Deutschlands!Weimarer Republik}{KPD} &
\vspace{-0.7em}
\begin{tablist}
\item Errichtung einer sozialistischen Gesellschaftsordnung
\item revolutionäre Umgestaltung von Staat, Gesellschaft und
Wirtschaft
\item Übernahme der staatlichen Funktionen durch Arbeiterräte 
(Räterepublik)
\end{tablist}
&
\vspace{-0.7em}
\begin{tablist}
\item Enteignung von Banken, Industrie --
Vergesellschaftung der Industrie und des Kapitals
\item Enteignung von Großgrundbesitz -- Bildung sozialistischer
landwirtschaftlicher Genossenschaften
\end{tablist}
\\

\bottomrule
\end{tabularx}
\end{table}
%\end{supertabular*}
%}

\subsection*{Rechtsextremistische Strömung}
\index{Antimarxismus}
\index{Antiliberalismus}
\index{Antisemitismus}
\index{Jugendkult}
\index{Führerkult}
\index{Nationalsozialismus!Weimarer Republik}

Zu Anfang der Weimarer Republik noch von geringer Bedeutung, nahm die
Bedeutung der rechtsextremistische Strömung, politisch vor allem durch
die \ins{NSDAP, Nationalsozialistische Deutsche Arbeiterpartei}{NSDAP}
verkörpert, rasch zu. Sie war charakterisiert durch ideologische
Prägung mit \emph{Antimarxismus}, \emph{Antiliberalismus} und
\emph{Antisemitismus}, Betonung von Werten wie Einigkeit und Treue,
Militarismus als Lebenform und völkisch mystifizierten Jugend- und
Führerkult.

Mit den Konservativen gemein hatten sie die starke Ablehnung der
Weimarer Republik (\jar{Republik der Novemberverbrecher}) und die
Forderung nach einem starken Staat. Sie zeichneten sich aber durch
größere Radikalität, die auch vor Gewalttaten bis hin zu Morden nicht
zurückschreckte, Bildung einer \jar{kriegerischen Elite} sowie das
Ziel eines \jar{nationalen Sozialismus} aus.

\endinput
