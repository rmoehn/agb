\section{Lösungsansätze für die soziale Frage}
\label{sec:soz-frag-loes}
\index{soziale Frage!Lösungsansätze}

\begin{aufgabe}
Stellen Sie einen Maßnahmenkomplex zur Abmilderung/Lösung der Sozialen
Frage vor und prüfen Sie seine Wirksamkeit anhand Hegels Auffassungen! 
\end{aufgabe}

\subsection{Der Marxismus -- Veränderung durch Umbruch}
\label{ssc:marxismus}
\index{Marxismus}
\index{Materialismus}
\mar{\cite{DudPolGes} bietet hier eine gute Darstellung.}

\begin{aufgabe}
Erarbeiten Sie aus der Quelle Marx' Sicht auf die Rolle der
Bourgeoisie in der Geschichte!

Untersuchen Sie die Richtigkeit der fettgedruckten Position anhand des
kapitalistischen Produktionsprozesses in Marx' sogenannter \jar{Basis}! 

Überprüfen sie anhand des Hegelschen Systems der Sozialen Frage,
inwiefern die Ideen Marx' eine Lösung dieser darstellen!
\end{aufgabe}

Im Hefter ist dies in vorerst ausreichender Form dargestellt.

%%%%%%%%%%%%%%%%%%%%%%%%%%%%%%%%%%%%%%%%%%%%%%%%%%%%%%%%%%%%%%%%%%%%%%

\subsection{Die Arbeiterschaft -- Hilfe zur Selbsthilfe}
\label{ssc:arb-bew}
\label{ssc:soz-frag-loes-arb}
\index{Arbeiterbewegung}

\begin{aufgabe}
Untersuchen Sie, inwiefern die Gewerkschaften zur Lösung der sozialen
Frage beitrugen!
\end{aufgabe}

\subsubsection{Ausgangspunkt}

Der Vorteil der Arbeiter war, daß sie eine äußerst breite Schicht der
Bevölkerung bildeten. Unter der Voraussetzung, daß sie sich
zusammenschließen und als die Masse handeln, die sie waren, kann man
ihnen große Chancen, ihre Ziele und Interessen durchzusetzen,
einräumen.

Genau diese Voraussetzung ist aber das Problem, denn die Arbeiter
waren keine homogene Masse. Vielmehr gab es auch hier eine soziale
Schichtung, die sich in der Gehaltsstruktur ausdrückte und zum
ständigen Konflikt beispielsweise zwischen Meistern und
\glq{}Angelernten\grq{} führte. Das ständig bereitstehende
\emph{Ersatzheer} führte zu großer Konkurrenz untereinander. Dieses
Konfliktpotential wurde noch durch die Fabrikordnungen vergrößert,
denn diese sahen Kollektivstrafen vor.

Zur unterschiedlichen sozialen kam noch die unterschiedliche regionale
Herkunft. Da es der Arbeiterschaft an organisatorischem Wissen fehlte,
war es auch hier schwer, eine breite Basis für die Durchsetzung der
Ziele zu finden.

Die Lebensumstände der Arbeiter waren weiterhin schon so miserabel,
dass diese entweder gar keine Zeit fanden, sich um andere Probleme als
ihre eigenen, zu kümmern oder die ständigen Sorgen von vornherein in den
Alkohol oder den Sport flohen.\\


\subsubsection{Entwicklung der Arbeiterbewegung in Großbritannien}
\index{Arbeiterbewegung!Großbritannien}

\begin{chronik}
\item[nach 1814/15] zunehmende Politisierung des Aufbegehrens der
Arbeiter, Streiks

\item[1824] Aufhebung des Koalitionsverbots, Bildung von
Gewerkschaften und Gewerkschaftsverbänden in der Folge

\item[1840] Aus der chartistischen\footnote{Der Name \emph{Chartismus}
leitet sich aus dem Gesetzentwurf ab, der dieser Strömung entstammte
und Forderungen nach beschränkungslosen (Zensus etc.) jährlichen
geheimen allgemeinen Wahlen, Diäten für Abgeordnete und anderem
umsetzen sollte.} Strömung entstand die \ins{National
Chartist} als erste allerdings illegale Arbeiterpartei unserer Zeit.

\item[1860] Nachdem die politische Richtung der Arbeiterbewegung ins
Stocken gekommen war, bildet sich aus den entstehenden \emph{Trade
Unions} (Gewerkschaften) der \ins{Trade Union Council}.

\item[1864] Gründung der \ins{Internationalen
Arbeiterassoziation}\footnote{\Nam{Marx, Karl}{Karl Marx} war
eines der führenden Mitglieder.} (IAA) als Zusammenschluss aller
Arbeiterorganisationen

\item[1876] Auflösung der IAA nach Erfüllung ihrer Aufgabe,
Fortsetzung der Arbeit in Parteien
\end{chronik}


\subsubsection{Arbeiterparteien in Deutschland}
\index{Arbeiterparteien!Deutschland}
\mar{Zur Entwicklung siehe vorerst das Organigramm im Hefter.}

Ziele/Forderungen:

\begin{itemize}
\item Brechung des \Beg{Ehernes Lohngesetz}{Ehernen
Lohngesetzes}\footnote{Da \cite{WiLexEhLohnGes} eine andere Definition
bringt, als ich im Glossar niedergeschriebenen habe, ist fraglich, ob
der hier abgedruckte Sachverhalt stimmt.}
\item allgemeine, gleiche, direkte Wahl (Druckmittel gegen
Unternehmer)
\item Befreiung der Arbeiterklasse durch die Arbeiterklasse
\item Besetitigung der Abhängigkeit des Lohnarbeiters
\item politische Freiheiten (Voraussetzung für ökonomische
Freiheiten)
\item marxistische Strömungen: Revolution \index{Marxismus}
\item politisch-praktische Strömungen: Reformen
\end{itemize}

Mittel:

\begin{itemize}
\item Parteibildung und -arbeit
\item Wahlrechtskämpfe
\item Zusammenschluss von Arbeiterorganisationen
\item Gründung von Arbeiterproduktivgenossenschaften (Vorschlag
\Nam{Lassalle, Ferdinand}{Lassalle}s) -- Arbeiter selbst als
Unternehmer
\item Programme, Zeitschriften
\end{itemize}


\subsubsection{Gewerkschaften in Deutschland\mycite{MustaGeGe}}
\index{Gewerkschaften!Deutschland}

Die deutsche Gewerkschaftsbewegung wies einige Unterschiede zu der in
Großbritannien auf. So wurde in Deutschland durch späte Einführung des
Koalitionsrechts und Sozialistengesetz ein erheblicher Druck ausgeübt.
Dies führte dazu, dass die Vereinigungen im Untergrund
weiterexistieren mussten. Dadurch zu kluger Planung und Organisation
gezwungen, erfuhren die Gewerkschaften eine Stärkung.

Weiterhin waren die deutschen Gewerkschaften ideologisch breiter
aufgestellt. Das bedeutete natürlich, dass alle ihre
Interessenvertretung fanden, hatte aber auch den Nachteil, dass die
Möglichkeiten, als Masse zu handeln, eingeschränkt waren.

Letztendlich legte man in Deutschland sehr großen Wert auf die
organisatorische Trennung zwischen Gewerkschafts- und Parteiarbeit.
-- Die Gewerkschaften übernahmen die überparteiliche soziale
Unterstützung und Basisarbeit während politische und parlamentarische
Arbeit den Parteien zufiel. Man suchte so, die Sorge für die Lage der
Arbeiter unabhängig von politischen Meinungsverschiedenheiten zu
machen, rief die Arbeiter aber gleichzeitig auf, durch Parteieintritt
den Einfluss des Proletariats auf das Staatsgeschehen zu
vergrößern.\mycite{ResErfGeKo} \mycite{ResKonfGeVoGo}\\

\noindent Entwicklung:

\begin{chronik}
\item[nach 1848/49] lokale Arbeiterkomitees und -zusammenschlüsse

\item[1868] Gründung des \Ins{Allgemeiner Deutscher
Arbeiterschaftsverband}{Allgemeinen Deutschen
Arbeiterschaftsverbandes}

\item[1868] Gründung der \Ins{}{\Nam{Hirsch, Max}{Hirsch}-\Nam{Duncker,
Franz}{Duncker}schen-Gewerksverbände}\footnote{\cite{LEMOHiDu} datiert
hier auf 1869.}

\item[1869] Gründung der \Ins{Internationale
Gewerksgenossenschaften}{Internationalen Gewerksgenossenschaften}
durch \Nam{Bebel, August}{August Bebel} und \Nam{Liebknecht,
Wilhelm}{Wilhelm Liebknecht}

\item[1869] Koalitionsfreiheit für Preußen

\item[1878] \ges{Gesetz gegen die gemeingefährlichen Bestrebungen der
Sozialdemokratie} -- Weiterexistenz im Untergrund

\item[1886] Streikerlass in Preußen -- Verfolgung illegaler
Gewerkschaften

\item[1890] Aufhebung des Sozialistengesetzes

\item[1890] Gründung der \ins{Generalkommission der Freien
Gewerkschaften Deutschlands} als erster Dachorganisation für
sozialistische Gewerkschaften auf Initiative von \Nam{Legien,
Carl}{Carl Legien}

\item[1891] Gründung des \Ins{Deutscher
Metallarbeiterverband}{Deutschen Metallarbeiterverbandes} -- erste
Industriegewerkschaft

\item[1890er] Gründung christlicher Gewerkschaften, Formierung zum
Verband \mar{Irgendwie stimmen hier einige Zahlen nicht.}
\end{chronik}

Ziele:

\begin{itemize}
\item soziale Ziele --Loslösung von politischen Fragestellungen
\item kürzere Arbeitszeit
\item höhere Löhne
\item Abbau der Frontstellung des Proletariats gegen das Bürgertum
(Hirsch-Duncker)
\item Förderung und Wahrung der Würde und des materiellen Interesses
der Arbeiter
\end{itemize}

Mittel:

\begin{itemize}
\item Streik
\item lokale Komitees und Arbeiterzusammenschlüsse $\longrightarrow$
Gewerkschaften $\longrightarrow$ Gewerkschaftsverbände
\item Presseorgane
\item Kassen zur Unterstützung von Arbeitslosen, Not leidenden,
Kranken, Invaliden, Alten, Wandernden
\end{itemize}

\newpage
%%%%%%%%%%%%%%%%%%%%%%%%%%%%%%%%%%%%%%%%%%%%%%%%%%%%%%%%%%%%%%%%%%%%%%

\subsection{Die Rolle der Unternehmer}
\label{ssc:soz-frag-loes-unt}

\subsubsection{Maßnahmen}

% Breite der ersten Spalte der Tabelle
\newlength{\frstcol}
\settowidth{\frstcol}{\textsc{Harkort}}
\addtolength{\frstcol}{1ex}

% Breite der zweiten und dritten Spalte der Tabelle
\newlength{\sndthrdcol}
\newlength{\wholecols} % Breite von zweiter und dritter Spalte zusammen
\setlength{\wholecols}{\textwidth}
\addtolength{\wholecols}{-\frstcol}
\addtolength{\wholecols}{-5\tabcolsep}
\setlength{\sndthrdcol}{0.5\wholecols}



\tablefirsthead{
\toprule
Untern. & fürsorglicher Charakter & unterdrückender Charakter \\
\midrule}
\tablelasttail{\bottomrule}

\begin{supertabular*}{\textwidth}%
{p{\frstcol}p{\sndthrdcol}p{\sndthrdcol}}
\vspace{0.01pt}
\Nam{Stumm-Halberg, Carl Ferdinand Freiherr von}{Stumm}
& \begin{tablist}
\item Schulen
\item Verantwortlichkeit für außerfabrikale Arbeiterhandlungen
\item Ausschluss von Kinderarbeit
\item niedrigvermietete Werkswohnungen
\item Bibliotheken, Park für die Arbeiter, Militärkapellen
\item Kantinen, Teuerungszulagen
\item Bestrafungs- und Entlassungserlaubnis
\item Sprechstunden für die Belegschaft
\item überdurchschnittliche Löhne
\item protektorale Betriebsverfassungen
\end{tablist}
\noindent $\Longrightarrow$ Schutz der Arbeiter
& \begin{tablist}
\item Heiratserlaubnis nach Eigenschaften und Gesundheit der Partner,
Vorbeugung von Arbeitsausfall während Schwangerschaften
\item Erziehungsüberwachung
\item Zwang zum Kirchenbesuch
\item Arbeiter als geborene Untertanen (\emph{König Stumm})
\index{König Stumm}
\end{tablist}
\noindent $\Longrightarrow$ Eindringen in das Privatleben
\\

\vspace{0.01pt}
\Nam{Krupp, Alfred}{Krupp}
& \begin{tablist}
\item überdurchschnittliche Löhne
\item Betriebskrankenkasse
\item Sterbegelder an Hinterbliebene
\item Werkswohnungen
\item Arbeiterpensionskasse -- Altersabsicherung
\item \ins{Gemeinschaft der Kruppianer} -- Stammbelegschaft
\end{tablist}
& \begin{tablist}
\item Nutzung des \beg{Ersatzheeres}
\item Entlassung als Druckmittel
\item Entlassung bei Parteihörigkeit
\item Betäubung von sozialdemokratischen und gewerkschaftlichen
Bestrebungen
\item leistungsabhängige Entlohnung
\item Strafgelder
\item Beitrittspflicht zur Betriebskrankenkasse
\end{tablist}
\\

\vspace{0.01pt}
\Nam{Harkort, Friedrich}{Harkort}
& \begin{tablist}
\item betriebsinterne Sparkassen -- Sicherung von Grunderwerb
\item Bildungssystem für Kinder und Erwachsene -- geistliche,
sittliche und staatsbürgerliche Bildung seiner Angestellten
\item Ablehnung von Kinderarbeit
\end{tablist}
&\vspace{0.01pt} keine klassische Unterdrückung
\\
\end{supertabular*}

\ \\

Ein weiteres bedeutendes Unternehmen, in dem man die Lage der Arbeiter
zu bessern versuchte, war \ins{Carl Zeiss Jena}. Dabei waren die
dortigen Maßnahmen einmal und auch völlig andere als bei den oben
Aufgeführten.

\endinput
