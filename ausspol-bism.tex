\section[Außenpolitik des Deutschen Reichs unter Bismarck]{Außenpolitik des Deutschen Reichs unter Bismarck\mycite[248\,f.]{braunesGeschichts}}
\mar{Bismarck: Friedensverträge}

\Nam{Bismarck, Otto von}{Bismarck} war entschlossen, den
entscheidenden Einfluss des Deutschen Reiches in Mitteleuropa zu
nutzen und das als \jar{Erzfeind} bezeichnete Frankreich zu isolieren.
Er hegte aber keinerlei Gebietsansprüche, vielmehr betonte er, dass
Deutschland \jar{saturiert} sei.

\begin{chronik}
\item[1873] \ges{Dreikaiserabkommen} zwischen dem Deutschen Reich,
Österreich-Ungarn und Russland mit verlängerbarer dreijähriger
Laufzeit

\item[1878] \ins{Berliner Kongreß} unter Leitung Bismarcks zur
Regulierung der Orientkrise: Russland sieht sich benachteiligt, die
deutsch-russischen Beziehungen verschlechtern sich und das Deutsche
Reich befürchtet eine russisch-französische Annäherung.

\item[1879] \ges{Zweibund} zwischen dem Deutschen Reich und
Österreich-Ungarn: geheimes Verteidigungsbündnis für den Fall eines
Angriffs Russlands

\item[1882] \ges{Dreibund} zwischen dem Deutschen Reich,
Österreich-Ungarn und Italien mit verlängerbarer fünfjähriger Laufzeit
als Erweiterung des Zweibunds

\item[1884] letzte Verlängerung des Dreikaiserabkommens, Ausschluss
einer militärischen Unterstützung Frankreichs durch Russland im Falle
eine deutsch-französischen Konflikts

\item[1887] geheimer \ges{Rückversicherungsvertrag} des Deutschen
Reichs mit Russland: beiderseitige Neutralität im Kriegsfall
\end{chronik}
