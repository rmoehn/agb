\section[Gründung der DDR]{Die Gründung der DDR
\footref{ftn:grob-ueb}
\mycite[334, 335]{WeltgeschNeuz}}
\label{sec:gruend-ddr}
\index{DDR!Gründung}

\begin{chronik}
\item[10.\,7.\,1945]
zentrale Zulassung von Parteien in der \textsf{\ort{SBZ}} durch die
\beg{SMAD}

\item[14.\,7.\,1945]
Gründung des \Ins{Block antifaschistisch-demokratischer
Parteien}{Blocks antifaschistisch-demokratischer Parteien} --
Zusammenarbeit von KPD, SPD, CDU und LDPD
\begin{itemize}
\item nur einstimmige Beschlussfassung
\item Bindung für alle Parteien an die Beschlüsse
\item Festlegung eines nicht näher definierten gemeinsamen Weges
\end{itemize}

\item[22.\,4.\,1946]
Auf Druck der sowjetischen Gesatzungsmacht vereinigen sich SPD und KPD
zur SED\index{SED!Gründung}, da man befürchtete, dass die KPD in
freien Wahlen nicht genügend Stimmen gewinne.

\item[Feb. 1946]
Gründung des \textsf{\ins{FDGB}}\index{FDGB!Gründung}
(Einheitsorganisation) -- Während zwar
alle Parteien in der Leitung vertreten waren, sicherten sich die
Kommunisten dennoch die Vorherrschaft.

\item[Mitte März 1948]
Zusammentreten des \Ins{Zweiter Deutscher Volkskongresses}{Zweiten
Deutschen Volkskongresses} bestehend aus 2\,000
Delegierten\footnote{Die Stimmenmehrheit lag bei der SED und deren
Verbündeten} der Blockparteien und der Massenorganisationen
(beispielsweise FDJ und FDGB) auf Initiative der SED, Wahl des
\Ins{Deutscher Volksrat}{Deutschen Volksrates} (400 Mitglieder) --
Aufgabe: Ausarbeitung einer Verfassung für eine \enquote{unteilbare
deutsche Republik}

\item[19.\,3.\,1949]
Verabschiedung eines Verfassungsentwurfes durch den Deutschen Volksrat

\item[15./16.\,5.\,1949]
Wahl des \Ins{Dritter Deutscher Volkskongress}{Dritten Deutschen
Volkskongresses} durch die Bevölkerung der \ort{SBZ} nach
Einheitsliste.

Der \Ins{Dritter Deutscher Volkskongress}{Dritte Deutsche
Volkskongress} wählt den \Ins{Zweiter Deutscher Volksrat}{Zweiten
Deutschen Volksrat}, welcher sich zur provisorischen \ins{Volkskammer}
erklärt und die \ges{Verfassung der Deutschen Demokratischen Republik}
nach Bestätigung durch den \emph{Volkskongress} in Kraft setzt. Der
Präsident der DDR wurde \nam{Wilhelm Pieck}.

\item[7.\,10.\,1949]
Inkraftsetzung der \ges{Verfassung der Deutschen Demokratischen
Republik}

\item[15.\,10.\,1950]
Wahlen zur \ins{Volkskammer} über Einheitsliste\footnote{Artikel 51
der \ges{Verfassung der Deutschen Demokratischen Republik} sah
eigentlich eine Verhältniswahl vor.} durch die ostdeutsche
Bevölkerung
\end{chronik}

\endinput
