\section{Die Epoche der Aufklärung}
\label{sec:aufkl}
\index{Aufklärung}

Die Ideen der Aufklärung bildeten die Grundlage für die politischen,
ökonomischen und sozialen Entwicklung, die schließlich zur Revolution
von 1848 führten. Sie beeinflussten außerdem seit Mitte des 18.
Jahrhunderts das Denken aller Menschen in Europa und Nordamerika und
sind bis heute gültig und wirksam.

Dieser Abschnitt soll einen kurzen Überblick über die für das Weitere
relevanten Aspekte der Aufklärung geben.


\subsection{Konflikt mit der alten Ordnung}

% wieder mal Spaltenbreiten berechnen

\begin{tabularx}{\textwidth}{XcX}
\toprule
Situation in Deutschland vor 1815 &
&
Ziele der Aufklärung \\

\midrule
\vspace{-45mm}
\begin{tablist}
\item absolutistische Monarchie 
\item Ständegesellschaft
\item territorialstaatlicher Absolutismus -- Vielzahl von
Einzelstaaten
\item Leibeigenschaft (Bindung als Person)
\item Schollenbindung (Bindung als Pacht)
\item Zünfte/Gilden
\end{tablist}
&
\unitlength=2.5mm
\begin{picture}(3,19)
\thicklines
\put(2,18){\line(-2,-9){2}}
\put(0,9){\line(3,1){3}}
\put(3,10){\vector(-2,-9){2}}
\end{picture}
&

\vspace{-45mm}
\begin{tablist}
\item Vernunftorientierung 
\item gegen Dogmatik der Kirche
\item mündiger Bürger
\item geistige und persönliche Freiheit
\item Toleranz (Religionstoleranz)
\item politisches Mitspracherecht
\end{tablist}
\\

\bottomrule
\end{tabularx}

%%%%%%%%%%%%%%%%%%%%%%%%%%%%%%%%%%%%%%%%%%%%%%%%%%%%%%%%%%%%%%%%%%%%%%

\subsection{Leistungen}

Zu einer sehr komprimierten Übersicht über die Leistungen der
Aufklärung siehe Tabelle \ref{tab:leist-aufkl}.

% angenehme Breite der ersten Spalte
\settowidth{\tymin}{\textbf{Natur-}}

\begin{sidewaystable}
\label{tab:leist-aufkl}
\caption{Leistungen der Aufklärung}
\footnotesize
\renewcommand{\arraystretch}{1.5}

\begin{tabulary}{\textheight}{LLLL}
\toprule

&
\textbf{Die alte Ordnung} &
\textbf{Die Ideen der Aufklärung} &
\textbf{Anwendung der neuen Erkenntnisse} \\

\midrule

\textbf{Naturwissenschaft} &
Spekulation \newline
Überlieferung &
empirische Methode (\Nam{Newton, Sir Isaac}{Newton}): Beobachtung --
Verallgemeinerung bzw. Experiment -- Erkenntnis \newline
einzige Grundlage: \emph{Vernunft} &
\Nam{Newton, Sir Isaac}{Newton}: Gravitationsgesetz \newline
\Nam{Kepler, Johannes}{Kepler}: Gesetze der Planetenbewegung \newline
\Nam{Watt, James}{Watt}: Dampfmaschine \newline
Diese und zahlreiche weitere Erfindungen bringen großen technischen
Fortschritt. Wissenserweiterung und Bildung gewinnen an Bedeutung.
\\

\textbf{Gesellschaft} &
Ständegesellschaft &
natürliche Freiheit und Gleichheit der Menschen (\Nam{Locke,
John}{John Locke}) &
Formulierung der Menschenrecht \\

\textbf{Religion} &
Gott und die Bibel als alleinige Autorität &
von Gott nach vernünftigen Gesetzen geschaffene Welt \newline
Religionsfreiheit &
Toleranz zwischen den Religionen \\

\textbf{Politik} &
Gottesgnadentum -- Absolutheitsanspruch des Herrschers &
Kontrolle des Herrschers durch die Bürger (\Nam{Montesquieu, Baron
de}{Montesquieu}, \Nam{Rousseau, Jean-Jacques}{Rousseau}) &
Gewaltenteilung, demokratische Verfassung, Volkssouveränität \\

\textbf{Wirtschaft} &
staatlich gelenkte Wirtschaft &
Wohlstand aller Bürger des Staates durch wirtschaftliche Freiheit des
Einzelnen (\Nam{Smith, Adam}{Smith}) &
Wirtschaftsliberalismus \\

\bottomrule 
\end{tabulary}
\end{sidewaystable}

%%%%%%%%%%%%%%%%%%%%%%%%%%%%%%%%%%%%%%%%%%%%%%%%%%%%%%%%%%%%%%%%%%%%%%

\subsection{Philosophen der Aufklärung}

Es gab zahlreiche Philosophen, die die Ideen der Aufklärung
entwickelten und verfolgten, darunter zum Beispiel \Nam{Leibniz,
Gottfried Wilhelm}{Leibniz}, \Nam{Kant, Immanuel}{Kant}, \Nam{Herder,
Johann Gottfried}{Herder}, \Nam{Lessing, Gotthold Ephraim}{Lessing}.
Im Folgenden wird jedoch nur Einblick in die Philosophie zweier Männer
gegeben, die sich besonders mit Staatstheorien beschäftigt haben und
deshalb für dieses Kapitel relevant sind.


\subsubsection{\Nam{Montesquieu, Baron de}{Charles de Secondat, Baron
de Montesquieu}}
\index{Gewaltenteilung}
\index{Legislative}
\index{Exekutive}
\index{Judikative}
\index{politische Freiheit}

\Nam{}{Montesquieu} geht in seinem Werk \emph{Vom Geist der
Gesetze}\mycite{VomGeistderGes} unter anderem auf die Frage ein, wie
ein Staat gestaltet sein muss, damit die politische Freiheit der
Bürger gewährleistet ist. Dabei definiert er \jar{politische Freiheit}
als \textquote{jene geistige Beruhigung, die aus Überzeugung
hervorgeht, die jedermann von seiner Sicherheit hat}.

Er stellt dazu fest, dass es in jedem Staat drei Gewalten gibt, die
wir heute \emph{Legislative} (Erlassen von Gesetzen), \emph{Exekutive}
(Umsetzung öffentlicher Beschlüsse) und \emph{Judikative}
(Verhandlung von Streitfällen und Verbrechen). Um also die politische
Freiheit des Bürgers zu sichern, ist es essentiell, dass jene Gewalten
personell strikt voneinander getrennt werden:

\SetBlockThreshold{0}
\blockquote[{\mycite[216\,f.]{VomGeistderGes}\,\footnote{Der
Quellenverweis gilt auch für den obigen Abschnitt.} }]{Alles wäre
verloren, wenn ein und derselbe Mann beziehungsweise die gleiche
Körperschaft entweder der Mächtigsten oder der Adligen oder des Volkes
folgende drei Machtvollkommenheiten ausübte: Gesetze erlassen,
öffentliche Beschlüsse in die Tat umsetzen, Verbrechen und private
Streitfälle aburteilen.}


\subsubsection{\Nam{Rousseau, Jean-Jacques}{Jean-Jacques Rousseau}}
\index{Grundrechte}
\index{Menschenrechte}
\index{Französische Revolution}

\blockquote{Der Mensch wird frei geboren, und dennoch liegt er in
Ketten.}

\blockquote[{\mycite{DerGesVertr}\,\footnote{Ich beziehe mich in
diesen Zitaten und in den folgenden Ausführungen auf den mir
vorliegenden Text. Dieser ist allerdings aus verschiedenen Teilen des
Werkes zusammengestückelt und weder mit Auslassungsmarken, noch mit
einer Quellenangabe versehen. Mit \cite{DerGesVertr} ist damit ein
Hinweis auf eine Volltextfassung gegeben, die allerdings im Wortlaut
nicht exakt meinem Text entspricht.}}]{Auf seine Freiheit verzichten,
heißt auf seine Eigenschaft als Mensch, auf die Menschenrechte, sogar
auf seine Pflichten zu verzichten.}

Wie man an diesen Zitaten erkennen kann, legt \Nam{}{Rousseau}
besonders großen Wert auf Freiheit. Deshalb setzt er auf seiner Suche
nach einer Gesellschaftsform, die Person und Eigentum eines jeden
schützt, also Grund- und Menschenrechte garantiert, die Prämisse, dass
der Mensch in dieser Gesellschaft den gleichen Grad an Freiheit
genießt, wie außerhalb\footnote{Das ist im
\emph{Naturzustand}\index{Naturzustand}}.  Das Ergebnis von
\Nam{}{Rousseau}s Überlegungen ist der \ges{Gesellschaftsvertrag},
welcher die Voraussetzungen erfüllt und den Rahmen für die
Zusammensetzung der Gesellschaft festlegt: Jedes Mitglied der
Gemeinschaft gibt sich mit all seinen Rechten an die Gesellschaft hin.

Inwiefern diese Auffassung utopisch beziehungsweise realistisch ist,
ist eine andere Frage. Dennoch sind die Ideen, die \Nam{}{Rousseau}
hier formuliert, essentiell und waren eine äußerst wichtige geistige
Grundlage für die \emph{Französische Revolution}.


\subsubsection{Zusammenfassung: Wesentliche Ordnungsvorstellungen der
Aufklärer}
\index{Gewaltenteilung}
\index{Volkssouveränität}
\index{Grundrechte}
\index{Menschenrechte}

\begin{enumerate}
\item  das Prinzip der \emph{Gewaltenteilung}, das die Trennung von
ausführender und gesetzgebender Gewalt sowie die Sicherung einer
unabhängigen Rechtssprechung vorsieht
\item das Prinzip der \emph{Volkssouveränität}, das das Recht zur
Gesetzgebung den Vertretern des Volkes vorbehält
\item die Bindung des Gesetzgebers an allgemein gültige \emph{Grund-
und Menschenrechte}
\end{enumerate}

%%%%%%%%%%%%%%%%%%%%%%%%%%%%%%%%%%%%%%%%%%%%%%%%%%%%%%%%%%%%%%%%%%%%%%

\subsection{Umsetzung der Ideen der Aufklärung in der
\index{Französischen Revolution}}
\index{Bürgerrechte}
\index{Menschenrechte}
\index{Grundrechte}

\mar{Siehe Blatt mit schicken Unterstreichungen aus
dem Hefter.}
Während der die \dat{\ges{Erklärung der Bürger- und Menschenrechte}
von 1789} und in die \dat{französischen Verfassungen}, deren erste
\dat{1791} verabschiedet wurde, flossen zahlreiche Ideen der
Aufklärer, insbesondere \Nam{Montesquieu, Baron
de}{Montesquieu}s und \Nam{Rousseau, Jean-Jacques}{Rousseau}s ein. Die
Französische Revolution wiederum hatte großen Einfluss auf die
politischen Entwicklungen in Deutschland, die in diesem Kapitel
behandelt werden.

\endinput
