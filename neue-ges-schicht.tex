\section{Neue gesellschaftliche Schichtung}
\label{neue-ges-schicht}
\index{Gesellschaft}
\index{Agrargesellschaft}
\index{Industriegesellschaft}
\index{Dienstleistungsgesellschaft}

Die Gesellschaft in Deutschland entwickelte sich seit dem \dat{Ende der
Agrargesellschaft 1850} über die Industriegesellschaft zur
\dat{Dienstleistungsgesellschaft ab 1990}.

Dabei kann man drei große Linien erkennen: \emph{Urbanisierung}
\index{Urbanisierung}, \emph{Trennung von Arbeit und Leben}
beziehungsweise Hausgemeinschaft und Arbeitsstätte und eine zunehmende
\emph{Differenzierung der Arbeitnehmergesellschaft} in \beg{Arbeiter}
und \beg{Angestellte}. Außerdem setzte um 1900 die Bürokratisierung
ein. \index{Bürokratisierung} 

\endinput
