\section{Innenpolitik}

\subsection*{NSDAP im Kabinett Hitler ab 30. Januar 1933}

\begin{description}
\item[\Nam{Hitler, Adolf}{Hitler}] Reichskanzler: Leitung der
Kabinettssitzungen, Bestimmung der politischen Richtlinien 

\item[\Nam{Frick, Wilhelm}{Frick}] Innenminister: verantwortlich für
die innere Sicherheit -- Vorbereitung und Durchführung von Gesetzen
und Notverordnungen (Zeitungs-, Versammlungs, Parteiverbot)

\item[\Nam{Göring, Hermann}{Göring}] Minister ohne Geschäftsbereich:
erhält das \ins{Reichskommissariat für das preußische
Innenministerium} -- Kontrolle über die preußische Polizei
\end{description}

%%%%%%%%%%%%%%%%%%%%%%%%%%%%%%%%%%%%%%%%%%%%%%%%%%%%%%%%%%%%%%%%%%%%%%

\subsection*[Maßnahmen Februar bis März 1933]{Maßnahmen Februar bis
März 1933\mycite[402]{gelbesGeschichts}}

Am \dat{1. Februar 1933} ließ \Nam{Hitler, Adolf}{Hitler} durch
\Nam{Hindenburg, Paul von}{Hindenburg} den \dat{Reichstag auflösen}.
Er begründete dieses Handeln damit, dass die Bildung einer
arbeitsfähigen Mehrheit im Parlament nicht möglich sei. Vorher hatte
er seine Scheinverhandlungen mit \Nam{Kaas, Ludwig}{Ludwig Kaas}
(Zentrum) scheitern lassen.

Danach begann der staatliche Terror, der die Ausschaltung
beziehungsweise Vernichtung der \jar{inneren} oder auch
\jar{Reichsfeinde} (Juden, gegnerische Politiker -- hauptsächlich KPD
und SPD) zum Ziele hatte. Gesetzliche Grundlage dafür war die
\dat{\ges{Reichstagsbrandverordnung} vom 28. Februar}. Diese führte
die \beg{Schutzhaft} ein; außerdem entstanden erste
Konzentrationslager.

Außerdem beeinflusste man im Hinblick auf die für \dat{März}
angesetzten \dat{Wahlen} die Bevölkerung durch Forcierung
propagandistischer Maßnahmen.

%%%%%%%%%%%%%%%%%%%%%%%%%%%%%%%%%%%%%%%%%%%%%%%%%%%%%%%%%%%%%%%%%%%%%%

\subsection*[Maßnahmen März 1933 bis August 1934]{Maßnahmen März 1933
bis August 1934\mycite[392\,--\,398]{gelbesGeschichts}}

Siehe dazu das Arbeitsblatt mit der Übersicht.

\endinput
