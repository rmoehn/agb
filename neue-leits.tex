\section{Neue Leitsektoren}
\label{sec:neue-leits}
\index{Leitsektor}

\subsection[Chemische Industrie]{Chemische Industrie\footnote{Die
Grundlage dieses Abschnitts ist ein schnell im Rahmen des Unterrichts
im Internet recherchierter Kurzvortrag. Da dieses Thema weniger
relevant ist, beschränke ich mich bei der Literatur auf die
Wikipedia-Artikel \cite{WikChemInd}, \cite{WikFarbst}, \cite{WikHaBoVerf}, \cite{WikStaudi} und \cite{WikSteikohteer}.}}
\label{ssc:chem-ind}
\index{Chemische Industrie}

\mar{Warum galt die chemische Industrie in jener Zeit als Friedens-
und damit Staatserhaltend?}
In der ersten Hälfte des 19. Jahrhunderts wurden Verfahren erfunden,
den bei der Gasherstellung anfallenden \emph{Steinkohlenteer}
\index{Steinkohlenteer} nutzbringend zu verwenden. Die neuen Produkte
-- allen voran \emph{Anilin} und \emph{Phenol}
\index{Anilin}\index{Phenol} -- waren Grundstoff für zahlreiche
Synthesen. -- Damit war der Aufstieg der Farbenindustrie eingeleitet.

Es wurden Möglichkeiten gefunden, Medikamente, Düngemittel und
Sprengstoffe künstlich herzustellen. Dadurch konnten bisher unheilbare
Krankheiten geheilt und die landwirtschaftliche Produktion erheblich
gesteigert werden. -- Gesundheits- und Versorgungszustand der
Bevölkerung verbesserten sich erheblich, sodass auch die Nachfrage
nach den immer vielseitigeren Produkten der chemischen Industrie
stieg.

Die enge Kooperation dieses Industriezweigs mit technischen
Hochschulen verbesserte das Bildungssystem und trieb die Forschung
voran. So kam es zu weiteren bedeutenden Erfindungen:

\dat{1910} wurde das \dat{\emph{\Nam{Haber, Fritz}{Haber}-\Nam{Bosch,
Carl}{Bosch}-Verfahren}} \index{Haber-Bosch-Verfahren} erfunden,
welches die Synthese von Ammoniak ermöglichte. Dieser dient als
Grundlage für Sprengstoffe und Kunstdünger. Die Verfügbarkeit neuer
Düngemittel bewirkte neue Forschungen in der Landwirtschaft.

\dat{1922} stellte \Nam{Staudinger, Hermann}{Hermann Staudinger} die
These auf, dass Polymere aus Makromoleküle bestehen und begründete
damit die Polymerchemie. Dies führte zur \dat{großtechnischen
Produktion zahlreicher Kunststoffe ab 1930} (beispielsweise
Polystyren, Polyvinylchlorid, Nylon, Buna). Die \dat{1925 gegründete}
\beg{IG Farben} war hier marktführend.

%%%%%%%%%%%%%%%%%%%%%%%%%%%%%%%%%%%%%%%%%%%%%%%%%%%%%%%%%%%%%%%%%%%%%%

\subsection[Elektroindustrie]{Elektroindustrie\footnote{Dieser und die
folgenden vier Abschnitte entstammend Kurzvorträgen, die ich nicht
selbst gehalten habe und deswegen auch keine Quellen angeben kann.}}
\label{ssc:el-ind}
\index{Elektroindustrie}

\begin{chronik}
\item[ab 1900]
erste kommerzielle Sende- und Empfangsanlagen

\item[1904]
erste Röhrendiode -- Gleichrichtung\index{Röhrendiode}

\item[1906]
erste Triode -- Grundlage für Radio und andere Unterhaltungselektronik 
\index{Triode}

\item[1926\,--\,1931] Entwicklung des Fernsehens\index{Fernsehen}

\item[1900\,--\,1950] auch international marktbeherrschende Stellung
von \beg{AEG} und \beg{Siemens}

\item[1931]
Erfindung des Elektronenmikroskops\index{Elektronenmikroskop}

\item[1941]
Erfindung des Computers (Z\,3)\index{Computer}\index{Z\,3}
\end{chronik}

Diese Entwicklungen läuteten des Computer- und Informationszeitalter
ein: Die neuen Möglichkeiten erweckten sofort großes in der
Bevölkerung, in der Politik und beim Militär, sodass sich die Anzahl
der Arbeitsplätze von \dat{80\,000 Beschäftigten 1900} bis \dat{1950
auf 650\,000} steigerte.

%%%%%%%%%%%%%%%%%%%%%%%%%%%%%%%%%%%%%%%%%%%%%%%%%%%%%%%%%%%%%%%%%%%%%%

\subsection{Fahrzeugbau}
\label{ssc:fahrzbau}
\index{Fahrzeugbau}

Sich aus dem Waggonbau entwickelnd erfuhr dieser Industriezweig in
Folge der \dat{Forderung \Nam{Hitler, Adolf}{Hitler}s} nach einem
Wagen für breite Schichten \dat{1934} großen Aufschwung. So wurde
\dat{1937} die \dat{\ins{Gesellschaft zur Vorbereitung der Deutschen
Volkswagen mbH} gegründet}, \dat{1938 in \ins{Volkswagen GmbH}}
umbenannt.

Das Werk wurde im gleichen Jahr in \ort{Wolfsburg}
errichtet.\index{VW} Es war äußerst günstig gelegen: In der Mitte
Deutschlands mit Anbindung an Autobahn, Eisenbahn und durch den
Mittellandkanal ebenfalls an Wasserstraßen. Außerdem waren Stahlwerke,
beispielsweise in \ort{Salzgitter} nicht fern. Im Krieg leisteten hier
20\,000 Kriegsgefangene und KZ-Insassen Zwangsarbeit.
\index{Zwangsarbeit}

Bei der Produktion des neu entwickelten
\emph{KdF-Wagens}\index{KdF-Wagen}, heute als \emph{VW Käfer}\index{VW
Käfer} bekannt, orientierte man sich am
\emph{Fließbandbetrieb}\index{Fließbandproduktion}, wie er bei
\ins{Ford} in \ort{Detroit} praktiziert wurde. Das neue Auto konnte
vier Personen transportieren, war zuverlässig, einfach reparierbar und
billig. So konnte man die Wirtschaft ankurbeln und die Verwendung beim
Militär war auch möglich.  Im Krieg stand dann auch die
Rüstungsproduktion im Vordergrund.  \index{Rüstung}

Nach dem Krieg war das VW-Werk der britischen Militärregierung
\index{Militärregierung} unterstellt, die die Umstellung auf
Zivilproduktion veranlasste.

\endinput
