\section{Jugend}

\ges{Dawes-Plan} (siehe \ref{sec:rev-erf-pol}) und
Inflation\index{Inflation} (siehe \ref{sec:soz-veraend})
bewirkten

\begin{itemize}
\item sozial:

\begin{itemize}
\item Jugendarbeitslosigkeit -- Jugendliche fallen den Familien länger
zur Last
\item Verbitterung durch fehlende Freizeitangebote:

\begin{itemize}
\item verstärkte Neigung zu Protest- und Gewalttaten 
\item Verlust der Autorität der Familie
\item Jugendkriminalität
\end{itemize}

\item \jar{vergreiste Republik} -- Die Jugend fühlt sich um Lebens-
und Karrierechancen betrogen.
\end{itemize}

\item politisch:

\begin{itemize}
\item hohe Anfälligkeit gegenüber radikalen Ideologien und
Gruppierungen
\item studentische Wahlen Ender der 1920er\mar{?} -- hoher
Stimmenanteil für völkisch-nationalistische Gruppen
\end{itemize}
\end{itemize}

Insgesamt fallen also auf:

\begin{itemize}
\item antidemokratisches Denken
\item starker Widerwille gegen westlich-liberale Weltreaktion \mar{?}
-- Radikalismus nach links und rechts
\item glauben an Ideen, statt an aufgeklärten Verstand
\end{itemize}

\endinput
