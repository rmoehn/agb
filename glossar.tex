\chapter{Glossar}
\label{chp:glossar}

% Achtung! Bei der Verwendung von \textquote muss die Literaturangabe
% im optionalen Argument zusätzlich in geschweiften Klammern stehen.

\begin{description}
\item[AEG]
\textbf{A}llgemeine \textbf{E}lekticitäts-\textbf{G}esellschaft. Nach
der Gründung 1876 lange Zeit eines der bedeutendsten deutschen
Elektrounternehmen; 1996 aufgelöst.\mycite{WikAEG}

\item[Angestellte]
Arbeitnehmer, die in Abgrenzung zum \beg{Arbeiter}
statt Stunden- oder Leistungslohn ein festes Monats- oder Jahresgehalt
erhalten, welches in der Regel stabiler und höher als jener ist. Als
gesellschaftliche Schicht für Arbeistorganisation und -vorbereitung,
Staatsdienst, kommunalen Dienst und andere höhere Berufe zuständig.

\item[Arbeiter]
siehe \beg{Angestellte}

\item[Blut-und-Eisen-Politik]
\textquote[{\mycite[268]{BI5Bde1}}]{Bezeichnung für eine Politik der
offenen Gewalt, insbesondere für die von \nam{Bismarck} durchgeführte
\emph{Revolution von oben} (Schaffung des bürgerlichen deutschen
Nationalstaates durch dynastische Kriege Preußens)}

\item[Burgfriedenspolitik]
Für die Dauer des \emph{Ersten Weltkrieges} zwischen den Parteien des 
deutschen Reiches geschlossener Konsens. Endete mit Nachlassen der 
Siegesgeiwissheit 1816.\mycite[189]{WeltgeschNeuz} 

\item[Chauvinismus]
Glaube an die Überlegenheit der eigenen Gruppe

\item[Ehernes Lohngesetz] \index{Ehernes Lohngesetz}
Der durchschnittliche Lohn hat ein Maß, das dem Arbeiter gerade noch
die Existenz ermöglicht.\footnote{\cite{WiLexEhLohnGes} stellt dies
anders dar. Ich muss jedoch vorerst die hier stehende Version
belassen, da der Zusammenhang im vorhergehenden Text dies so verlangt.
Das impliziert, dass dieser möglicherweise auch falsch ist.}

\item[Ersatzheer] \index{Ersatzheer}

\item[FDGB] \index{FDGB}
Freier deutscher Gewerkschaftsbund

\item[Feudalunternehmer] \index{Feudalunternehmer}
Großgrundbesitzer, die auf ihrem Land Fabriken ansiedeln.

\item[Frühindustrialisierung] \index{Frühindustrialisierung}
Vorherrschen der vorindustriellen Produktion. Nur in einzelnen
Branchen (Textil, Bergbau) wurden Fortschritte in der
Produktionstechnik erzielt, die technische Innovation erlangte jedoch
keine große gesamtwirtschaftliche Bedeutung (nur geringe
Produktivitätssteigerungen).

\item[gentry] \index{gentry}
\textquote[{\mycite[251]{BI5Bde2}}]{seit dem 15./16.\,Jh.
die aus Großpächtern, kapitalistisch wirtschaftenden Adligen und
Angehörigen der Handelsbourgeoisie hervorgegangene herrschende Klasse
Englands}

\item[Ghetto] \index{Ghetto}
In der Zeit des Nationalsozialismus hermetisch abgeriegelter 
Teil größerer verkehrgünstig gelegener polnischer Städte, in denen 
Juden unter miserablen hygienischen Bedingungen zusammengepfercht 
wurden und Zwangsarbeit leisten mussten.\mycite[211]{GeschDrReich} 

\item[Gründerjahre] \index{Gründerjahre}
\textquote[{\mycite[338]{BI5Bde2}}]{Zeit der Hochkonjunktur in
Deutschland 1871\,--\,1873, in der sich die Wirtschaft sprunghaft
entwickelte. Die nationalstaatliche Einigung und die 5 Milliarden
Goldfrancs französische Kriegskontributionen begünstigten die
Konjunktur. Wirtschaftliche Disproportionen sowie wilde
Spekulationsgeschäfte und Unternehmensgründungen kennzeichneten die
Gründerjahre. 1873 brach eine Weltwirtschaftskrise aus, die in
Deutschland ab 1874 voll wirksam wurde und als sogenannter
Gründerkrach bezeichnet wurde, da zahlreiche, vor allem auch neu
gegründete Unternehmen zusammenbrachen.}


\item[Gründerkrach] \index{Gründerkrach}
siehe \beg{Gründerjahre}

\item[Holocaust] \index{Holocaust}
Vernichtung der Juden im Nationalsozialismus

\item[I.\,G. Farbenindustrie AG] \index{I.\,G. Farbenindustrie AG}
Seinerzeit größtes Chemieunternehmen der Welt. Am 2.\,12.\,1925 aus
\ins{BASF}, \ins{Agfa}, \ins{Farbwerke Hoechst}, \ins{Cassella
Farbwerke Mainkur}, \ins{Chemische Fabrik Kalle} gegründet. Nach dem
\emph{Zweiten Weltkrieg} zur Abwicklung freigegeben.\mycite{WikIGFarb}

\item[Interdependenz] \index{Interdependenz}
\textquote[{\mycite[400]{Bertelsmann}}]{wechselseitige Abhängigkeit.}
Bezogen auf den Industrialisierungsprozeß beispielsweise die
wechselseitige Abhängigkeit von Stahlindustrie und Eisenbahn so, daß
wirtschaftsfördernde Kreisprozesse entstehen.

\item[Kinderarbeit] \index{Kinderarbeit}
Kinder werden arbeiten geschickt, um die Versorgung der Familie zu
gewährleisten. In der Zeit der Industrialisierung ab einem Alter von
zehn, in der Textilindustrie sogar ab sieben Jahren üblich.

\item[Legitimität] \index{Legitimität}
Innere Rechtfertigung der Gesetzmäßigkeiten einer monarchischen 
Regierungsform. 

\item[Madagaskarplan] \index{Madagaskarplan}
\textquote[{\mycite[433]{gelbesGeschichts}}]{Durch \nam{Reinhard
Heydrich} nach dem Sieg über Frankreich 1940 vorgeschlagene
\nat{territoriale Endlösung der Judenfrage}: Umsiedlung der Juden auf
die Insel Madagaskar, Verwaltung durch die SS.}

\item[Mediatisierung] \index{Mediatisierung}
Angliederung kleiner politischer Einheiten (Ritterschaften, freie 
Städte u.\,a.) an größere Staaten.\mycite[48]{WeltgeschNeuz} 

\item[Merkantilismus] \index{Merkantilismus}
\textquote[{\mycite[620]{gelbesGeschichts}}]{Begriff für die Politik
eines Staates im Zeitalter des Absolutismus, die die Staatsfinanzen
und den Handel als entscheidend für die Stärkung der staatlichen Macht
betrachtet. Mittel dazu waren: Stärkung der Ausfuhr und Beschränkung
der Einfuhr von Gütern, Errichtung von staatlichen
Wirtschaftsbetrieben (Manufakturen), Bau von Straßen und Kanälen
u.\,a.}

\item[Mietskaserne] \index{Mietskaserne}
\textquote[{\mycite{WikMietskas}}]{Als Mietskaserne [\dots] bezeichnet
man eine mehrgeschossige innerstädtische Wohnanlage mit einem oder
mehreren Innenhöfen aus der Zeit der Industrialisierung [\dots] für
die breite Bevölkerungsschicht der kleinen Arbeiter und Angestellten.
[\dots] Beim Bau einer Mietskaserne wurde die Grundstücksfläche im
Sinne der Gewinnoptimierung im Rahmen der Bauvorschriften bestmöglich
ausgenutzt.}

\item[Neue Ära] \index{Neue Ära}
Zeit des durch den preußischen Regenten 1858 berufenen
liberal-konservativen Ministeriums

\item[Nicht-Arier] \index{Nicht-Arier}
Eltern- oder Großelternteil

\item[Parlamentarischer Rat] \index{Parlamentarischer Rat}
verfassungsgebende Versammlung. Gremium mit parlamentarischem
Charakter, welches durch die Ministerpräsidenten der Länder auf
Anweisung der Westalliierten eingesetzt wurde.

\item[Reichsdeputationshauptschluss]
\index{Reichsdeputationshauptschluss}
\textquote[{\mycite[315]{BI5Bde4}}]{Beschluß eines
Reichstagsausschusses zur territorialen Neugliederung des deutschen
Reiches, unter Druck \Nam{Napol\'eon \Rm{1}}{Napoleons \Rm{1}} am
25.\,2.\,1803 gefaßt. Er beseitigte fast alle geistlichen Fürstentümer
und 44 Reichsstädte sowie alle Reichsdörfer zugunsten größerer
Territorialstaaten und damit die schlimmsten Auswüchse der staatlichen
Zersplitterung; leitete die endgültige Auflösung des \ort{Heiligen
Römischen Reiches Deutscher Nation} ein.}

Siehe auch \beg{Mediatisierung} und  \beg{Säkularisation}. 

\item[Restauration] \index{Restauration}
\textquote[{\mycite[624]{gelbesGeschichts}}]{Wiederherstellung früherer
Zustände, z.\,B. der monarchischen Ordnung eines Staates, Als
Epochenbezeichnung für die Jahre 1815\,--\,1849 betont der Begriff,
dass die staatliche Politik dieser Jahre alte Grundsätze der Zeit vor
der Französischen Revolution wieder zur Geltung bringen wollte.}

\item[Rheinbund] \index{Rheinbund}
\textquote[{\mycite[335]{BI5Bde4}}]{am 12.\,7.\,1806 geschlossener Bund
von zunächst 16 deutschen Staaten unter dem Protektorat \nam{Napoleons
\Rm{1}}; sollte die Herrschaft Frankreichs über Deutschland sichern
sowie zusätzliche Truppen und Mittel für naoleonische Eroberungskriege
bereitstellen.  Seine Gründung führte zur Auflösung des deutschen
Reiches. Bis 1811 schlossen sich alle deutschen Staaten, außer Preußen
und Osterreich an. [\dots]} 

\item[Säkularisation] \index{Säkularisation}
Verstaatlichung kirchlichen Besitzes. 

\item[SBZ] \index{SBZ}
Sowjetische Besatzungszone

\item[Schaukelstuhlpolitik] \index{Schaukelstuhlpolitik}

\item[Schlafgänger] \index{Schlafgänger}
In der Zeit der Industrialisierung Personen, die bei den armen
Arbeitern ein Bett für wenig Geld \nat{mieteten}, während deren
\nat{Inhaber} arbeiteten.

\item[Schutzhaft]
verfahrenslose und auf Verdacht basierende Inhaftierung von
Regimegegnern in der Zeit des
Nationalsozialismus\mycite[402]{gelbesGeschichts}

\item[Schutzzollpolitik] \index{Schutzzollpolitik}
Durch \nam{Bismarck} ab 1879 eingeführte Zölle, die die Wirtschaft des
Deutschen Reiches stärken und die Preise im Land stabil halten
sollten.\mycite{WikSchutzz}\mycite{WikZoll}

\item[Siemens] \index{Siemens}
1847 gegründetes großes deutsches Elektrounternehmen.

\item[SMAD] \index{SMAD}
Sowjetische Militäradministration

\item[Solidarität] \index{Solidarität}
Gegenseitige Hilfe der Staaten.

\item[Soziale Marktwirtschaft]
Wettbewerbsgesteuerte Wirtschaftsordnung, die mehr Gerechtigkeit und
soziale Abfederung durch Eingriffe des Staates vorsieht. Zu dessen
Maßnahmen gehören Antimonopolpolitik, Einkommensumverteilung durch
progressive Steuer- und Subventionspolitik sowie ein durch
verschiedene staatliche Versicherungen gewährleistetes soziales
Netz.\footnote{Genauere Erklärungen finden sich in gut sortierten
Gemeinschaftskundeheftern, aber auch in \cite[366\,f.]{DudPolGes}.}

\item[Take-off-Phase] \index{Take-off-Phase}
Durch Wachstum bestimmte Phasen beschleunigter Industrialisierung.
Leitsektoren des technischen Fortschritts und der
Produktionssteigerung, die durch starke Vorwärts- und
Rückkopplungseffekte (Interdependenzen) verbunden waren, sind die
Eisenbahn- und die Schwerindustrie. Industrielle Produktionsmethoden
setzen sich durch, neue Produktionszweige expandieren, Güteraustausch
sowie die gesellschaftliche Arbeitsteilung weiten sich stark aus.

\item[Taylorismus] \index{Taylorismus}
Nach dem amerikanischen Ingenieur \Nam{Taylor, Frederick
Winslow}{Frederick Winslow Taylor} benanntes Produktionsprinzip. Mit
dem Ziel der Steigerung der menschlichen Arbeitsproduktivität wird der
Produktionsprozess in kleinste Einheiten zergliedert. Jeder
Arbeitsschritt wird nur von einem Arbeiter ausgeführt, sodass durch
starke Repetitivität kaum Denken erforderlich ist und durch
Optimierung ein hoher Arbeitstakt erreicht wird.\mycite{WiLexTaymus}

\item[Vernichtungslager] \index{Vernichtungslager}
In der Zeit des Nationalsozialismus betriebene spezielle
Konzentrationslager, in denen hauptsächlich Juden
gleichsam industriell emordet wurden.

\item[Vernunftrepublikaner]
Bürger der Weimarer Republik, die diese aus Vernunft, aber nicht aus
innerer Überzeugung unterstützten.

\item[Wirtschaftsliberalismus] \index{Wirtschaftsliberalismus}
Nach der Theorie Adam Smiths aufgebaute Wirtschaftsordung ohne 
Kontrolle durch den Staat mit Handels-, Wettbewerbs, Produktions- und 
Vertragsfreiheit. Das Gemeinwohl wird durch das Gewwinnstreben und 
die Arbeit des Einzelnen gefördert. Der Markt reguliert sich über 
Angebot und Nachfrage.\mycite[155, 166]{braunesGeschichts} 

\item[Zuckerbrot und Peitsche] \index{Zuckerbrot und Peitsche}
Bezeichnung für das Vorgehen Bismarchs gegen die Sozialdemokraten bei
gleichzeitiger Einführung der Sozielgesetzgebung. 

\item[Zweite Industrialisierung] \index{Zweite Industrialisierung}
Die Industrie gewann die beherrschende Stellung in der Vokswirtschaft.
Neue Wachstumsindustrien, vor allem Chemie und Elektrotechnik,
entwickeln sich zu neuen Leitsektoren der Wirtschaft.  Drastische
Konzentrations- und Expansionsprozesse, aus denen die für die deutsche
Industrialisierung typischen großen Aktiengesellschaften
(Groß"-unternehmen, Großbanken) und über 200 Kartelle hervorgingen.
\end{description}

\endinput
