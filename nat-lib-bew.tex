\section[Nationale und liberale Bewegungen]{Nationale und liberale
Bewegungen\mycite{19Jh1}}
\label{sec:nat-lib-bew}
\index{nationale Bewegungen}
\index{liberale Bewegungen}

\mar{Belege aus IzpB?}
\subsection{Befreiungskriege gegen \Nam{Napol\'eon \Rm{1}}{Napol\'eon}}
\index{Befreiungskriege}
\index{Napol\'eonische Kriege}
\index{Nationalromantik}
\index{Nationalismus}
\index{Franzosenhass}

\mar{Muss das genauer gewusst werden?}
Der Begriff \emph{Befreiungskriege} steht hier für die Kriege
vornehmlich der von \Nam{}{Napol\'eon} besetzten Länder gegen die
französische Fremdherrschaft in der Zeit zwischen dem \dat{Rückzug der
\emph{Grande Arm\'ee} aus Russland 1812} und der \dat{Abdankung des
Kaisers 1814}. Jene Erfahrungen, die des gemeinsam geführte Krieges
und der Fremdherrschaft, waren eine wichtige Wurzel der
nationalen Bewegung, deren Formen von Nationalromantik bis zum
radikalen Nationalismus reichten. Ein verbreitetes Symptom war auch
der teils extreme Hass auf Franzosen.


\subsubsection{Ausgangslage}

Der \dat{Russlandfeldzug} und die
einhergehende Niederlage der Grande Arm\'ee hatten \dat{1812} Napol\'eon
gewaltige Verluste eingebracht.  Dadurch fasste Europa neuen Mut, eine
Befreiung von der französischen Herrschaft zu versuchen.


\subsubsection{Erhebung Preußens}

\begin{chronik}
\item[30.\,12.\,1812]
Unterzeichnung der \ges{Konvention von Tauroggen} als
preußisch-russisches Neutralitätsabkommen -- Anfang vom Ende der
erzwungenen militärischen Unterstützung Frankreichs durch Preußen

\item[28.\,2.\,1813]
Unterzeichnung des \Ges{Bündnis von Kalisch}{Bündnisses von Kalisch}
durch \nam{Wilhelm \Rm{3}} -- Zusammenarbeit Preußens und Russlands
für die Befreiung Europas

\item[10.\,3.\,1813]
erstmalige Verleihung des \emph{Eisernen Kreuzes} \index{Eisernes
Kreuz} durch Wilhelm \Rm{3}

\item[16.\,3.\,1813]
Kriegserklärung Preußens an Frankreich

\item[17.\,3.\,1813]
Aufruf \ges{An Mein Volk} von Wilhelm \Rm{3}\mycite{AnMeinVolk}
\end{chronik}


\subsubsection{Frühjahrsfeldzug 1813}
\index{Antinapoleonische Koalition}

\begin{chronik}
\item[2. und 20. Mai]
Schlachten bei \ort{Großgörschen} und \ort{Bautzen} -- Siege Napol\'eons

\item[12. Juli]
\emph{Trachenbergplan} \index{Trachenbergplan} -- gemeinsame Strategie
Preußens, Russlands und Schwedens gegen Napol\'eon

\item[11. August]
Kriegserklärung Österreichs an Frankreich. Zuvor waren Österreich,
Schweden und Großbritannien der Koalition gegen Napol\'eon beigetreten.
Deren Ziele waren die vollständige \emph{Wiederherstellung}
Österreichs und Preußens, die \emph{Unabhängigkeit} aller deutschen
Staaten sowie die Auflösung des \ins{Rheinbund}es.\footnote{Ich
vermute, dass das in dem Handout des Vortrags, der dieses Thema
behandelte, so gemeint war.}
\end{chronik}


\subsubsection{Herbstfeldzug 1813}

\dat{Ende August 1813} stießen drei russisch-schwedisch-preußische
Armeen unter österreichischem Oberbefehl gegen Napol\'eon vor:
\ins{Nordarmee} und \ins{Schlesische Armee} bei \ort{Dresden} und die
\ins{Hauptarmee} in \ort{Böhmen}.\footnote{Zumindest soweit ich das
Handout verstehe.}


\subsubsection{Weitere Schlachten 1813}
\index{Alliierte!Napol\'eonische Kriege}

\begin{chronik}
\item[23. Aug.] \ort{Großbeeren} -- Sieg der Alliierten
\item[26. Aug.] \ort{Katzbach} -- Sieg der Alliierten
\item[26./27. Aug.] \ort{Dresden} -- Sieg Napol\'eons
\item[20. Aug.] \ort{Kullm} und \ort{Nollendorf} -- Sieg der
Alliierten
\item[6. Sept.] \ort{Dennewitz} -- Sieg der Alliierten
\end{chronik}

In der Folge kündigte auch Bayern das Bündnis mit Napol\'eon und trat
der Koalition bei, was die Auflösung des Rheinbundes einleitete.

Mit der Umzingelung Napol\'eons bei \ort{Leipzig} kam es schließlich zur
\emph{Völkerschlacht}.


\subsubsection{Völkerschlacht bei Leipzig}
\index{Völkerschlacht}

Napol\'eon hatte sich mit stark geschwächtem Heer und knapper Munition
bei Leipzig aufgestellt. Die am \dat{16. Oktober 1813} beginnende
Schlacht verlief für die Alliierten erfolgreich. Der Sieg brachte den
Verlust des Rheinbundes und den Zusammenbruch der französischen
Herrschaft über Deutschland.

Da Napol\'eon jedoch am \dat{19. Oktober} rechtzeitig den
\dat{Rückzug} einleitete, konnte ein großer Teil seiner Armee
entkommen.


\subsubsection{Untergang Napol\'eons}

Die Alliierten marschierten nun in Frankreich ein und nahmen
\dat{\ort{Paris} am 31. März 1814}. Napol\'eon dankte ab und ging ins
Exil. Der \dat{\ges{Frieden von Paris} vom 30. Mai 1814} brachte das
vorläufige Ende der Napol\'eonischen Kriege.

%%%%%%%%%%%%%%%%%%%%%%%%%%%%%%%%%%%%%%%%%%%%%%%%%%%%%%%%%%%%%%%%%%%%%%

\subsection{Ausgangslage am Anfang des 19. Jahrhunderts}

War Deutschland bis 1871 keine Nation, existierte vor der
napol\'eonischen Besatzungszeit lediglich eine Vielzahl (über 100)
kleiner bis winziger Territorialstaaten. Zu einer Milderung dieses
Zustandes kam es durch Erscheinungen der \emph{Industrielle
Revolution}, durch \Nam{Napol\'eon \Rm{1.}}{Napol\'eon} selbst und im
Zuge der Befreiung von der französischen Fremdherrschaft:


\subsubsection[Veränderungen durch die Industrielle
Revolution]{Veränderungen durch die Industrielle Revolution (siehe
auch \ref{sec:aufh-rueckst-d})}

\begin{itemize}
\item Zollvereine: \ins{Allgemeiner deutscher Zollverein},
\ins{Zollvereinstaler}
\item Verbesserung der Infrastruktur (bspw. Eisenbahn) 
\item einheitlicher Markt
\end{itemize}


\subsubsection{Veränderungen durch
Napol\'eon\mycite[244-246]{gelbesGeschichts}}
\label{sss:veraend-nap}
\index{Flurbereinigung}
\index{Kleinstaaterei!Ende}

\begin{enumerate}
\item \emph{Flurbereinigung}:
Nachdem die linksrheinischen Fürsten ihre \dat{1801 Gebiete an
Frankreich abtreten} mussten, wurden sie durch den
\dat{\ges{Reichsdeputationshauptschluss} 1803} mit Land aus der von
Napol\'eon betriebenen \beg{Säkularisation} und \beg{Mediatisierung}
entschädigt.

\item \dat{Mediatisierung zahlreicher Reichsritterschaften 1805}

\item \dat{Gründung des \Ins{}{\beg{Rheinbundes}} 1806}:
In dieser Verwaltungseinheit von 16 deutschen Staaten führte Napol\'eon 
den \ges{Code civil} ein. Diese Gesetzessammlung galt bereits in
Frankreich, brachte liberale Reformen mit sich und bildete eine sich
über die Rheinbundstaaten erstreckende einheitliche Gesetzesgrundlage.
\end{enumerate}


\subsubsection{Veränderungen im Zuge der Befreiung von Napol\'eon}
\mar{Wo sind sie denn?}

%%%%%%%%%%%%%%%%%%%%%%%%%%%%%%%%%%%%%%%%%%%%%%%%%%%%%%%%%%%%%%%%%%%%%%

\subsection[Der Wiener Kongress 18. September 1814\,--\,9. Juni
1815]{Der Wiener Kongress 18. September 1814\,--\,9. Juni
1815\mycite[80\,--\,84, 88/89]{braunesGeschichts}\,\mycite[252\,--\,254]{gelbesGeschichts}}
\index{Wiener Kongress}
\index{Aufklärung}
\index{Nationalbewusstsein}

\begin{aufgabe}
Bewerten Sie anhand der Prinzipien den Charakter des Wiener
Kongresses! 
\end{aufgabe}

In deutschen Landen hatte sich während der Zeit der naol\'eonischen
Herrschaft einiges getan: Im \emph{territorialen Bereich} gab es
Veränderungen der Länder, der Grenzen und der Fürstentitel, in der
\emph{Wirtschaft} hatte sich der Markt vereinheitlicht und
\emph{politisch} gab es noch mehr Zündstoff: Die \emph{Aufklärung}
(siehe auch \ref{sec:aufkl} Griff um sich, die Befreiungskriege hatten
Anfänge eines \emph{Nationalbewusstseins} geschaffen und einige
Staatsleute begannen, der Entwicklung mit \emph{Reformen} beizukommen.

Um mit den Ereignissen Schritt zu halten, trafen sich 200 Vertreter
vieler europäischer Länder nach dem vorläufigen Sieg über Napol\'eon
in \ort{Wien} zu einem Kongress unter dem Vorsitz des \Nam{Metternich,
Klemens Wenzel Lothar von}{Fürsten von Metternich}.

Unter den zentralen Prinzipien

\begin{center}
\Large
%\begin{tabularx}{0.75\textwidth}{ccc}
%\textbf{\beg{Legitimität}} & \textbf{\beg{Solidarität}}
% & \textbf{\beg{Restauration}} 
%\end{tabularx}
\end{center}

\noindent berieten sie über die \emph{Neuregelung} europäischer
Grenzen und damit verbundenen territorialen \emph{Ausgleich}, über die
\emph{Fixierung} der Niederlage Frankreichs, politische
\emph{Regulierung} und ein europäisches
\emph{Sicherheitssystem}.\index{Sicherheitssystem} Damit einher ging
die Festsetzung der \emph{Pentarchie}, also die Voherrschaft der fünf
europäischen Großmächte Frankreich, Großbritannien, Österreich, Preußen 
und Russland in Europa.\index{Pentarchie}

An seinen Prinzipien eindeutig erkennbar, war der Wiener Kongress ein
Rückschritt in Bezug auf die nationale und liberalen Bemühungen in
deutschen Landen, wobei diese noch kaum begonnen hatten.

%%%%%%%%%%%%%%%%%%%%%%%%%%%%%%%%%%%%%%%%%%%%%%%%%%%%%%%%%%%%%%%%%%%%%%

\subsection{Nationale Bewegung}
\index{nationale Bewegungen}
\index{Einheit!19. Jahrhundert}
\index{Nationalstaat}

\noindent Wurzeln:

\begin{itemize}
\item die \emph{Französische Revolution}\index{Französische
Revolution} -- Die Leistungen, die die Menschen hier durch ihren
Patriotismus gebracht hatten, besonders natürlich die großen Erfolge
des materiell und personell Unterlegenen französischen Heeres, hatte
große Vorbildwirkung auf Deutschland.

\item die \emph{Preußischen Reformen} (siehe auch
\ref{sec:aufh-rueckst-d}) -- Sie erhöhten in den Befreiungskriegen
Kampfmoral und Bereitschaft zum Kampf und nährten die Hoffnung auf
einen Nationalstaat.

\item \Nam{Napol\'eon \Rm{1}}{Napol\'eon} als \emph{der} Übermittler
des Nationalbewusstseins -- das sich letztlich gegen ihn richtete.

\item Französische Fremdherrschaft und Ablehnung des
\jar{Franzosentums} betreiben die Bildung einer deutschen Identität.

\item der Aufruf \Nam{Wilhelm \Rm{3}}{Wilhelms \Rm{3}} \ges{An Mein
Volk}\mycite{AnMeinVolk}
\end{itemize}


\noindent Ziel:
\index{Kleinstaaterei!Ende}

Das zentrale Ziel der nationalen Bewegung war natürlich die Einheit
Deutschlands. Diese war zwar durch den \ins{Zollverein} (siehe
\ref{ssc:zollv} im wirtschaftlichen Bereich gegeben und Napol\'eon
hatte durch seine Flurbereinigung (siehe \ref{sss:veraend-nap})
der Kleinstaaterei ein Ende gesetzt. Dennoch konnte man nicht von
Deutschland als einem \emph{Nationalstaat} sprechen.


%%%%%%%%%%%%%%%%%%%%%%%%%%%%%%%%%%%%%%%%%%%%%%%%%%%%%%%%%%%%%%%%%%%%%%

\subsection{Liberale Bewegung}
\index{liberale Bewegungen}
\index{Gewaltenteilung}
\index{Volkssouveränität}
\index{Grundrechte}
\index{Menschenrechte}
\index{Gottesgnadentum!Abschaffung}

\noindent Ursprung:
\index{Gleichheit}
\index{Freiheit}
\index{Staatstheorie}
\index{Vertragstheorie}
\index{Gesellschaftsvertrag}
\index{Privateigentum}

Die Grundlage der liberalen Bewegungen war die bereits in Abschnitt
\ref{sec:aufkl} besprochene Aufklärung, die bekanntlich die Vernunft
ganz vornanstellt und an ihr Erkenntnis und Handeln bemisst. Über
diese gelangten dann auch die großen Vertrags- beziehungsweise
Staatstheoretiker \Nam{Locke, John}{John Locke}, \Nam{Montesquieu,
Baron de}{Baron de Montesquieu} und \Nam{Rousseau,
Jean-Jacques}{Jean-Jacques Rousseau}  mit ihren Ansätzen
genannt:\footnote{Natürlich gehört zu diesen auch \Nam{Hobbes,
Thomas}{Thomas Hobbes}. Doch dieser hat mit zur liberalen einer eher
unbedeutende Beziehung.} zu den Prinzipien, die auch zu den Prinzipien
der liberalen Bewegung wurden:

\begin{itemize}
\item Einführung eines Gesellschaftsvertrages zur eindeutigen
Festlegung der Rechte und Pflichten der Bürger und der Regierung eines
Landes und als Begründung beziehungsweise Legitimierung der Herrschaft
des regierenden Personenkreises -- kein \emph{Gottesgnadentum}
\item Volkssouveränität
\item Gewaltenteilung
\item Sicherung des Privateigentums
\item natürliche Gleichheit und Freiheit des Individuums
\item angeborene, unantastbare und unveräußerliche Grund- und
Menschenrechte
\end{itemize}


\noindent Zentrale Ziele:
\index{Feudalismus!Abschaffung}
\index{Wirtschaft!freie}
\index{Merkantilismus!Abschaffung}
\index{Wahl}

\begin{itemize}
\item Abschaffung der feudalen Gesellschaftsordnung 
\item Begrenzung der Fürstenmacht durch Verfassungen -- Anerkennung
von Grund- und Menschenrechten, Gewaltenteilung
\item Mitwirkung der Bürger durch gewählte Vertreter an der
Staatslenkung
\item freie Wirtschaft
\end{itemize}

Für die Erreichung dieser Ziele waren durch den \ges{Code civil} im
Rheinbund und die Reformen in Preußen zumindest Ansätze vorhanden.

%%%%%%%%%%%%%%%%%%%%%%%%%%%%%%%%%%%%%%%%%%%%%%%%%%%%%%%%%%%%%%%%%%%%%%

\subsection{Freiheit und Einheit}
\index{Französische Revolution}

Natürlich bildeten Anhänger der liberalen und nationalen Bewegung
nicht unterschiedliche Lager. Vielmehr entwickelten sich in immer
größeren Teilen der Bevölkerung Forderungen nach \emph{Freiheit und
Einheit}. Dies waren auch die Ideen der Französischen Revolution
gewesen, die dort umgesetzt worden waren und dann durch die
Eroberungskriege \Nam{Napol\'eon \Rm{1}}{Napol\'eon}s nach Deutschland
getragen wurden.

\subsubsection{Burschenschaften}
\index{Burschenschaften}

Die hauptsächlichen Träger solcher Ideen waren in Deutschland die
\emph{Burschenschaften}. \dat{1817} formulierten
sie ihre \dat{\emph{Grundsätze}}\mycite{GrundsBursch} -- sowohl
nationale als auch liberale Ziele und Forderungen, deren letztere hier
zusammenfassend genannt werden sollen: \\

\noindent konstitutionell:
\begin{itemize}
\item Gewählte Vertreter erlassen Gesetze
\item Alle höheren Ämter müssen sich vor dem Volk verantworten.
\item Die Fürst führt den Willen des Volkes aus. 
\end{itemize} 

\noindent wirtschaftlich:\\
Schaffung eines einheitlichen Marktes durch Abschaffung von Zöllen,
Einführung einheitlicher Maße und Gewichte etc.\\

\noindent sozial:
Freiheit und Gleichheit generell und vor dem Gesetz \\

Diese Forderungen standen natürlich den Auffassungen der Restauration
und des Wiener Kongresses entgegen. Deshalb wurden mit den
\dat{\ges{Karlsbader Beschlüsse}n 1819} scharfe Restriktionen
eingeführt (siehe \ref{ssc:kampf-lib-res}).


\subsubsection{Vormärz in der Literatur}
\index{Vormärz!Literatur}

Während neben den Burschenschaften nur kleine Teile der Bevölkerung
aktiv die Ideale Freiheit und Einheit verfolgten, entwickelte sich in
der Literatur rasch die Strömung des \emph{Vormärz}. -- Dichter wurden
zum Sprachrohr der Bewegung und zum Stimmungsbarometer. Daher sollen
hier Auszüge aus drei Gedichten der damaligen Zeit Beachtung finden.

Derer Erstes stammt von \Nam{Körner, Theodor}{Theodor Körner} und
damit aus der Zeit der Befreiungskriege\index{Befreiungskriege}:

\poemtitle*{\Nam{}{Theodor Körner} (1791\,--\,1813): Aufruf
(1813)\mycite{KoernerAufr}}
\settowidth{\versewidth}{Frisch auf, mein Volk! – Die Flammenzeichen rauchen,}

\begin{verse}[\versewidth]
Frisch auf, mein Volk! Die Flammenzeichen rauchen, \\
Hell aus dem Norden bricht der Freiheit Licht. \\
Du sollst den Stahl in Feindes Herzen tauchen; \\
Frisch auf, mein Volk! – Die Flammenzeichen rauchen, \\
Die Saat ist reif – ihr Schnitter, zaudert nicht! \\
Das höchste Heil, das letzte, liegt im Schwerte! \\
Drück' dir den Speer ins treue Herz hinein! – \\
Der Freiheit eine Gasse! – Wasch' die Erde, \\
Dein deutsches Land, mit deinem Blute rein! \\
\mbox{[\dots]}
\end{verse}


\poemtitle*{\Nam{Herwegh, Georg}{Georg Herwegh} (1817\,--\,1875):
Aufruf (1814)\mycite{HerweghAufr}}
\settowidth{\versewidth}{Spricht er wohl den Segen drein.}

\begin{verse}[\versewidth]
\mbox{[\dots]} \\
Reißt die Kreuze aus der Erden! \\
Alle sollen Schwerter werden, \\
Gott im Himmel wird's verzeihn. \\
Hört er unsre Feuer brausen \\
Und sein heilig Eisen sausen, \\
Spricht er wohl den Segen drein. \\
\mbox{[\dots]}
\end{verse}

Das Gedicht, aus dem der folgende Auszug stammt, bezieht sich auf den
\dat{Aufstand der schlesischen Weber 1844} (siehe
\ref{ssc:kampf-lib-res}). Das Wort \jar{Fluch} in der vierten
Zeile bezieht sich auf Gott, König und falsches Vaterland, die in den
vorhergehenden Strophen verflucht werden:

\poemtitle*{\Nam{Heine, Heinrich}{Heinrich Heine} (1797\,--\,1856):
Die schlesischen Weber (1847)\mycite{HeineWeber}}
\settowidth{\versewidth}{Altdeutschland, wir weben dein Leichentuch,}

\begin{verse}[\versewidth]
\mbox{[\dots]} \\
Das Schiffchen fliegt, der Webstuhl kracht, \\
Wir weben emsig Tag und Nacht -- \\
Altdeutschland, wir weben dein Leichentuch, \\
Wir weben hinein den dreifachen Fluch -- \\
\vin Wir weben, wir weben!\flq \\
\end{verse}

Heine war übrigens den Auswüchsen des Nationalismus gegenüber
recht kritisch eingestellt, wie in seinem Kommentare über das
Wartburgsfest und seine Zeit in einer \Ort{Göttingen}{Göttinger}
Burschenschaft zeigen:

\blockquote{Auf der Wartburg hingegen herrschte jener unbeschränkte
Teutomanismus, der viel von Liebe und Glaube greinte, dessen Liebe
aber nichts anderes war als Haß des Fremden und dessen Glaube nur in
der Unvernunft bestand, und der in seiner Unwissenheit nichts Besseres
zu erfinden wußte, als Bücher zu verbrennen!}

\blockquote[{\mycite{HeineBursch}}]{Im Bierkeller zu Göttingen musste
ich einst bewundern, mit welcher Gründlichkeit meine altdeutschen
Freunde die Proskriptionslisten anfertigten, für den Tag, wo sie zur
Herrschaft gelangen würden. Wer nur im 7. Glied von einem Franzosen,
Juden oder Slawen abstammte, ward zum Exil verurteilt. Wer nur im
mindesten etwas gegen Jahn oder überhaupt gegen altdeutsche
Lächerlichkeiten geschrieben hatte, konnte sich auf den Tod gefasst
machen\dots}



%%%%%%%%%%%%%%%%%%%%%%%%%%%%%%%%%%%%%%%%%%%%%%%%%%%%%%%%%%%%%%%%%%%%%%

\subsection{Politisierung der Bewegung}

\begin{table}
\caption{Anzahl und Charakter der Volksunruhen in Deutschland}
\label{tab:volksunr}

\newcolumntype{Y}{>{\centering}X}
\newcolumntype{Z}{>{\raggedright}X}
\newlength{\studlength}
\settowidth{\studlength}{Studenten/}
\newlength{\sozlength}
\settowidth{\sozlength}{sozioöko-}

\begin{tabularx}{\textwidth}{lYYYY|Y}
\toprule

&
\parbox{\studlength}{Studenten/\\Universität}\vspace{0.5ex} &
Religion &
Politik &
\parbox{\sozlength}{sozioöko-\\nomisch} &
Summe \\

\midrule

1816\,--\,29 &
13 &
9 &
4 &
3 &
29 \\

1830\,--\,39 &
13 &
20 &
72 &
28 &
133 \\

1840\,--\,47 &
5 &
17 &
33 &
103 &
158 \\

\midrule
Summe &
31 &
46 &
109 &
134 &
320 \\

\bottomrule 
\end{tabularx}
\end{table}

Wie Tabelle \ref{tab:volksunr} zeigt, gewannen \dat{ab 1830
politische Unruhen} deutlich an Dominanz. Sie bewirkten die Ausweitung
politischer Arbeit beziehungsweise politischen Interesses auf breitere
Bevölkerungsschichten. Die politischen Probleme wurden zwar im Zuge
der Industriellen Revolution\index{Industrielle Revolution} von
sozioökonomischen überdeckt. Doch diese vergrößerten ebenso das
Ursachenfeld für eine eventuelle Revolution, die dadurch natürlich
wahrscheinlicher wurde.


\subsubsection{Charakter der Ziele von Liberalen und Demokraten}
\index{Republik}
\index{parlamentarische Monarchie}
\index{Revolution}
\index{Reform}

Demokraten und Liberale vertraten unterschiedliche Ansichten in Bezug
auf die Neugestaltung des Staates. Wie im sogenannten \ges{Offenburger
Programm}\mycite{OffenbProg} deutlich wird, zielten die Demokraten auf
die Errichtung einer Republik oder parlamentarischen Monarchie ab.
Dabei hat ihre Schrift einen einigermaßen revolutionären Charakter.

Die Liberalen suchten mit ihrem \ges{Heppenheimer
Programm}\mycite{HeppProg} jedoch
eher nach Veränderungen auf der bestehenden Rechtsbasis. -- Man
erkennt den reformerischen Charakter.


\subsubsection{Demokratisches und restauratives Prinzip}

\begin{aufgabe}
Erklären sie wesentlich Unterschiede zwischen dem demokratischen und
dem restaurativen Prinzip! 
\end{aufgabe}

\noindent demokratisches Prinzip:
\index{demokratisches Prinzip}

\begin{itemize}
\item Gesetzte werden vom Volk oder Volksvertretern erlassen.
\item Gewaltenteilung
\item Gleichheit aller vor dem Gesetz
\item Gewährung persönlicher Freiheiten (Presse-, Meinungsfreiheit,
\dots) 
\end{itemize}

\noindent restauratives Prinzip
\index{restauratives Prinzip}

\begin{itemize}
\item Stützung der absolutistischen Monarchie
\item alle Macht beim Fürsten -- Verantwortung nur vor Gott
\item Einschränkung persönlicher Freiheit
\item keine Macht im Volk 
\end{itemize}

\endinput

% Ziel: Freiheit nach außen und Selbstbestimmung nach innen
