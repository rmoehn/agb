\section[Revisions- und Erfüllungspolitik -- Die Rolle politischer
Handlungsträger]{Revisions- und Erfüllungspolitik -- Die Rolle
politischer
Handlungsträger\mycite[339\,--\,348]{braunesGeschichts}\,\mycite[356\,--\,360]{gelbesGeschichts}\,\mycite[26\,f.,
32\,f.]{IzpBWeimRep}}
\label{sec:rev-erf-pol}
\index{Revisionspolitik}
\index{Erfüllungspolitik}

\subsection*{Vertrag von Rapallo}

Nach dem Ersten Weltkrieg standen die Russische Sozialistische
Föderative Sowjetrepublik (RSFSR) und das Deutsche Reich
wirtschaftlich und politisch isoliert da. -- Erstere wegen der
westlichen Angst vor dem Kommunismus, letzteres wegen seiner Rolle im
Krieg. In dieser Situation beschlossen sie \dat{1922 im \ges{Vertrag
von Rapallo}} gegenseitigen \Ort{}{Rapallo}
\emph{Reparationsverzicht}\index{Reparationen!Rapallo}, Aufnahme
\emph{diplomatischer Beziehungen} und
\emph{Meistbegünstigung}\index{Meistbegünstigung!Rapallo} in den
wirtschaftlichen Beziehungen. Gleichzeitig wurde der \ges{Vertrag von
Versailles} unterwandert, indem die Hilfe deutscher Experten bei der
Entwicklung russischen schweren Kriegsgeräts vereinbart wurde, was mit
der Möglichkeit der Übung deutscher Militärs am Material einherging.

Der Vertrag von Rapallo verärgerte also durch die Veränderung der
Situation in Osteuropa Großbritannien und Frankreich und übte
gleichzeitig Druck auf diese aus. Denn nun waren Polen und die
Tschechoslowakei, die eine Schutzgürtel gegen den Kommunismus bilden
sollten, in eine Zweifrontenlage geraten. Es war nun die Zeit für
Deutschland gekommen, außenpolitisch aktiv zu werden.

%%%%%%%%%%%%%%%%%%%%%%%%%%%%%%%%%%%%%%%%%%%%%%%%%%%%%%%%%%%%%%%%%%%%%%

\subsection*{Ära Stresemann}
\index{Ära Stresemann}

\nam{Stresemann, Gustav}{Gustav Stresemann} war von August bis
November 1923 Reichskanzler und danach bis zum seinem Tode im Jahre
1929 Außenminister in der Weimarer Republik. Seine Rolle in der
Politik der damaligen Zeit war bestimmen, weswegen man von der
\jar{Ära Stresemann} spricht, sodass sein Ableben zu einem Machtvakuum
führte, das in die Präsidialkabinette (siehe \ref{sec:praes-kab})
mündete.

Seine Ziele war das Wiedererstarken Deutschlands auf Grundlage der
Lösung des Reparationsproblems und der Revision des \Ges{Vertrag von
Versailles}{Vertrags von Versailles}, die mit der Korrektur der
Ostgrenzen, der Vereinigung mit Deutsch-Österreich und dem Schutz der
Auslandsdeutschen einhergingen. Er war also eindeutig ein
Revisionspolitiker, sah aber anders als viele andere Gegner des
Vertrags von Versailles die einzige Möglichkeit der Erreichung seiner
Ziele in der Friedenssicherung durch die Aussöhnung mit Frankreich und
den Eintritt in den \ins{Völkerbund}.\mycite{StresWil}

\subsubsection{Dawes-Plan}
\index{Dawes-Plan}

Da das Deutsche Reich offensichtlich Schwierigkeiten hatte
(\emph{Ruhrkampf}\index{Ruhrkampf}, siehe \ref{sec:krisj-1923}),
seinen Reparationspflichten nachzukommen, legte der US-amerikanische
Bankier \nam{Dawes, Charles}{Charles Dawes} ein Programm vor, dass
einen angemessenen Zahlungsplan sowie die Möglichkeit der
Kreditaufnahme vorsah. Dabei wollten die USA gleichfalls Absatzmärkte
schaffen -- das Geld aus dem Kredit über 800 Millionen Reichsmark
musste in den USA ausgegeben werden -- und den internationalen Markt
stabilisieren, um die Gefahr der Ausbreitung des Kommunismus mindern.

\dat{1924 unterzeichneten} die beteiligten Mächte auf der
\ins{Londoner Konferenz}\Ort{London}{} das betreffende Abkommen,
welches zugleich zur wirtschaftlichen und politischen internationalen
Integration des Deutschen Reiches (\emph{Eintrittskarte} nach Europa)
und damit zur Entspannung in Europa führte. Die ökonomische
Stabilisierung in Deutschlands war auch Ursache für die Scheinblüte der
\jar{Goldenen Zwanziger}.\index{Goldene Zwanziger}\index{Scheinblüte}
Allerdings war das Deutsche Reich nun finanziell von den USA abhängig.
Daher schlug die \ins{Weltwirtschaftskrise} 1929 hier besonders hart
zu.


\subsubsection{Verträge von Locarno}

Die \ges{Verträge von Locarno}\Ort{Locarno}{} wurden \dat{1925} auf
einer Konferenz von Politikern Belgiens, Deutschlands, Frankreichs,
Großbritanniens, Italiens, Polens und der Tschechoslowakei
\dat{geschlossen} und errichteten den Prinzipien der \Nam{Stresemann,
Gustav}{stresemann}schen Außenpolitik folgend ein europäisches
Sicherheitssystem.

\paragraph{Inhalte} Mit dem \ges{Westpakt} verzichteten Belgien, das
Deutsche Reich und Frankreich auf eine gewaltsame Veränderung der
gemeinsamen Grenzen.  Letzteres schloss damit auch Ansprüche auf
\ort{Eupen-Malmedy} und \ort{Elsaß-Lothringen}. Die Einhaltung dieses
Vertrages überwachten Großbritannien und Italien als Garantiemächte.

Der \ges{Rheinpakt} vereinbarte den schrittweisen Abzug der
französischen Truppen aus dem \ort{Rheinland}.

Stresemann wollte ausdrücklich keine Garantie der deutschn Ostgrenze.
Der \ges{Ostpakt} war deshalb lediglich ein Schiedsvertrag mit Polen,
der eine gewaltsame Veränderung der Grenze ausschloss.

\paragraph{Folgen} International bedeuteten die Verträge einen großen
weiteren Schritt zur Rehabilitation des Deutschen Reiches und beendete
seine außenpolitische Isolation endgültig. Sie ermöglichten nämlich
dessen \dat{Aufnahme in den \ins{Völkerbund} 1926} -- ein lange
verfolgtes Ziel Stresemanns. Hier war das Deutsche Reich ein
gleichberechtigter Partner zwischen den anderen Großmächten und hier
konnte der Außenminister der Welt seine Anliegen, insbesondere
bezüglich der Auslandsdeutschen\index{Auslandsdeutsche} vorzutragen.

National wurden die Verträge von rechts und links angefeindet. Jene
beschimpften das Vorgehen Stresemanns und der anderen Beteiligten als
\emph{Erfüllungspolitik}\index{Erfüllungspolitik}, als Schwäche und
\jar{Unehre} des Deutschen Reiches. Diese wetterten ob der angeblichen
\jar{imperialistischen} und \jar{antikommunistischen} Bestrebungen. So
oder so wurden die Verträge von Locarno eine Plattform für Propaganda
gegen die Weimarer Republik.


\subsubsection{Berliner Vertrag}

Wegen der Einbindung des Deutschen Reiches in das westliche
Staatensystem infolge der \ges{Verträge von Locarno}\Ort{Locarno}{}
befürchtete man in der UdSSR, dass sich jenes von den Festlegungen des
\Ges{Vertrag von Rapallo}{Vertrages von Rapallo}\Ort{Rapallo}{}
entfernen könnte.  Deshalb schlossen beide Staaten \dat{1926 den
\ges{Berliner Vertrag}}, indem sie sich die gegenseitige Neutralität
für den Fall eines Krieges zusicherten. Laut \cite[33]{IzpBWeimRep}
bedeutete das insbesondere, dass das Deutsche Reich den Durchmarsch
französischer Truppen bei einem Krieg zwischen der Sowjetunion und
Polen untersagen würde.


\subsubsection{Briand-Kellog-Pakt}

\dat{1928 initiierten} der französische Außenminister \Nam{Briand,
Aristide}{Aristide Briand}, der schon maßgeblich zum Zustandekommen
der \ges{Verträge von Locarno}\Ort{Locarno}{} beigetragen hatte, und
\Nam{Kellogg, Frank Billings}{Frank Billings Kellogg} einen
Vertrag zur Ächtung des Krieges, den \ges{Briand-Kellogg-Pakt}. Unter
den anfänglich 15 Unterzeichnerstaaten befand sich auch das Deutsche
Reich. Nach und nach traten dann weitere Staaten bei, sodass es
schließlich circa 60 wahren.


\subsubsection{Young-Plan}

Die nach wie vor nich gelöste Frage der deutschen Reparationen trieb
die Politiker weiter um, bis schließlich \dat{1929} ein unter der
Leitung des amerikanischen Wirtschaftsfachmann \Nam{Young, Owen}{Owen
Young} ausgearbeiteter Plan \ges{Young-Plan}{} \dat{verabschiedet}
wurde, der folgende Regelungen enthielt:

\begin{enumerate}
\item Die Gesamtsumme der Reparationen wird auf 114 Milliarden
Goldmark festgesetzt.
\item Deutschland soll 59 Jahre lang zahlen.
\item Die Höhe der Raten steigt in den ersten 37 Jahren von 1,7 auf
2,1 Milliarden Goldmark.
\item Ausländische Wirtschaftskontrollen in Deutschland entfallen.
\item Die alliierten Besatzungstruppen werden bereits 1930, also fünf
Jahre früher, aus Deutschland (\ort{Rheinland}) abgezogen.
\end{enumerate}

\mar{Tauch Lausanne irgendwann mit auf?}

\subsection*{Bilanz}\mar{Nur der Form halber.}

\begin{itemize}
\item Deutschland ist rehabilitiert, aber noch nicht vollständig
souverän. 
\item Der \ges{Vertrag von Versailles} (Kriegsschuldparagraph!) ist
nicht revidiert.
\item Die Verträge, insbesondere die von Locarno, bieten eine
Propagandaplattform für Republikfeinde.
\item Der \dat{Tod Stresemanns 1929} hinterlässt ein Machvakuum,
infolgedessen die Zeit der Präsidialkabinette beginnt.
\end{itemize}


\endinput
