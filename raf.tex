\section[Rote Armee Fraktion]{Rote Armee Fraktion\mycite{Dtl70er80er}}
\label{sec:raf}

Die \Ins{RAF, Rote Armee Fraktion}{Rote Armee Fraktion} (RAF) entstand
\dat{1970} aus einigen Aktivisten und Splittern der Studentenbewegung
(siehe \ref{sec:ap-bew}). Sie war eine linksgerichtete
Terrororganisation, die über Verbindungen in der ganzen Welt verfügte
und unter anderem Attentate und Sprengstoffanschläge verübte.

Die ideologischen Grundlagen der Bewegung waren vor allem
kommunistischer Art. Sie sahen Terror und Gewalt als Vorbereitung zur
Revolution, die eine Herrschaft der Arbeiterklasse errichten
sollte.\mar{Belege?}

\subsection*{Kritikpunkte}

\begin{itemize}
\item Wiederaufrüstung
\item Verbot der KPD 1956\mar{Nicht ein bisschen lange her?}
\item Notstandsgesetze
\item Vietnamkrieg
\item Präsenz der USA in der BRD
\item \jar{bundesdeutscher Imperialismus} -- Verhalten der
Oberen\mar{?}
\end{itemize}


\subsection*{Aktionen}

\begin{chronik}
\item[11. Mai 1972] Bombenanschlag auf das Hauptquartier der US-Armee
in Frankfurt
\item[24. April 1975] Stürmung der deutschen Botschaft in Stockholm --
zwei erschossene Diplomaten
\item[7. April] Ermordung des Generalbundesanwalt \Nam{Buback,
Siegfried}{Siegfried Buback}
\item[30. Juli 1977] Ermordung des Vorstandssprechers der
\ins{Dresdner Bank AG} \Nam{Ponto, Jürgen}{Jürgen Ponto}
\item[5. September 1977] Entführung des Präsidenten der
\ins{Bundesvereinigung der Deutschen Arbeitgebervereine} und des
\ins{Bundesverbandes der Deutschen Industrie} \Nam{Schleyer, Hanns
Martin}{Hanns Martin Schleyer} -- später Ermordung
\item[Oktober 1977] Entführung der Lufthansamaschine \Ins{Landshut
(Flugzeug)}{Landshut} durch palästinensische Verbündete
\item[25. Juni 1979] Anschlag auf den NATO-Oberbefehlshaber in Europa
\Nam{Haig, Alexander}{Alexander Haig}
\end{chronik}


\subsection*{Maßnahmen des Staates}

\renewcommand*{\dictumwidth}{0.8\textwidth}
\dictum[\Nam{Schmidt, Helmut}{Helmut Schmidt} am 15. Januar 1979
in einem Interview mit dem \ins{Spiegel}\mycite{BpbAusnZust}]{Ich kann
nur nachträglich den deutschen Juristen danken, daß sie das alles
nicht verfassungsrechtlich untersucht haben.}

\begin{itemize}
\item Erlass von Gesetzen zur Einschränkung von Rechten der
Verteidigung, beispielsweise Möglichkeit der Strafprozessführung in
Abwesenheit des Angeklagten, sowie die Verbrechensverfolgung durch
Staats- und Bundesanwaltschaft erleichterten

\item Kontaktverbot der Gefangenen untereinander und zur Außenwalt
während der Zeit der Entführung Schleyers

\item freiwillige Nachrichtensperre für die Medien
-- Vorwurf der Instrumentalisierung der Medien\mycite{BpbAusnZust}

\item \jar{Lauschangriffe} auf verdächtige Anwälte und Gespräche der
Verteidiger mit den Inhaftierten\mycite{BpbAusnZust}

\item Einführung der \ins{Trennscheibe} zur Trennung von Verteidiger
und Angeklagten nachdem aufgedeckt wurde, dass verschiedene
Gegenstände eingeschmuggelt worden waren

\item teilweise stark fragwürdige Methoden des zuständigen
Justizapparates als Reaktion auf die permanenten Provokationen durch
Angeklagte und Verteidiger\mycite{BpbStammProz}

\item Entlassung von Gefangenen im Austausch gegen Geiseln der RAF
\end{itemize}


\subsection*{Bewertung anhand des modernen Demokratiebegriffs}

Die außergewöhnliche Situation, die die Terroristen der RAF mit
Absicht herbeigeführten hatten, stellte den Rechtsstaat Bundesrepublik
Deutschland auf eine äußerst harte Probe. Vor diesem Hintergrund
müssen die Maßnahmen gesehen werden, die der Staat ergriff und die
die Grenzen eben jenes Rechtsstaates überschritten, die Verfassung
also verletzten (vgl. das obige Zitat Schmidts).

Der Umgang mit dem RAF-Terror und den RAF-Terroristen genügt also
dem modernen Demokratiebegriff nicht, sind jedoch im Kontext des
geringen Handlungsspielraumes zu sehen, der den Verantwortlichen zur
Verfügung stand.

\endinput

- Entlarvung des Rechtsstaats
- Grenzen des Rechtsstaats kaum durchbrochen

