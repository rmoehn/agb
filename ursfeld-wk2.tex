\section{Ursachenfeld des Zweiten Weltkriegs}

\begin{aufgabe}
Charakterisieren Sie die Motive und Rahmenbedingungen, die zum Zweiten
Weltkrieg führten! (Aufgabe bezieht sich auf \cite[47
oben]{IzpBNatsoz2}

Prüfen Sie die Argumente auf Vollständigkeit! 
\end{aufgabe}

\begin{itemize}
\item \Nam{Hitler, Adolf}{Hitler}s Wille zum Krieg

\item durch den Willen zum Krieg bewusst verschuldete wirtschaftliche
und soziale Zwangslagen -- scheinbare Legitimation für
politisch-militärische und wirtschaftliche Führungsgruppen

\item nationalpolitische Erfolge Hitlers -- wenig Anreiz zum
Widerstand

\item Versuchung der nationalkonservativen Eliten durch den
Nationalsozialismus führt zur Verwischung der Trennlinien zwischen dem
Großmachtdenken der Nationalkonservativen und der Eroberungspolitik
der Nationalsozialisten
\begin{itemize}
\item Militär: Erhaltung bedrohter sozialer Einflusspositionen,
Aufrüstung, großdeutsche Expansion
\item Wirtschaft: ökonomische Aufwärtsentwicklung und Gewinnsteigerung
durch Rüstungsprogramme, Osterweiterung des deutschen Wirtschaftsraums
-- Hinnahme des Machtverlusts
\end{itemize}

\item permanente innere und äußere Krise der europäischen Staaten,
egozentristische Politik der Staaten nach der Weltwirtschaftskrise --
schwacher Widerstand von außen, keine kollektive Konfliktregelung,
beinahe anarchische internationale Politik
\begin{itemize}
\item Appeasement-Politik Großbritanniens
\item Verzicht Frankreichs auf aktive Außenpolitik
\item Pakt der Sowjetunion mit Hitler
\item Schaukelpolitik Polens zwischen Ost und West
\item Anpassungsbereitschaft der ostmittel- und südosteuropäischen
Staaten 
\end{itemize}

\item Ideologisierung der Poltik, Ideologie staatstragend (Italien,
Japan, Deutschland)

\item Gespühr Hitlers für Schwächen des Gegners

\item Fehleinschätzung der Politik Hitlers -- nicht wirtschaftliche
Vorteile, sondern Sozialdarwinismus -- fehlgehende Appeasement-Politik
-- fehlgehende Appeasement-Politik
\end{itemize}
